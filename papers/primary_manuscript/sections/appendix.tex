\section{Well-mixed communicative fraction with symmetry breaking}
Here, we follow the steady-state communicative fraction derivation
in the previous EK study~\citep{tripp2022evolutionary}
and extend it to include a payoff asymmetry $\alpha$.
To keep the derivation tractable,
we only consider the well-mixed case.
We will assume a low mutation rate
so that any mutation fixates prior to the next mutation.
We then consider a population with one communicative species $E = (C, \phi_i)$
and one non-communicative species $F = (N, \phi_j)$.
Our first goal is to calculate
the (communicative) $\rho_E$ and (non-communicative) $\rho_F$
fixation probabilities of a single $E$ or $F$ invader, respectively.
First, we use our $\alpha$-dependent payoff matrix, \cref{eq:payoff-matrix},
to calculate the $\alpha$-dependent average payoff
of each strategy across the entire population,
\begin{align}
  \pi_E(k) &= \ab(\frac{k-1}{n-1}) \pab{B_0 - c}
                + \ab(\frac{n-k}{n-1}) \ab(\beta(\Delta\phi) 2\alpha - c) \\
           &= \frac{1}{n-1}
               \ab(k \ab(B_0 - 2 \alpha \beta(\Delta\phi))
                 + 2 \alpha n \beta(\Delta\phi) - B_0 - (n-1) c)
\end{align}
and
\begin{align}
  \pi_F(k) &= \ab(\frac{k}{n-1}) \ab(\beta(\Delta\phi) 2 (1-\alpha))
                + \ab(\frac{n-k-1}{n-1}) \ab(0) \\
           &= \ab(\frac{k}{n-1}) 2 (1-\alpha) \beta(\Delta\phi)
\end{align}
Here, $k \in [0,n]$ is the number of $E$-players
$\Delta \phi$ is the phase difference between the $E$ and $F$
strategies,
and we defined $\beta(\Delta \phi) = \beta_0 f(\Delta \phi)$.
Note that the first term of $\pi_E$ corresponds
to the $E$-$E$ payoff, which has $\Delta \phi = 0$,
hence $B_0 f(\Delta \phi) = B_0$.
As mentioned in the main text,
each node has an exponential total fitness
$f_E = \exp(\delta (n-1) \pi_E)$
or
$f_F = \exp(\delta (n-1) \pi_F)$,
depending on its strategy,
with selection strength $\delta$.

In this paragraph, we summarize the relevant steps;
see the prior work~\citep{tripp2022evolutionary}
for more detailed derivations.
We now model the Moran process
as a Markov chain where each state $i \in \Bab{0,1,\ldots,n}$
is the number of (communicative) $E$ strategies.
Interestingly, we note that since the Moran process changes, at most,
one strategy per time step,
this Markov chain has the same fixation probabilities as a gambler's ruin problem
(though it has different fixation times due to the possibility of zero-length steps).
We denote $x_i$ as the probability that state $i$ will fixate
to state $n$ with all players using strategy $E$.
Then, the $E$ and $F$ fixation probabilities defined above are given by
$\rho_E = x_1$ and $\rho_F = 1 - x_{n-1}$.
Now, the Markov chain is described by the transition probability $p_{i,j}$
from state $i$ to state $j$ as
\begin{align*}
  p_{0,0} &= 1, \\
  p_{n,n} &= 1, \\
  p_{i,i-1} &= \frac{i}{n} \frac{(n-i) f_F(i)}{i f_E(i) + (n-i) f_F(i)} \\
  p_{i,i+1} &= \frac{n-i}{n} \frac{i f_E(i)}{i f_E(i) + (n-i) f_F(i)} \\
  p_{i,i} &= 1 - p_{i,i+1} - p_{i,i-1}
\end{align*}
Then, we obtain a recurrence relation for $x_i$ by conditioning
on the outcome of the first step:
\begin{align*}
  x_i &= x_{i-1} p_{i,i-1} + x_i p_{i,i} + x_{i+1} p_{i,i+1} \\
\end{align*}
with boundary values
\begin{align*}
  x_0 &= 0 \\
  x_1 &= 1
\end{align*}
Using the $p_{i,i} = 1 - p_{i,i+1} - p_{i,i-1}$ relation from above
to replace the $p_{i,i}$ term gives
\begin{equation*}
  \pab{x_i - x_{i-1}} \gamma_i = \pab{x_{i+1} - x_i}
\end{equation*}
Then, defining $y_i \coloneqq x_i - x_{i-1}$,
we find $y_1 = x_1$ and $y_{i+1} = \gamma_i y_i$,
yielding $y_i = \prod_{j=1}^{i-1} \gamma_j x_1$ for $i \ge 2$.
Finally, we form a telescoping sum to yield
\begin{equation*}
  1 - x_1 = x_n - x_1
  = \sum_{i=1}^{n-1} y_{i+1}
  = \sum_{i=1}^{n-1} \prod_{j=1}^i \gamma_j x_1
\end{equation*}
Solving this for $x_1$ gives
\begin{equation}
  \rho_E = x_1 = \frac{1}{1 + \sum_{i=1}^{n-1} \prod_{j=1}^i \gamma_j}
  \label{eq:comm_fixation_prob}
\end{equation}
Likewise, $x_i = \sum_{j=1}^i y_i = \sum_{j=0}^{i-1} y_{i+1}$, so
\begin{equation*}
  x_i
  = x_1 + \sum_{j=1}^{i-1} \prod_{k=1}^j \gamma_k x_1
  = \frac{1 + \sum_{j=1}^{i-1} \prod_{k=1}^j \gamma_k}
    {1 + \sum_{j=1}^{n-1} \prod_{k=1}^j \gamma_k}
\end{equation*}
Therefore, we find
\begin{equation*}
  \rho_F = 1 - x_{n-1}
  = 1 - \frac{1 + \sum_{j=1}^{n-2} \prod_{k=1}^j \gamma_k}
    {1 + \sum_{j=1}^{n-1} \prod_{k=1}^j \gamma_k}
  = \frac{\prod_{k=1}^{n-1} \gamma_k}{1 + \sum_{j=1}^{n-1} \prod_{k=1}^j \gamma_k}
\end{equation*}
This implies
\begin{equation}
  \frac{\rho_F}{\rho_E} = \prod_{k=1}^{n-1} \gamma_k
  \label{eq:fixation_prob_ratio}
\end{equation}
Therefore, we've found $\rho_E$ and $\rho_F$ for a population
with two fixed strategies.

Now, we need to derive the form of the $\gamma_k$,
which will include the new asymmetry factor $\alpha$:
\begin{align*}
  \gamma_k &= \frac{f_F(k)}{f_E(k)} \\
           &= \exp[\delta (n-1) (\pi_F(k) - \pi_E(k))]
           \\
           &= \exp\ab[\delta \ab(
    k 2 (1-\alpha) \beta(\Delta\phi)
    - k \ab(B_0 - 2 \alpha \beta(\Delta\phi))
                 - 2 \alpha n \beta(\Delta\phi) + B_0 + (n-1) c)] \\
      &= \exp\ab[\delta \ab(
    \ab(2 \beta(\Delta\phi) - B_0) k
                 + B_0 - 2 \alpha n \beta(\Delta\phi) + (n-1) c)]
\end{align*}
We also calculate
\begin{equation}
  \begin{aligned}
    \prod_{k=1}^j  \gamma_k
      &= \prod_{k=1}^j \exp\ab[\delta \ab(
      \ab(2 \beta(\Delta\phi_{qr}) - B_0) k
      + B_0 - 2 \alpha n \beta(\Delta\phi_{qr}) + (n-1) c)]
      \\
      &= \exp\ab[\delta \sum_{k=1}^j \ab(
      \ab(2 \beta(\Delta\phi_{qr}) - B_0) k
      + B_0 - 2 \alpha n \beta(\Delta\phi_{qr}) + (n-1) c)]
      \\
      &= \exp\ab[\delta \ab(
      \ab(2 \beta(\Delta\phi_{qr}) - B_0) \frac{j(j+1)}{2}
      + j \ab(B_0 - 2 \alpha n \beta(\Delta\phi_{qr}) + (n-1) c))]
      \\
      &= \exp\ab[\delta \ab(
      \ab(\beta(\Delta\phi_{qr}) - \frac{B_0}{2}) j^2
      + j \ab(\frac{B_0}{2} + \beta(\Delta\phi_{qr}) (1 - 2 \alpha n)  + (n-1) c))]
  \end{aligned}
  \label{eq:gamma_prod}
\end{equation}
Additionally, we note that substituting
\cref{eq:gamma_prod} into \cref{eq:fixation_prob_ratio}
with $j=n-1$ shows that $\rho_F/\rho_E$
depends of $\Delta \phi_{qr}$.
This is in contrast to the $\alpha=0.5$ case~\citep{tripp2022evolutionary}
where $\rho_F/\rho_E$ simplifies to the $\Delta \phi_{qr}$-independent form
$\rho_F/\rho_E = \exp \ab(\delta \ab[(n-1)c - B_0 (n-2)/2])$.

Next, we consider the long-time dynamics incorporating more than two strategies.
By again using the low-mutation-rate assumption,
we can assume that any mutation either fixates
or dies out before the next mutation.
Therefore, the system evolves from one homogeneous state to another
using the probabilities we just derived.
We can model this behavior with a new Markov chain
using the $m=20$ phases and two communicative strategies (C and N)
to give a $2m$-dimensional state space:
$\Bab{(C,\phi_1), \ldots (C,\phi_m), (N,\phi_1), \ldots (N,\phi_m)}$.
Then, the transition probability $\rho_{NC,\Delta \phi_{qr}}$
from state $(N,\phi_q)$ to state $(C,\phi_r)$ is just $\rho_E$ above.
Similarly, the probability $\rho_{CN,\Delta \phi_{rq}}$
that a $C,\phi_r)$ state is successfully invaded by
an $(N,\phi_q)$ state is simply $\rho_F$ above.
By the rotation symmetry of $\phi$, each $\phi_i$ has the same probability of fixation.
Therefore, consider an arbitrary $\phi_q$ and denote its communicative fixation probability as
$s_1$ and its non-communicative fixation probability $s_2$.
We can calculate the ratio of these probabilities
by incorporating the fixation probabilities against all other $\phi_r$ phases:
\begin{align}
  \frac{s_2}{s_1} &= \frac
    {\sum_{r=1}^{m} \rho_{CN,\Delta \phi_{rq}}}
    {\sum_{r=1}^{m} \rho_{NC,\Delta \phi_{qr}} }
  \\
  &=
  \frac
  {\sum_{r=1}^{m} \rho_{NC,\Delta \phi_{qr}}
    \exp \Bab{
      \delta (n-1)
      \bab{
        (n-1) c + n \beta(\Delta \phi_{qr}) (1 - 2 \alpha)
        - \frac{n-2}{2} B_0
      }
    }
  }
  {\sum_{r=1}^{m} \rho_{NC,\Delta \phi_{qr}}}
  \\
  &=
  \bab{
    \frac
    {\sum_{r=1}^{m} \frac
      {\exp \Bab{ \delta (n-1)
        \bab{
          (n-1) c + n \beta(\Delta \phi_{qr}) (1 - 2 \alpha) - \frac{n-2}{2} B_0
        }}
      }
      {1 + \sum_{j=1}^{n-1} \exp\Bab{
        \delta \bab{
         \pab{\beta(\Delta \phi_{qr}) - B_0/2} j^2
         + j \pab{B_0/2 + \beta(\Delta \phi_{qr}) (1 - 2 \alpha n) + (n-1) c}
        }}
      }
    }
    {\sum_{r=1}^{m} \frac{1}
      {1 + \sum_{j=1}^{n-1} \exp\Bab{
        \delta \bab{
         \pab{\beta(\Delta \phi_{qr}) - B_0/2} j^2
         + j \pab{B_0/2 + \beta(\Delta \phi_{qr}) (1 - 2 \alpha n) + (n-1) c}
        }}
      }
    }
  }
  \label{eq:full_analytic_frac}
\end{align}
The first equality used \cref{eq:fixation_prob_ratio}
and the second equality used \cref{eq:comm_fixation_prob}.
Unlike in the $\alpha = 0.5$ symmetric case~\citep{tripp2022evolutionary},
the ratio $\rho_F/\rho_E$ depends on $\Delta \phi_{qr}$,
so we cannot factor the exponential component out of the sum
and cancel the $\rho_{CN,\Delta \phi_{qr}}$ terms.
However, we can asymptotically expand $s_2/s_1$ for small
$B_0/c \coloneqq \epsilon \ll 1$;
additionally, since all of our simulations use $\beta \propto B$,
we also assume $\beta/c \sim \epsilon \ll 1$.
Finally, to simplify the calculation,
we define the asymptotic expansion of
$\rho_{NC,\Delta \phi_{qr}} \coloneqq
\rho_{NC,\Delta \phi_{qr}}^{(0)}
+
\rho_{NC,\Delta \phi_{qr}}^{(1)}
+
\order{\epsilon^2}
$
where
$
\rho_{NC,\Delta \phi_{qr}}^{(i)} \sim \epsilon^i
$
depends only on terms of total order $i$ in $B_0/c$ and $\beta/c$.

Then, to first order in $\epsilon$
we have
\begin{align*}
  \frac{s_2}{s_1}
  &=
  \frac
  {\sum_{r=1}^{m} \rho_{NC,\Delta \phi_{qr}}
    \exp \Bab{
      \delta (n-1)
      \bab{
        (n-1) c + n \beta(\Delta \phi_{qr}) (1 - 2 \alpha)
        - \frac{n-2}{2} B_0
      }
    }
  }
  {\sum_{r=1}^{m} \rho_{NC,\Delta \phi_{qr}}}
  \\
  &=
  e^{\delta (n-1)^2 c}
  \Biggl\{
  \\
  &\qquad
  \frac
  {\sum_{r=1}^{m}
    \pab{
      \rho_{NC,\Delta \phi_{qr}}^{(0)}
      +
      \rho_{NC,\Delta \phi_{qr}}^{(1)}
    }
    \Bab{1 +
      \delta (n-1)
      \bab{
        n \beta(\Delta \phi_{qr}) (1 - 2 \alpha)
        - \frac{n-2}{2} B_0
      }
    }
  }
  {\sum_{r=1}^{m} \pab{
    \rho_{NC,\Delta \phi_{qr}}^{(0)}
    +
    \rho_{NC,\Delta \phi_{qr}}^{(1)}
  }}
  \Biggr\}
  \\
  &\qquad
  + \order{\epsilon^2}
  \\
  &=
  e^{\delta (n-1)^2 c}
  \Biggl\{
  \frac
  {\sum_{r=1}^{m}
    \pab{
      \rho_{NC,\Delta \phi_{qr}}^{(0)}
      +
      \rho_{NC,\Delta \phi_{qr}}^{(1)}
    }
  }
  {\sum_{r=1}^{m} \pab{
    \rho_{NC,\Delta \phi_{qr}}^{(0)}
    +
    \rho_{NC,\Delta \phi_{qr}}^{(1)}
  }}
  \\
  &\qquad
  +
  \frac
  {\sum_{r=1}^{m} \rho_{NC,\Delta \phi_{qr}}^{(0)}
    \delta (n-1)
    \bab{
      n \beta(\Delta \phi_{qr}) (1 - 2 \alpha)
      - \frac{n-2}{2} B_0
    }
  }
  {\sum_{r=1}^{m} \rho_{NC,\Delta \phi_{qr}}^{(0)}
  }
  \Biggr\}
  + \order{\epsilon^2}
  \\
  &=
  e^{\delta (n-1)^2 c}
  \Bab{
    1
    - \delta (n-1) \frac{n-2}{2} B_0
    +
    \frac{\delta n (n-1)}{m} (1 - 2 \alpha)
    \sum_{r=1}^{m} \beta(\Delta \phi_{qr})
  }
  + \order{\epsilon^2}
  \\
  &=
  \exp \Bab{\delta (n-1) \bab{
    (n-1) c
    - \frac{n-2}{2} B_0
    +
    \frac{n (1 - 2 \alpha)}{m}
    \sum_{r=1}^{m} \beta(\Delta \phi_{qr})
  }}
  + \order{\epsilon^2}
  \\
  &=
  \exp \Bab{ \delta (n-1) \bab{
    (n-1) c
    - \frac{n-2}{2} B_0
    +
    \frac{n (1 - 2 \alpha)}{2} \beta_0
  }}
  + \order{\epsilon^2}
\end{align*}
Where, in the last line, we used
the definition of
$\beta(\Delta \phi_{qr}) = \beta_0 \bab{1 + \cos(\Delta \phi_{qr})}/2$
to write
$\sum_{r=1}^m \beta(\Delta \phi_{qr})
= \beta_0 \sum_{r=1}^m \Bab{1 + \cos\bab{2 \pi (q-r)/m}}/2
= m \beta_0/2$
since the cosine sum gives zero.

Finally, since each $\Delta \phi_i$ is equally likely,
the probability of fixing to \emph{any}
communicative state is $m s_1$;
likewise the non-communicative fixation probability is $m s_2$.
Then, since the process must eventually absorb to either $C$ or $N$,
we have
$m s_1 + m s_2 = 1$, giving the probability of communicative fixation as
\begin{equation}
  m s_1 = \frac{1}{1 + s_2/s_1}
  \label{eq:full_analytic}
\end{equation}
with $s_2/s_1$ given by \cref{eq:full_analytic_frac},
and the asymptotic, small-$\epsilon$ approximation given by
\begin{equation}
  m s_1 = \frac{1}{1 + s_2/s_1}
  \approx
  \frac{1}
  {1 + \exp \Bab{\delta (n-1) \bab{
    (n-1) c
    - \frac{n-2}{2} B_0
    +
    \frac{n (1 - 2 \alpha)}{2} \beta_0
    }}
  }
  \label{eq:analytic_first_term}
\end{equation}

\section{Plurality game type criteria}
Now, we wish to identify criteria in which
non-synchronized game types can become the plurality.
While having all or most of the players with the same phase
easily makes the near-synchronized $\Delta \phi \approx 0$
game types the plurality,
it is more challenging to find conditions
when the $\Delta \phi = 0$ game-type is \emph{not} the plurality.
This can occur when the number of edges \emph{between} groups
is greater than the number of edges \emph{amongst} groups.
In \cref{sec:discussion}, we described conditions under which
the near-synchronized $\Delta \phi \approx 0$ games would \emph{not} be
the most common game type for a complete graph.
Here, we will show the derivations leading to those conditions.

\subsection{Multiple equal sized groups}\label{sec:multiple_equal_groups}
Let us assume the $N$ total players are evenly divided
into $p$ groups.
Then, the non-synchronized $\Delta \phi \neq 0$ game types
arise from the edges between the $p$ groups.
Thus, we maximize the proportion of these $\Delta \phi \neq 0$ games
by having the same $\Delta \phi$ between as many groups as possible.
In other words, we want the phases of the $p$ groups
to be evenly distributed on the unit circle with phases $2 \pi j/p$
for $j \in [0,p-1]$.

Since there are $N/p$ players in each group,
the number of edges within a group is $(N/p)(N/p - 1)/2$.
Since there are $p$ groups, the number of $\Delta \phi = 0$ edges are
$N (N/p - 1)/2$.

The next most-frequent game type will be that corresponding to
$\Delta \phi = 2 \pi/p$:
that is, games between adjacent groups on the unit circle.
Since there are $p$ groups,
$N/p$ nodes in each group,
and each node connects to all $N/p$ nodes in the group
$2 \pi/p$ phase ahead of it,
the total number of $\Delta \phi = 2 \pi/p$ connections
are $(N/p)^2 p = N^2/p$.

It is easy to see that the number of $\Delta \phi = 2\pi/p$ edges,
$N^2/p$,
is always greater than the number of $\Delta \phi = 0$ edges,
$N(N/p - 1)/2$.
Therefore, when there are $p$ equal sized groups
with phases differing by $2\pi/p$,
the near-synchronized $\Delta \phi \approx 0$ games will \emph{not} be the plurality.

\subsection{Two unequal groups}
Next, we consider the case where we have two groups, $A$ and $B$,
with different phases and unequal sizes.
We denote the number of players in group $A$ by $N_A$.

\subsubsection{Complete graphs}\label{sec:two_unequal_groups}
For a complete graph,
the number of edges amongst group $A$ is $N_A (N_A - 1)/2$.
Likewise, the number of edges amongst group $B$ is
$(N-N_A) (N - N_A - 1)/2$.
To find the number of edges between groups $A$ and $B$,
we observe that each of the $N_A$ nodes in group $A$
connects to each of the $N - N_A$ nodes in group $B$,
giving $N_A (N - N_A)$ total edges.

Thus, the number of $\Delta \phi = 0$ edges
equal the number of $\Delta \phi \neq 0$ edges when
\begin{equation*}
  N_A (N_A - 1)/2 + (N - N_A) (N - N_A - 1)/2 = N_A (N - N_A)
  \implies \frac{N_A}{N} = \frac{1}{2} \pm \frac{1}{2 \sqrt{N}}
\end{equation*}
Therefore, when the fraction of players in one group is greater than
$\frac{1}{2} + \frac{1}{2\sqrt{N}}$,
then the synchronized $\Delta \phi = 0$ game will be the plurality.
But when the fraction of players in each group is within
$\pab{1/2 - 1/2\sqrt{N}, 1/2 + 1/2\sqrt{N}}$,
then the non-synchronized $\Delta \phi \neq 0$ game is the majority.

\subsubsection{Incomplete graphs}\label{sec:two_unequal_groups_incomplete}
To study incomplete graphs,
consider a random $\overline{d}$-regular graph
with two populations distributed spatially randomly.
Then, this is equivalent to only keeping complete-graph edges
with probability $\omega = \overline{d}/(N-1)$.
Thus, the average number of edges
amongst group $A$ is $N_A (N \omega - 1)/2$,
amongst group $B$ is $(N-N_A)\pab{(N-N_A)\omega - 1}/2$,
and between groups $A$ and $B$ is $N_A (N-N_A) \omega$.
Setting up the same inequality and simplifying yields
\begin{equation*}
  \frac{N_A}{N} = \frac{1}{2} \pm \frac{1}{2 \sqrt{\omega N}}
  = \frac{1}{2} \pm \frac{1}{2} \sqrt{\frac{N-1}{N \overline{d}}}
\end{equation*}
