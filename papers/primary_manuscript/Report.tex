%! TeX program = lualatex
\documentclass[pdflatex,lineno,referee,sn-nature]{sn-jnl}

% Silence warning from caption about unknown document class
\usepackage{silence}
\WarningFilter*{caption}{Unknown document class}

% Memoize tikz drawings
\usepackage{nomemoize}
%\usepackage{memoize}
\mmzset{
  padding=0pt, % Remove padding around figure for journal submission
  context={fsize={\csname f@size\endcsname}}, % Add font size to context
  verbatim,
}

% Import custom style file containing common packages and options
\usepackage{preamble}

% Include definitions and custom commands
% Define binomial nomenclatures
\setabbreviationstyle[species]{long-em-short-em}
\glssetcategoryattribute{species}{nohyper}{true}
\newabbreviation[category={species}]{celegans}{C. elegans}{Caenorhabditis elegans}
\newabbreviation[category={species}]{dmelanogaster}{D. melanogaster}{Drosophila melanogaster}

% Define custom macros
\newcommand{\im}{\mathrm{i}}
\DeclareMathOperator{\cov}{cov}

% Define custom caption separator
\DeclareCaptionLabelSeparator{bar}{|}
\captionsetup{labelsep=bar}

% Prevent floats from leaving subsection
%\pretocmd{\section}{\FloatBarrier}{}{}
%\pretocmd{\subsection}{\FloatBarrier}{}{}
%\pretocmd{\subsubsection}{\FloatBarrier}{}{}

% Add comma separator for numbers
\sisetup{group-minimum-digits=3}
\sisetup{group-separator = {,}}

\defasforeign[aka]{a.k.a.} % Define abbreviation of aka


% Import tikz options and preamble
% Define pgfplots classes
\pgfplotsset{simulation scatter/.style={only marks, blue}}
\pgfplotsset{full theory line/.style={orange}}
\pgfplotsset{approx theory line/.style={purple, dashed}}

\pgfplotsset{fraction communicative chart/.style={
        ymin={0},
        ymax={1},
        ytick={0.0,0.5,1.0},
        grid,
        table/x=B0,
        table/y=communicative_fraction,
        cycle list={
          simulation scatter,
          full theory line,
          approx theory line,
        },
        enlarge y limits=0.05,
    }
}

\pgfplotsset{
  harmony/.style={fill=Paired-G,draw=Paired-G},
  chicken/.style={fill=Paired-F,draw=Paired-F}, % exponential fixation time
  battle/.style={fill=Paired-E,draw=Paired-E}, % exponential fixation time
  hero/.style={fill=Paired-H,draw=Paired-H}, % exponential fixation time
  compromise/.style={fill=Paired-C,draw=Paired-C},
  concord/.style={fill=Paired-D,draw=Paired-D},
  staghunt/.style={fill=Paired-B,draw=Paired-B},
  dilemma/.style={fill=Paired-K,draw=Paired-K},
  deadlock/.style={fill=Paired-A,draw=Paired-A},
  assurance/.style={fill=Paired-J,draw=Paired-J},
  coordination/.style={fill=Paired-I,draw=Paired-I},
  peace/.style={fill=Paired-L,draw=Paired-L},
  all_communicative/.style={fill=lightgray,draw=lightgray},
  all_noncommunicative/.style={fill=darkgray,draw=darkgray},
}


\begin{document}

\title{Evolutionary Kuramoto Dynamics unravels origins of chimera states
in neural populations}

\author{
\href{https://orcid.org/0000-0003-3039-172X}{Thomas Zdyrski}$^{1}$,
\href{https://orcid.org/0000-0002-0281-2868}{Scott Pauls}$^{1}$,
and
\href{https://orcid.org/0000-0001-8252-1990}{Feng Fu}$^{1}$
}

\affil{$^{1}$Department of Mathematics, Dartmouth College, Hanover, NH 03755
}

%\keywords{xxxx, xxxx, xxxx}
%\pw{}

\abstract{
  % Basic introduction
  Chimera states of networked oscillators
  are configurations with a simultaneous coexistence
  of synchronized and incoherent populations.
  These states have been observed
  in various brain systems~\citep{santos2017chimera,bansal2019cognitive},
  and may be a key component
  in human cognition~\citep{bansal2019cognitive}.
  % Detailed background
  Recent work~\citep{deng2024chimera} has investigated
  chimera-like states in \gls{dmelanogaster}
  using a Kuramoto model
  of sinusoidally-coupled oscillators,
  a frequent model for complex neuron behavior~\citep{cabral2011role}.
  % General problem
  However, the small-scale evolutionary factors
  that could give rise to such chimera-like states
  remains poorly understood.
  % Key takeaway
  In this work,
  we use evolutionary graph theory
  to connect small-scale, single-neuron payoffs
  with large-scale, chimera-like states
  on a graph of the \gls{celegans} connectome.
  % Main results
  We extend an evolutionary Kuramoto model from complete graphs
  to weighted, directed graphs.
  The \gls{celegans} network exhibits much stronger disorder
  than the well-mixed population.
  Additionally, we find that the network has
  a low metastability ($\lambda < \num{0.01}$)
  but a relatively large chimera-like index ($\xi \approx \num{0.13}$).
  % General context
  Our results connect individual neuron fitness
  and non-trivial connectivity topology
  to synchronization via Kuramoto-like mechanisms.
  % Broader perspective
  Chimera-like states
  in this \gls{celegans} network of simple neurons
  suggest that brain topology
  is a primary driver of chimera-like brain states.
  Understanding the role of neuronal structure in brain dynamics
  is an active area of research,
  and this evolutionary graph theory Kuramoto model
  could be a valuable tool for isolating
  the influence of graph structure on brain dynamics.
}

\maketitle
%\tableofcontents

\section{Introduction}
%\subsection{Impact of structure on game evolution}
Evolutionary game theory (EGT) is the application
of game theory to evolving populations
of non-rational players with fixed strategies.
This tool is particularly well-suited to studying how microscopic interaction rules
give rise to macroscopic population patterns
such as cooperation~\citep[\eg{}][and references therein]{sigmund1999evolutionary}
and has been applied to fields from international politics to ecology
to protein folding~\citep[\cf{}][and references therein]{traulsen2023future}.
Evolutionary \emph{graph} theory places evolutionary game theory
on graphs to investigate the role of structure on population evolution.
Numerous studies have found that the introduction of structure
can qualitatively change the population's evolution.
For instance, cooperation can be enhanced by
decreasing the average number of edges~\citep{ohtsuki2006simple}
or having an intermediate proportion
of unidirectional edges~\citep{su2022evolution},
while cooperation is suppressed
for edge-weighted graphs~\citep{bhaumik2024constant}.
Thus, the complex influences of
incompleteness, directedness, and weightedness combine
to determine the variety of outcomes for evolving populations.

%\subsection{Evolutionary Kuramoto dynamics}
Previously, Tripp, Fu, and Pauls \citep{tripp2022evolutionary} (abbrevitated \tripp{} hereafter)
created an EGT model
to capture the behaviour of Kuramoto oscillators
using game-theoretic approaches.
Kuramoto oscillators are networked groups of oscillators
where the coupling strength depends sinusoidally
on the oscillators' phase difference.
Kuramoto networks are studied for their ability to display
tunable synchronization.
This synchronization has also made them popular
as a model for neuron behaviour~\citep{cabral2011role,deng2024chimera},
While other studies~\citep{antonioni2017coevolution}
have motivated this Kuramoto connection
by modeling the continuous phase parameter $\phi$ with EGT,
those models were limited to only the prisoner's dilemma game type.
However, \tripp{} discretized the phase parameter,
allowing them to incorporate games beyond the prisoner's dilemma.
It also allowed them to investigate how cooperation might arise
amongst collections of communicative and non-communicative neurons
in the suprachiasmatic nucleus.
Though permanent cells such as neurons don't reproduce,
the activation features (\ie{} strategies) can still
be studied with evolutionary game theory
if learning is treated as
an evolutionary process~\citep{cohen2009evolutionary}.
This evolutionary Kuramoto model
supplemented the usual communicative/non-communicative
strategy pair with an independent dimension representing the player's ``phase''.
\tripp{} assumed a well-mixed model of players
on a complete graph, and the players' payoffs were modulated
by a sinusoidal-dependence on their relative phases,
representing the classic Kuramoto oscillator coupling.
This study showed the emergence of various communicative (C)
and non-communicative (N) regimes depending on the relative sizes
of the joint CC benefit $B_0$, the mixed CN/NC benefit $\beta_0$,
and the cost $c$ imposed on communicators.

%\subsection{Chimera states}
One intriguing aspect of neuron oscillations
is the observation of chimera states~\citep
[\eg{}][and references therein]{majhi2019chimera}.
These states exhibit the simultaneous existence
of coherent and incoherent phases~\citep{abrams2004chimera}.
Chimera states have been proposed
as a key component of human cognitive organization~\citep{bansal2019cognitive},
a facilitator of spiking and bursting phases~\citep{santos2017chimera},
and an outcome of modular networks~\citep{hizanidis2016chimera}.
Despite the observed importance of these chimera states,
the factors that give rise to coherent/incoherent coexistence
remain incompletely characterized,
and the origins of chimera states
remain particularly poorly understood.

%\subsection{\glsfmtshort{celegans} connectome}
The nematode \gls{celegans}
is a model organism in neuroscience due to
its relatively simple brain connectome~\citep{cook2019whole}.
With a connectome containing just \num{302} neurons,
the hermaphroditic \gls{celegans} is far more tractable to study
than the nearly \num{100} billion neurons
in the human brain~\citep[\eg][]{von2016search}.
Despite their simplicity, models of the \gls{celegans} brain
still display a wide array of complex phenomena including
topologically-central rich clubs
crucial to motor neurons~\citep{towlson2013rich},
phenomenological connections to control theory~\citep{yan2017network},
and chimera states~\citep{hizanidis2016chimera}.
Therefore, this work will investigate the applicability
of evolutionary Kuramoto dynamics and chimera states
to the \gls{celegans} connectome model.

%\subsection{Outline}
In this paper,we will generalize
the Evolutionary Kuramoto (EK) model
of \tripp{} by adding an asymmetry between
the communicator and non-communicator payoffs
and applying it to incomplete, directed, and weighted graphs.
These generalizations will allow us to apply the EK model
to the \gls{celegans} connectome
and analyze its communicativeness properties.
The model's simulation statistics
will allow us to connect
the individual neurons' payoffs
to the large-scale communicativeness properties
while also investigating the influence
the \gls{celegans} connectome shape on chimera-like states.

\section{Results}
\label{sec:results}

\subsection{Model Enhancements}
\begin{figure}
  \centering
  \begin{nomemoize} % NiceMatrix uses the tikz option `remember picture`;
                    % this automatically aborts memoization, so disable it to save time
    \begin{subcaptiongroup}
      \stackinset{l}{10pt}{t}{2.5in}%
        {\phantomcaption\label{fig:payoff_matrix}\captiontext*{}}{%
      \stackinset{l}{2.5in}{t}{10pt}%
        {\phantomcaption\label{fig:player_interactions}\captiontext*{}}{%
      \stackinset{l}{10pt}{t}{10pt}%
        {\phantomcaption\label{fig:graph_well-mixed}\captiontext*{}}{%
      {\includestandalone{tikz/model-setup}}%
    }}}
    \end{subcaptiongroup}
  \end{nomemoize}
  \caption{
    \textbf{
      Evolutionary Kuramoto dynamics with weighted neural connectivity.
    }
    \protect{\subref{fig:graph_well-mixed}}
    The graph of the well-mixed population showing $N=20$ players,
    with each pair players being connected by a directed edge in each direction.
    \protect{\subref{fig:player_interactions}}
    The connectivity between two sample players, $i$ and $j$,
    showing directed, weighted edges $w_i$ and $w_j$.
    Each player has a strategy-phase pair $(s, \phi)$,
    with the strategy being either communicative ($C$) or non-communicative ($N$)
    and the phase being $\phi = 2\pi k/m$ with $m$ the number of phases
    and $k \in 0,\ldots,m-1$.
    \protect{\subref{fig:payoff_matrix}}
    The payoff matrix shows the reward the row-player $(s_i, \phi_i)$
    receives after playing a game with the column-player $(s_j, \phi_j)$
    depending on each player's strategy $s$ and the phase difference $\Delta \phi$.
  }
  \label{fig:connectivity}
\end{figure}

We extend the EK game from \tripp{} in two ways:
placing the player network on a directed, weighted graph;
and introducing a payoff asymmetry.
Introducing a graph structure, with players represented by nodes
and interactions by edges, allows us to tailor the interactions.
For instance, \cref{fig:graph_well-mixed} shows the interaction graph
for the well-mixed population in \tripp{} represented as a complete graph.
Each directed edge on the interaction graph
represents a single game between the two connected nodes (players).
The game's payoffs only flow to the tail node,
but the usual, bidirectional game can be represented by
a pair of edges in both directions,
as shown in \cref{fig:player_interactions}.
We note that the \gls{celegans} connectome
has \num{38} self-loops, \num{669} bidirectional edge pairs,
and \num{2331} unpaired edges.
Finally, each payoff is scaled by the edge's weight;
since the \gls{celegans} weights are all integers,
the edge weight can be interpreted as the number of games played
between the two nodes.

We also generalize the payoff structure to incorporate
an asymmetry between communicators and non-communicators.
Each player (node) is represented by a 2-tuple of parameters,
its strategy $s_i$---either communicative $C$ or non-communicative $N$---
and its phase $\phi_i$---taking one of \num{20} evenly-spaced values
between $0$ and $2 \pi$.
The payoffs between a pair of players depend only on their strategies
and their relative phase difference, $\Delta \phi$.
Communicative players always pay a cost $c$.
If both players are non-communicative, the payoff is zero;
conversely, if both players are communicative,
the payoff is $B_0 f(\Delta \phi) - c$,
where the sinusoidal Kuramoto coupling
$f(\Delta \phi) = [1+\cos(\phi_j - \phi_i)]/2$
is scaled by the joint CC benefit, $B_0$.
When exactly one of the players is communicative,
we introduce an asymmetry $\alpha \in [0,1]$ to allow for differing payoffs
between the communicative and non-communicative players.
If the tail-player is communicative,
the payoff is $2 \alpha \beta_0 f(\Delta \phi) - c$,
where $\beta_0$ is the mixed benefit parameter.
If the tail player is non-communicative,
the payoff is $2 (1-\alpha) \beta_0 f(\Delta \phi)$.
This asymmetry $\alpha$ allows us to promote or suppress communicativeness,
and $\alpha = 0.5$ reproduces the symmetric case of \tripp{}.
These payoffs are summarized in the payoff matrix
of \cref{fig:payoff_matrix}.

\subsection{Parameter space}
\begin{figure}
  \centering
  \begin{subcaptiongroup}
    \stackinset{l}{2.7in}{t}{0pt}%
      {\phantomcaption\label{fig:phase-diagram-B_alpha}\captiontext*{}}{%
    \stackinset{l}{10pt}{t}{0pt}%
      {\phantomcaption\label{fig:phase-diagram-B_beta}\captiontext*{}}{%
    {\includestandalone{tikz/phase-diagram}}%
  }}
  \end{subcaptiongroup}
  \caption{
    \textbf{
      Payoff asymmetry enriches neural interactions
      well beyond the classic prisoner's dilemma game type.
      Shown are region plots illustrating the diverse range
      of game types that neural populations can engage in
      during evolutionary dynamics.
    }
    Slices of the three-parameter game type phase diagram
    in the
    \protect{\subref{fig:phase-diagram-B_beta}}
    $B$-$\beta$ plane
    and
    \protect{\subref{fig:phase-diagram-B_alpha}}
    $B$-$\alpha$ plane.
    For two players with phase difference $\Delta \phi$,
    $B = B_0 [1 + \cos(\Delta \phi)]/2$
    and
    $\beta = \beta_0 [1 + \cos(\Delta \phi)]/2$.
    The game type corresponding to each color is displayed in the legend.
    The white dots represent the $m=20$ potential phase differences.
    The cost $c$ is \num{0.1};
    for
    \protect{\subref{fig:phase-diagram-B_beta}},
    $\alpha = 0.5$,
    and for
    \protect{\subref{fig:phase-diagram-B_alpha}},
    $\beta_0 = \num{0.95} B_0$.
    \protect{\subref{fig:phase-diagram-B_beta}}
    is an extension of figure 1 from \tripp{}.
  }
  \label{fig:phase-diagram}
\end{figure}

One of the key aspects of the EK model
is that many different $2 \times 2$ game types
are available to the players during the population's evolution,
such as deadlock, dilemma, chicken, \etc{}
\citep[\cf{}][for definitions]{bruns2015names}.
We can visualize this by looking at the phase space of games
as a function of the three parameters $\beta$, $B$, and $\alpha$.
\Cref{fig:phase-diagram} shows two particular slices
of this three-dimensional phase space:
\cref{fig:phase-diagram-B_beta} shows a $B$-$\beta$ slice of phase space
and is an extension of figure 1 in \tripp{},
while \cref{fig:phase-diagram-B_alpha} shows a $B$-$\alpha$ slice.



\subsection{Complete graphs}
\label{sec:complete_graph}

\begin{figure}
  \centering
  \begin{subcaptiongroup}
    \stackinset{l}{3.4in}{t}{2.5in}%
      {\phantomcaption\label{fig:time-series_well-mixed_alpha-1}\captiontext*{}}{%
    \stackinset{l}{1in}{t}{2.5in}%
      {\phantomcaption\label{fig:time-series_well-mixed_alpha-075}\captiontext*{}}{%
    \stackinset{l}{3.4in}{t}{10pt}%
      {\phantomcaption\label{fig:time-series_well-mixed_alpha-0}\captiontext*{}}{%
    \stackinset{l}{1in}{t}{10pt}%
      {\phantomcaption\label{fig:multi-comm-frac}\captiontext*{}}{%
    {\includestandalone{tikz/well-mixed}}%
  }}}}
  \end{subcaptiongroup}
  \caption{
    \textbf{
      Impact of symmetry breaking on neural synchronization
      in well-mixed populations.
    }
    \protect{\subref{fig:multi-comm-frac}}
    Time-averaged fraction of players that are communicative as a function
    of the maximum joint benefit $B_0$
    for different values of the payoff asymmetry $\alpha$.
    The marks represent the simulation results
    and the lines represent the theory predictions.
    The $B_0$ step size is \num{0.04},
    and the simulations were run for \num{2E8} time steps.
    \protect{\subref{fig:time-series_well-mixed_alpha-0}}--\protect{\subref{fig:time-series_well-mixed_alpha-1}}
    Scatter plots where
    the left vertical axis of each subplot gives
    the instantaneous fraction of players that are communicative
    as a function of time.
    These points are color-coded
    grey if all players are communicative (``all-C'') or
    black if all players are non-communicative (``all-N'');
    otherwise, they are colored according
    to the plurality game type as indicated in the legend.
    The right vertical axis corresponds to the order parameter $\rho$
    (\pcref{eq:order_parameter}), in magenta, as a function of time.
    The asymmetry is
    \protect{\subref{fig:time-series_well-mixed_alpha-0}}
    $\alpha = 0$,
    \protect{\subref{fig:time-series_well-mixed_alpha-075}}
    $\alpha = 0.75$,
    and
    \protect{\subref{fig:time-series_well-mixed_alpha-1}}
    $\alpha = 1$,
    and the simulation were run for \num{8E5} time steps.
    For all four subplots,
    the network topologies are the
    $N=20$ well-mixed population,
    the maximum mixed benefit $\beta_0$ is $\num{0.95} B_0$ at each step,
    the selection strength is $\delta=0.2$,
    the asymmetry is $\alpha=0.75$,
    the mutation rate is $\mu=\num{1E-4}$,
    and
    the cost $c$ is \num{0.1}.
  }
  \label{fig:well-mixed}
\end{figure}

First, we will explore the influence of the newly introduced asymmetry
on the population's evolution by
re-analyzing the $N=20$ well-mixed population
originally analyzed in \tripp{}.
\Cref{fig:multi-comm-frac} compares the frequency of communicative strategies,
$f_{\text{comm}}$, to the maximum joint benefit $B_0$.
The marks represent the simulation results
and the lines represent the full analytic
result (\pcref{eq:full_analytic}) for a well-mixed population.
In general, $f_{\text{comm}}$ is low for small $B_0$,
rises to $f_{\text{comm}} = 0.5$ at some break-even $B_0$,
and plateaus to $f_{\text{comm}} \approx 1$ for large $B_0$.
We note that the $\alpha = 0.5$ case corresponds
to the setup in \tripp{} and produces qualitatively similar results,
with our break-even $B_0 \approx 0.21$
corresponding to their $B_0 = 2 (N-1) c/(N-2) = 0.21$ break-even condition.
We also see that increasing (decreasing)
the asymmetry $\alpha$ dilates (stretches) this sigmoid function
in the $B_0$ direction.
In particular, for the larger $\alpha$'s, the $f_{\text{comm}}$ curve
rises more steeply and at smaller values of $B_0$.
This $\alpha$-dependence is reasonable,
as increasing $\alpha$ corresponds to biasing the payoff
in a mixed $C$-$N$ interaction towards the communicative partner.
Therefore, it is unsurprising that the proportion of communicative players
increases with increasing $\alpha$.

While \cref{fig:multi-comm-frac} displays
the time-averaged system state,
it is also useful to investigate the time-dependent variations.
\Crefrange{fig:time-series_well-mixed_alpha-0}{fig:time-series_well-mixed_alpha-1}
depict the frequency
of communicative strategies $f_{\text{comm}}$
as scatter plots of time on the left vertical axis
for different values of the asymmetry, $\alpha$.
These time-series points are color-coded
grey if all players are communicative or
black if all players are non-communicative;
otherwise, the points are colored according
to the plurality game type, as indicated in the legend.
On the right vertical axes,
magenta line plots depict the order parameter
$\rho \in [0,1]$
given by \cref{eq:order_parameter}.
These plots shows the results
for a well-mixed population of $N=20$ players,
and the simulations were run
for \num{8E5} time steps with
selection strength $\delta = 0.2$,
asymmetry $\alpha = 0.75$,
mutation rate $\mu=\num{1E-4}$,
cost $c = \num{0.1}$,
maximum joint benefit $B_0 = 0.15$,
and maximum mixed benefit $\beta_0$ is $\num{0.95} B_0$.

\Cref{fig:time-series_well-mixed_alpha-0} show the time-variation
when the system heavily favors non-communicative players
with $\alpha = 0$.
It is unsurprising, then, that the system is virtually
always in a completely non-communicative regime, as indicated by the
black line at $f_{\text{comm}} = 0$,
with only a single, early excursion to an unsynchronized
deadlock-type game (light blue).
\Cref{fig:time-series_well-mixed_alpha-075} depicts an asymmetry $\alpha =0.75$
that moderately encourages communicativeness.
This plot shows slightly less metastability:
the system spends most of its time in a completely communicative state
(gray line at $f_{\text{comm}} = 1$)
but has occasional dips to complete non-communication
(black line at $f_{\text{comm}} = 0$)
and even unsynchronized states with hero-type games
(orange line with $0 < f_{\text{comm}} < 1$).
Nevertheless, it is apparent that the system does not spend appreciable time
in these unsynchronized states, as the orange lines are a small fraction
of the total time.
This observation is consistent with \cref{fig:multi-comm-frac},
which shows that these simulation parameters
($B_0 = 0.15$ and $\alpha = 0.75$)
should give rise to a communicative regime
approximately \SI{75}{\percent} of the time.
Finally, \cref{fig:time-series_well-mixed_alpha-1} shows the
$\alpha = 1$ case where communication is heavily incentivized.
Here, we see the opposite result from
\cref{fig:time-series_well-mixed_alpha-0}:
the system is virtually always in the completely communicative regime,
with only brief excursions to unsynchronized,
deadlock-type games (light blue).
We note that all three scenarios are nearly always synchronized;
the order parameter $\rho$ (magenta) is almost always $\rho = 1$,
with only occasional dips.
The temporary drops in communicative frequency $f_{\text{comm}}$
and order parameter $\rho$ are likely the result of mutations,
which are expected to occur, on average, every $1/\mu = \num{1E4}$ time steps.

\subsection{\glsfmtshort{celegans} graphs}
\label{sec:elegans_graph}

Here, we consider the weighted, directed hermaphroditic \gls{celegans} connectome.
For reference, \cref{fig:graph} shows the $N = 300$ nodes and the directed edges
(we exclude the two unconnected neurons CANL/CANR).
The network layout shows that there are two main clusters of neurons,
one large group on the bottom of \cref{fig:graph} and a smaller group
near the top.
The nodes are colored according to their strategy at a particular time step,
with blues representing communicative strategies
and reds representing non-communicative ones,
with different shades corresponding to different phases $\phi$.
We note the large, synchronized group
of red, non-communicative nodes in the bottom of the graph,
as well as the remaining unsynchronized nodes around it.
This mixture of synchronized and desynchronized nodes
is reflective of a chimera state and will be discussed more
in \cref{sec:discussion}.

\begin{figure}
  \centering
  \begin{subcaptiongroup}
    \stackinset{l}{0.35in}{t}{3.7in}%
      {\phantomcaption\label{fig:time-series_celegans-full}\captiontext*{}}{%
    \stackinset{l}{2.7in}{t}{4.5in}%
      {\phantomcaption\label{fig:comm-frac_celegans-undirected}\captiontext*{}}{%
    \stackinset{l}{2.7in}{t}{2.5in}%
      {\phantomcaption\label{fig:comm-frac_celegans-unweighted}\captiontext*{}}{%
    \stackinset{l}{2.7in}{t}{0.5in}%
      {\phantomcaption\label{fig:comm-frac_celegans-full}\captiontext*{}}{%
    \stackinset{l}{10pt}{t}{0.5in}%
      {\phantomcaption\label{fig:graph}\captiontext*{}}{%
    {\includestandalone{tikz/c-elegans}}%
  }}}}}
  \end{subcaptiongroup}
  \caption{
    \textbf{
      Rise of chimera states in \gls{celegans} neural network.
    }
    \protect{\subref{fig:graph}}
    The network topology for the \gls{celegans}
    $N=300$ weighted, directed connectome.
    The colors represent the $2m = 40$ strategies
    at a particular time step;
    blue colors are communicative,
    red colors are non-communicative,
    and shades represent different phases $\phi$.
    \protect{\subref{fig:comm-frac_celegans-full}}--\protect{\subref{fig:comm-frac_celegans-undirected}}
    Time-averaged fraction of players that are communicative as a function
    of the maximum joint benefit $B_0$.
    The network topologies are the
    \protect{\subref{fig:comm-frac_celegans-full}}
    weighted, directed \gls{celegans} connectome,
    \protect{\subref{fig:comm-frac_celegans-unweighted}}
    unweighted, directed \gls{celegans} connectome,
    and
    \protect{\subref{fig:comm-frac_celegans-undirected}}
    weighted, undirected \gls{celegans} connectome.
    The $B_0$ step size is \num{0.04} and
    the simulations were run for \num{2E8} time steps.
    \protect{\subref{fig:time-series_celegans-full}}
    Scatter plot with the same axes and coloring
    as \protect{\crefrange{fig:time-series_well-mixed_alpha-0}{fig:time-series_well-mixed_alpha-1}}.
    The simulation was run for \num{8E5} time steps.
    For all cases,
    the selection strength is $\delta=0.2$,
    the asymmetry is $\alpha=0.75$,
    the mutation rate is $\mu=\num{1E-4}$,
    the maximum mixed benefit $\beta_0$ is $\num{0.95} B_0$ at each step,
    and
    the cost $c$ is \num{0.1}.
  }
  \label{fig:c-elegans}
\end{figure}

Next, to quantify the observations of \cref{fig:graph},
\cref{fig:comm-frac_celegans-full} shows the fraction of players
using communicative strategies $f_{\text{comm}}$ averaged across
the entire simulation of \num{2E8} time steps as a function
of the maximum joint benefit $B_0$
for the \gls{celegans} connectome network topologies.
In addition, two variants of the full weighted, directed \gls{celegans}
connectome are shown:
a \subref{fig:comm-frac_celegans-unweighted} \emph{directed} but unweighted case
and
a \subref{fig:comm-frac_celegans-undirected} \emph{weighted} but undirected case.
The plots use a selection strength of $\delta=0.2$
and an asymmetry of $\alpha=0.75$.

We note that the behaviour
of the \subref{fig:comm-frac_celegans-full} \gls{celegans} case
is qualitatively distinct
from the $\alpha = 0.75$ well-mixed case in \cref{fig:multi-comm-frac}.
We see that the \gls{celegans} population is almost universally non-communicative
for $B_0 \le 0.08$,
followed by an immediate jump to a maximum communicative fraction
of \num{0.77} of $B_0 = 1.6$.
This communicative fraction then slowly declines
to a horizontal asymptote around \num{0.60}.
This deviation implies that the \gls{celegans} case
has qualitatively different behavior compared to the baseline, well-mixed case.
We can investigate the cause of the deviation
between the \cref{fig:multi-comm-frac} well-mixed
and \cref{fig:comm-frac_celegans-full} full \gls{celegans} case
by looking at variations to the \gls{celegans} network topology.
First, the \subref{fig:comm-frac_celegans-unweighted}
\emph{directed}, unweighted connectome
is qualitatively similar to the
\subref{fig:multi-comm-frac}
well-mixed case :
while the directed \gls{celegans} case exhibits a steeper rise
than well-mixed case,
the directed case still shows a monotonic increase
from low communicativeness for $B_0 < 0.1$
to full communicativeness for $B_0 \ge 0.2$.
This implies that the quality of directedness
plays a relatively small role in the qualitative shape
of the \subref{fig:comm-frac_celegans-full} full \gls{celegans} case.
On the other hand,
the \subref{fig:comm-frac_celegans-undirected}
\emph{weighted}, undirected connectome
looks quite similar to the \subref{fig:comm-frac_celegans-full}
full \gls{celegans} case .
In particular, this \subref{fig:comm-frac_celegans-undirected}
weighted case
shows the same plateau at $f_{\text{comm}} \approx 0.7$
seen in the \subref{fig:comm-frac_celegans-full} full \gls{celegans} case,
though the peak around $B_0 = 0.15$ is less pronounced.
This similarity implies that the connectome's edge weights
are the most relevant aspect influencing the deviation
of the \subref{fig:comm-frac_celegans-full} \gls{celegans}
results from the well-mixed results.

We can also investigate the time-evolution of the \gls{celegans} system
as we did for the well-mixed case in
\crefrange{fig:time-series_well-mixed_alpha-0}{fig:time-series_well-mixed_alpha-1}.
The setup is the same as in
\crefrange{fig:time-series_well-mixed_alpha-0}{fig:time-series_well-mixed_alpha-1}
except that the complete interaction graph is replaced with the \gls{celegans} connectome.
Compared to the well-mixed case,
the \subref{fig:time-series_celegans-full} \gls{celegans} case
depicts a far more heterogeneous population.
Here, the population never stabilizes
to a fully communicative or non-communicative state.
Instead, both its communicative frequency and order parameter
stay between \SIrange{40}{70}{\percent}.
With a communicative frequency $f_{\text{comm}}$ averaging approximately \SI{60}{\percent},
this is similar to the $f_{\text{comm}} \approx \SI{70}{\percent}$ seen
in \cref{fig:comm-frac_celegans-full} with $B_0 = 0.15$;
the deviation from \SI{70}{\percent} is reflective of the
stochasticity in this small, \num{8E5} time-step sample.
Compared to the \cref{fig:time-series_well-mixed_alpha-075} well-mixed case,
the \gls{celegans} case shows less stability to mutations.
Instead of the homogeneous communicative state with impulse disturbances,
the \gls{celegans} case depicts heterogeneous communicativeness
interrupted by discontinuous jumps.
Again, these are most likely the results of mutations introduced
to the population.

\subsubsection{Chimera-like index}
\begin{figure}
  \centering
  \pgfplotsset{width=0.4\textwidth}
  \begin{subcaptiongroup}
    \stackinset{l}{3in}{t}{2.5in}%
      {\phantomcaption\label{fig:graph_celegans_asymmetry1}\captiontext*{}}{%
    \stackinset{l}{0.71in}{t}{2.5in}%
      {\phantomcaption\label{fig:graph_celegans_asymmetry0}\captiontext*{}}{%
    \stackinset{l}{3in}{t}{15pt}%
      {\phantomcaption\label{fig:metastability_index}\captiontext*{}}{%
    \stackinset{l}{0.7in}{t}{15pt}%
      {\phantomcaption\label{fig:chimera_index}\captiontext*{}}{%
    {\includestandalone{tikz/chimera-states}}%
  }}}}
  \end{subcaptiongroup}
  \caption{
    \textbf{
      Characterizing chimera states.
    }
    \protect{\subref{fig:chimera_index}}
    The chimera-like index (\protect{\pcref{eq:chimera_index}})
    and
    \protect{\subref{fig:metastability_index}}
    metastability index (\protect{\pcref{eq:metastability_index}})
    as functions of the asymmetry $\alpha$
    for the
    weighted, directed \gls{celegans} connectome.
    The indices were calculated after splitting the graph
    into two communities according to the nodes' relative covariance
    for the $\alpha = \num{0.75}$ simulation
    and grouped according to whether the sum
    of a node's relative covariances are above
    \SI{81}{\percent} of the maximum covariance sum.
    The simulation were run for \num{8E6} time steps.
    \protect{\subref{fig:graph_celegans_asymmetry0}}--\protect{\subref{fig:graph_celegans_asymmetry1}}
    A snapshot of the \gls{celegans} connectome where
    blue colors are communicative,
    red colors are non-communicative,
    and shades represent different phases $\phi$.
    The asymmetry is
    \protect{\subref{fig:graph_celegans_asymmetry0}}
    $\alpha = 0$
    and
    \protect{\subref{fig:graph_celegans_asymmetry1}}
    $\alpha = 1$.
    For all cases,
    the selection strength is $\delta=0.2$,
    the mutation rate is $\mu=\num{1E-4}$,
    the cost $c$ is \num{0.1},
    $B_0$ is \num{0.15},
    and
    $\beta_0$ is $\num{0.95} B_0$.
  }
  \label{fig:chimera-index}
\end{figure}

Time-lapse animations of the system's time evolution
(see supplementary materials)
show a subset of the nodes exhibiting high synchronicity
with others remaining incoherent: this is characteristic of a chimera state.
In order to quantify this chimera-like effect, we will investigate
the chimera-like index $\chi$ (\pcref{eq:chimera_index})
and metastability index $\lambda$ (\pcref{eq:metastability_index}).
As discussed in \cref{sec:stats_setup},
the chimera-like index
measures the coherence difference between communities of players,
is calculated as the time-averaged community covariance,
and has a theoretical maximum value of \num{0.29}.
Conversely, the metastability index
represents how often the system transitions
between synchronicity and desynchronicity,
is calculated as the community-average of the temporal covariance,
and has a practical maximum of \num{0.08}.
\Cref{fig:chimera-index} shows these results
as functions of the asymmetry $\alpha$.
The left vertical axis depicts the chimera-like index $\chi$
in a solid line,
while the right vertical axis shows the metastability index
in a dashed line.
These simulations were run with
for \num{8E6} time steps with
selection strength $\delta = 0.2$,
mutation rate $\mu=\num{1E-4}$,
cost $c = \num{0.1}$,
maximum joint benefit $B_0 = 0.15$,
and maximum mixed benefit $\beta_0 = \num{0.95} B_0$.

The metastability displayed in \cref{fig:chimera-index}
is quite low at $\le \num{0.01}$ compared
to a theoretical maximum of \num{0.08}.
This implies that the system is quite stable
and spends most of its time at a fairly constant
synchronicity $\rho_m(t)$.
Furthermore, while the metastability is higher (more unstable)
at $\approx \num{0.01}$ for the $\alpha = 0$ case,
the metastability drops precipitously for larger $\alpha$,
always staying below \num{0.004} implying that
higher asymmetry $\alpha$ makes the system more stable.
On the other hand, the chimera-like index $\chi$
attains fairly large values of \numrange{0.06}{0.13}
indicating a high chimeric quality.
Considering the theoretical maximum is \num{0.28},
the system is quite chimeric at nearly half the maximum.
Given that the chimera-like index is averaged over time,
this deviation from a complete chimera state
arises from both imperfect separation of the coherent/incoherent populations
as well as time fluctuations in the chimeric quality.
While we observe that $\chi$ has a maximum at $\alpha = \num{0.75}$,
we caution that this is likely the result of calculating the communities
based off the $\alpha = \num{0.75}$ covariances.
Nevertheless, the fact that $\chi$ remains elevated
for the other asymmetries $\alpha$
implies that this particular community decomposition
corresponds to an intrinsic separation rather than an artifact
of the particular $\alpha = \num{0.75}$ simulation.

\subsubsection{Game types}
\begin{figure}
  \centering
  \begin{subcaptiongroup}
    \stackinset{l}{2.3in}{t}{0pt}%
      {\phantomcaption\label{fig:game-type_celegans-full}\captiontext*{}}{%
    \stackinset{l}{0in}{t}{0pt}%
      {\phantomcaption\label{fig:game-type_well-mixed}\captiontext*{}}{%
    {\includestandalone{tikz/game-types}}%
  }}
  \end{subcaptiongroup}
  \caption{
    \textbf{
      The underlying game types explain the persistence
      and origins of chimera states.
    }
    The plurality game type amongst all player interactions
    (weighted by the interaction graph edge weight),
    expressed as a fraction of all games played,
    for different values of the asymmetry $\alpha$.
    The game types are color-coded according to the legend;
    additionally, ``all-C'' and ``all-N'' represent when the population
    is entirely synchronized to communicativeness or non-communicativeness, respectively.
    The network topologies are the
    \protect{\subref{fig:game-type_well-mixed}}
    $N=20$ well-mixed population
    and
    \protect{\subref{fig:game-type_celegans-full}}
    weighted, directed \gls{celegans} connectome.
    The selection strength is $\delta=0.2$,
    the mutation rate is $\mu=\num{1E-4}$,
    the cost $c$ is \num{0.1},
    $B_0$ is \num{0.15},
    $\beta_0$ is $\num{0.95} B_0$,
    and the simulation were run for \num{8E6} time steps.
  }
  \label{fig:game-type}
\end{figure}

Next, we would like to understand the types of games being played
during the population's evolution.
In \cref{fig:game-type}, we investigate the plurality game type
amongst all player interactions at each time step~(\cf \cref{sec:stats_setup}),
where the number of interactions between two players
is the corresponding edge weight of the interaction graph.
\Cref{fig:game-type} shows what fraction of time
a given game type is the plurality for different values
of the asymmetry $\alpha$ for
\subref{fig:game-type_well-mixed}
$N=20$ well-mixed,
and
\subref{fig:game-type_celegans-full}
\gls{celegans} connectome
network topologies.
The \subref{fig:game-type_well-mixed} well-mixed case
shows that the time evolution for each asymmetry $\alpha$
is dominated by synchronized play
for over \SI{99.3}{\percent} of the time steps.
In contrast, the
\subref{fig:game-type_celegans-full} \gls{celegans} connectome system
is never synchronized.
Instead, non-synchronized (\ie states other than ``all-C'' or ``all-N'')
game types dominate,
with the dominant game type changing as the asymmetry is varied:
dilemma for $\alpha = 0, 0.25$, chicken for $\alpha = 0.5$,
hero for $\alpha = 0.75$, and deadlock for $\alpha = 1$.
Additionally, the \gls{celegans} system displays
some game-type heterogeneity, with
deadlock-type game being the plurality
\SIrange{1}{14}{\percent} of the time
for asymmetry $\alpha \le 0.5$.
This shows that, as the asymmetry is varied,
different game types dominate the population's evolution,
and network topology influences how dominant these game types are.

\section{Discussion}
\label{sec:discussion}

%\subsection{Game-type dynamics}
The makeup of game types displayed in \cref{fig:game-type}
deserves additional discussion.
For the \cref{fig:game-type_well-mixed} well-mixed case,
the system is virtually always synchronized,
as corroborated by the order parameter $\rho \approx 1$ in
\crefrange{fig:time-series_well-mixed_alpha-0}{fig:time-series_well-mixed_alpha-1}.
Additionally, the specific frequency of communicative synchronization (``all-C'')
and non-communicative synchronization (``all-N'')
approximately matches the predicted $f_{\text{comm}}$
frequency displayed in \cref{fig:multi-comm-frac} for $B_0=0.15$.
For the $< \SI{1}{\percent}$ of the time when the well-mixed system is
\emph{not} synchronized, the other game types are
dilemma (\SI{0.2}{\percent} for $\alpha = 0, 0.25$)
chicken (\SI{0.3}{\percent} for $\alpha = 0.5$),
hero (\SI{0.3}{\percent} for $\alpha = 0.75$),
and
deadlock (\SI{0.3}{\percent} for $\alpha = 1$).
We note that these non-synchronized game types
are the same as the most-frequency game types,
for each $\alpha$, as those played in
the \cref{fig:game-type_celegans-full} \gls{celegans} setup.

We can compare the $\alpha = 0.5$ case
to the results in figure 4 of \tripp{}.
Those results also show the system spending the vast majority
of its time in a synchronized state.
When the \tripp{} system was not synchronized,
it displayed chicken (denoted ``snowdrift'')
and staghunt (denoted ``coordination'',
not to be confused with the coordination-type game
in our \cref{fig:phase-diagram}) games.
The presence of staghunt-type games is surprising,
as \cref{fig:phase-diagram-B_beta}
shows that the staghunt region should be inaccessible
for the given parameter values (white dots).
However, the chicken-type game is consistent with our
\SI{0.3}{\percent} chicken games for $\alpha = 0.5$.

Next, we consider the distribution of non-synchronized games
in \cref{fig:game-type} as functions of $\alpha$.
Recall that the non-synchronized game types
in the \cref{fig:game-type_well-mixed}
well-mixed setup
and the most-frequent plurality game types in the \cref{fig:game-type_celegans-full}
\gls{celegans} setup
are identical,
so this discussion applies to both setups.
We note that the most-frequent plurality game types
across each $\alpha$ all correspond
to the nearly synchronized $\Delta \phi \approx 0$ game type
(\ie the game type of the rightmost white dots \cref{fig:game-type}).
This indicates that the majority game type almost always
corresponds to games between nearly-synchronous individuals.
This is somewhat surprising for the $\alpha = 1$ case:
there, four of the $m = 20$ phase differences
($\Delta \phi = 4 \pi/20, 5 \pi/20, 15\pi/20, 16\pi/20$)
fall in the pink, battle-type region.
The warm colors (red, orange, and pink)
represent the game types (chicken, hero, and battle, respectively)
that have exponentially slow fixation times~\citep{antal2006fixation},
so one might expect that these games would be over-represented
due to their long fixation times.
Nevertheless, the strong $\alpha = 1$ communicativeness bias
enhances synchronization and ensures that
chimeric qualities are suppressed
(\cf the low $\chi$ for $\alpha = 1$ in \cref{fig:chimera-index}).

We can further explain this distribution of non-synchronized games
by considering the well-mixed graph's average degree.
Since \cref{fig:game-type} depicts the plurality game at each time step,
we need to consider the conditions under which a game type can be the plurality.
For the complete, well-mixed case, the game type corresponding
to the synchronized $\Delta \phi = 0$ state
is easily obtained---for instance, if the entire graph is synchronized,
or largely synchronized.
We can easily derive two conditions when the $\Delta \phi = 0$ game type
is \emph{not} the plurality game type:
when there are $p$ equal-sized groups
with phases that differ by $2 \pi/p$ (\cf \cref{sec:multiple_equal_groups});
or when the game is almost evenly split between two groups
with fraction within $\pm 1/(2 \sqrt{N})$ of $0.5$
(\cf \cref{sec:two_unequal_groups}).
Given the low mutation rate, it is unlikely to have
multiple populations with roughly equal sizes,
so the evenly-split case is more likely.
For the $N = 20$ well-mixed case, this corresponds
to each group containing \numrange{8}{12} players,
but the magenta order parameter in \cref{fig:time-series_well-mixed_alpha-075}
shows that the well-mixed system spends must of its time
with very few invaders---fewer
than the \numrange{8}{12} required to flip the plurality game type.

This same analysis of graph degree explains why
\cref{fig:game-type_celegans-full} \gls{celegans} game types
for the $\alpha = 0,0.25,0.5$ games
exhibit game-type heterogeneity,
with significant levels of less-frequent plurality games
(\SI{13}{\percent}, \SI{3}{\percent}, and \SI{1}{\percent} deadlock games,
respectively).
While the number of players $N = 300$ is much larger
than the complete, well-mixed case,
the average degree $\overline{d}$
is smaller (\num{12.36} \vs \num{19}).
Therefore, if we consider the non-complete, \gls{celegans} connectome graph
with two major populations
and assume they are randomly distributed (for ease of computation),
then the same calculations as above
(\cf \cref{sec:two_unequal_groups_incomplete}) show that
the synchronized game type will \emph{not} be the plurality game
if the fraction of invaders is within
$\pm 1/(2 \sqrt{\overline{d}} + 1)$
of $0.5$, or \numrange{0.37}{0.63}.
We notice that the order parameter in \cref{fig:time-series_celegans-full}
actually spends a good deal of time in this range,
allowing for game types that don't correspond to the synchronized state.
While the $\alpha = 0.75$ case depicted in \cref{fig:time-series_celegans-full}
still mostly displays hero-type games
(likely because the hero region contains most of the white $\Delta \phi$ dots
for $\alpha = 0.75$ in \cref{fig:phase-diagram-B_alpha}),
the $\alpha = 0, 0.25, 0.5$ cases all have wider $\Delta \phi$ regimes where
deadlock-type games can appear, as illustrated in \cref{fig:game-type_celegans-full}.
Furthermore, the decreased in-degree of the \gls{celegans} could allow
for additional enhancement of the non-synchronized game types.
For instance, if regions of the connectome have a structure with more edges
\emph{between} player populations than \emph{amongst} them
(similar to a bipartite graph),
this would preferentially enhance non-synchronized game type edges.
More research needs to be done to investigate the role of this particular effect.

%\subsection{Chimera states}
Next, we will discuss the observed chimera states.
The chimera-like quality is inherently time-dependent
(\cf{} \cref{eq:chimera_index})
and is therefore most evident in time-varying animations (see supplemental materials).
Nevertheless, carefully chosen snapshots can still convey some of the chimera-like aspects.
\Cref{fig:graph_celegans_asymmetry0} shows a snapshot of the communication strategies
for asymmetry $\alpha = 0$ using the same color scheme as \cref{fig:graph}:
blues represent communicative strategies
and reds represent non-communicative ones,
with different shades corresponding to different phases $\phi$.
We notice a large region of light red representing
a synchronized group of non-communicators,
as well as a mix of desynchronized neighbors.
This coexistence of strong synchronization and disorder
is consistent with the relatively high
chimera-like index $\chi$ observed in \cref{fig:chimera-index} for $\alpha = 0$.
Likewise, the $\alpha = 0.75$ case depicted in \cref{fig:graph}
has an even higher $\chi = \num{0.13}$.
It displays a very large
synchronized group of non-communicative players with identical phases in red,
as well as a spread of desynchronized players with varying phases/colors.
In contrast,
\cref{fig:graph_celegans_asymmetry1} shows a snapshot for $\alpha = 1$.
From \cref{fig:chimera-index}, we expect a lower chimera-like index
of $\chi = \num{0.06}$, which corresponds to the smaller group
of synchronized communicators in dark blue
(not to be confused with the separate group of communicators
in \emph{light blue} on the center-left of the cluster).
Therefore, these snapshots give a glimpse of how the chimera-like quality
manifests on the \gls{celegans} graph.

Previous studies \citep[\eg][]{bansal2019cognitive,santos2017chimera}
have noted the significance of chimera states in cognitive systems.
Therefore, the strength of the chimera state
(with a notably high chimera-like index of \num{0.13}) is particularly striking.
Additionally, this is similar to the maximum value of \num{0.12}
observed in previous studies on chimera states
for modular \gls{celegans} networks~\citep{hizanidis2016chimera}.
The agreement between these very different models
of the \gls{celegans} connectome
support the claim that the connectivity structure---rather than
a particular neuron model or any biochemical qualities---is responsible
for the chimera states.

%\section{Conclusion}
In this paper, we extended the evolutionary Kuramoto model
introduced by \tripp{}
to include structured interaction and reproduction graphs
as well as an asymmetry between
the communicator and non-communicator payoffs.
We first applied this model to a well-mixed population
and found that increasing (decreasing) the asymmetry
inhibits (promotes) communicativity.
Next, we applied the model to
a family of graphs deriving from the \gls{celegans} connectome,
including: the full directed, weighted graph;
an unweighted, directed graph;
and an undirected, weighted graph.
This revealed that the graph's directedness
had a minor effect of steepening the transition
from non-communicative to communicative
as the maximum joint benefit was increased.
However, the weightedness of the graph had a much more drastic effect
on the population by strongly suppressing the maximum communicativity.
For the weighted connectome graphs (both directed and undirected),
the communicative frequency asymptote decreased from unity
to approximately \num{0.7} for large $B_0$.
Finally, we found qualitative differences between the composition
of the well-mixed populations and the \gls{celegans} populations.
While the well-mixed players had largely homogeneous populations
that occasionally switched between fully communicative
and fully non communicative,
the \gls{celegans} population remained far more heterogenous
with a stable, chimeric mix of both communicators and non-communicators.
This chimera state in a model of the \gls{celegans} brain connectome
was suggestive of prior studies finding chimera states in human brains.

A future goal of this work is to apply the EK model
to entire families of random graphs.
the limited number of graphs analyzed in this study
made it challenging to identify exact relationships
between communicative fraction and graph properties
such as edge degree, weight, and directionality;
applying the model to parameterized families could allow
for fine-tuning these parameters to extract relations.
Furthermore, the parameter space analysis provided clues
regarding the types of games the neurons played,
but additional study is needed to determine a causal relation.
Studying attractors
of the parameter space of game types
could shed light on the long-time game distributions,
with important implications for the neuron fitness environment.
Finally, since this study only considered a single connectome,
it is unclear if the findings regarding chimeric activity's
primary dependence on graph structure are generalizable.
We plan to investigate this connection further
by extending the model to other model organisms,
such as that of
\glsxtrlong{dmelanogaster}~\citep{schlegel2024whole}.

A key takeaway from this study was the importance
of edge weight on communicativity;
this has important parallels to neural computing,
where edge weights are a primary driver of functionality.
Additionally, the observation of chimeric states
arising from such simple neuron models
implies that brain structure is likely
more important that specific neuron properties
in producing these critical brain states.
Overall, evolutionary graph theory allowed us
to connect low-level payoff details for individual neurons
to high-level phenomena such as chimera-states,
and this model could serve as a valuable tool
for clarifying the influence of network structure on neural dynamics.

\section{Methods}
\label{sec:methods}

\subsection{Game setup}
\label{sec:game_setup}
We model the system of evolving, coupled oscillators
by discretizing the $2\pi$ phase angle into $m$ discrete phases $2 \pi j/m$
for $j \in 0, \ldots, m-1$.
The game's strategy space is the outer product of the $m$ phases
($m = 20$ for our simulations) and $2$ communicative choices,
communicative $C$ and non-communicative $N$.
For a given pair of phases, $\phi_i$ and $\phi_j$, the game is specified
by the payoff matrix
\begin{equation}
\begin{bNiceMatrix}[first-col,first-row,hvlines]
  & C, \phi_j & N, \phi_j \\
  C, \phi_i & B_0 f(\Delta \phi) - c & 2 \alpha \beta_0 f(\Delta \phi) - c \\
  N, \phi_i & 2 (1 - \alpha) \beta_0 f(\Delta \phi) & 0
\end{bNiceMatrix}
\label{eq:payoff-matrix}
\end{equation}
with $f(\Delta \phi) = [1+\cos(\phi_j - \phi_i)]/2$
(\pcref{fig:payoff_matrix}).
Here, $B_0$, $\beta_0$, $\alpha$, and $c$ are fixed parameters
defining the game.
The cost $c$ represents the penalty paid by communicative players,
and $B_0$ and $\beta_0$ are the maximum benefits paid with
joint CC communicators and mixed CN players, respectively.
The phase-dependent function $f(\Delta \phi)$ encodes
the influence of phase mismatches.
Finally, the benefit asymmetry $\alpha \in [0,1]$ breaks the symmetry
between the payoff for the communicator and the non-communicator
when exactly one player is communicative.

\subsection{Population setup}
\label{sec:pop_setup}

In this study, we consider a collection of different population graphs
to highlight the role of structure on communicativeness.
Given $N$ players, we associate a pair
of weighted, directed graphs to the population.
First, we use an interaction graph with $N$ nodes representing players
and weighted, directed edges representing games between players.
Second, we implement a reproduction graph with the $N$ nodes
still representing players
but the edges now corresponding to the ability of nodes to replace one another.
For simplicity, our reproduction graphs are identical to the interaction graph
with a single self-loop added to each node.
These self-loops are necessary in the reproduction graph
to ensure that each node has positive indegree
as required by the Moran process described in the next section.

\subsection{Birth-death Moran process}
\label{sec:evo_setup}
% TODO: Bd is hugely different from death-Birth dynamics (dB);
% this dB is more favorable for cooperation
% Try this instead?
The population is updated according to a birth-death Moran process
with exponential fitness~\citep[\eg \cf][]{lieberman2005evolutionary}.
On each turn, the following steps are performed.
First, a game is played on each edge $j$ in the interaction graph,
each edge's payoff $\pi_j$ is scaled by the edge weight $w_j$,
and the relevant payoff $w_j \pi_j$ is awarded to the tail node only.
\Cref{fig:player_interactions} shows an illustration of this process
for a single pair of interacting players connected by a pair of
weighted, directed edges.
The total fitness for node $i$ is the exponential of the product
between the selection strength $\delta$
and the sum of payoffs to node $i$,
or $f_i = \exp(\delta \sum_j w_j \pi_j)$ with the sum
over all edges inwardly incident to node $i$.
Then, a single focal node is chosen for reproduction
with probability proportional to the node's fitness $f_i$.
Finally, a node is chosen for replacement amongst the birth node's outneighbors
with probability proportional to the reproduction graph's edge weight.
With mutation probability $\mu$,
the death node is replaced by a player with a uniformly random strategy;
otherwise, it is replaced by a player with the same strategy as the birth node.
This birth-death process is repeated for each turn.

\subsection{Quantities of interest}
\label{sec:stats_setup}
A variety of statistics will be used to quantify the evolution
of the games' populations.
First, we determine the frequency of communicative strategies
$f_{\text{comm}}$ as the fraction of players employing
a communicative strategy $C$ at a given time step.
We are also interested in defining the time-averaged communicative frequency
by averaging $f_{\text{comm}}$ over the entire simulation;
for simulation times long compared to the mutation turnover time
$T_{\text{turn}} \gg N/\mu$,
the initial, random distribution of strategies will be negligible
and the time-average will correspond to the long-time limit.

We are interested in comparing the $f_{\text{comm}}$ simulation results
to an analytic model.
In the appendix, we analytically extend the work of \tripp{}
to account for benefit asymmetry $\alpha$.
The resulting communicative fraction \cref{eq:full_analytic}
is derived for the well-mixed case.
To apply \cref{eq:full_analytic} to non-complete graphs such as
the \gls{celegans} connectome,
we replace the degree $N-1$ with the average degree of the graph
$N-1 \to \overline{d}$.
Note that $\overline{d} = \num{19}$
for the well-mixed case with $N=20$ players
and $\overline{d} = \num{12.36}$ for the \gls{celegans} case.

We will analyze the types of games being played by
identifying the plurality game type across all player interactions,
weighted by the interaction graph edge weights,
at each time step.
Then, we calculate the frequency of this ``plurality game'' across
all time steps of a given simulation to determine the distribution
of games commonly played.
Note that a player pair's game type is determined solely
by their relative phase $\Delta \phi$ and is independent
of the players' communicativeness.
Furthermore, this metric is only sensitive to the
plurality game and therefore provides no information
on the presence/absence of minority game types.

Given that this evolutionary game model was inspired
by the Kuramoto system of coupled oscillators,
we will also use the standard Kuramoto order parameter:
\begin{equation}
  \rho = \frac{1}{N} \abs{\sum_{j=1}^N e^{\im \phi_j}}
  \label{eq:order_parameter}
\end{equation}
This parameter ranges from zero to one, inclusive,
and represents how coherent the population is,
with $\rho = 1$ for completely coherent and $\rho = 0$ completely incoherent.

Finally, we need a way to quantify
the chimera-like quality of the population.
To compare with a previous analysis~\citep{hizanidis2016chimera} of
chimera-like states \gls{celegans} models,
we will define a pair of indices related to
chimera-like quality and metastability~\citep{shanahan2010metastable}.
First, we need to organize the game's nodes
into $M$ disjoint communities.
We split the nodes into $M=2$ communities $C_m$
by taking a subset
(the first \num{8E4} steps for computations ease)
of the simulation results
for the \gls{celegans} simulation with asymmetry $\alpha = \num{0.75}$.
We then calculate the covariance matrix $K_{i,j}$ of the strategy indices
(ensuring that communicative ($C$, $\phi_i$)
and non-communicative ($N$, $\phi_i$) are treated distinctly)
and calculate the row-wise covariance-sums  $\sum_i K_{i,j}$.
After finding the node with the maximum covariance-sum
$\cov_{\text{max}} \coloneqq \max_j{\sum_i K_{i,j}}$,
we form a community by collecting all nodes $j$ with covariance-sum
at least \num{1500}, $\sum_i K_{i,j} \ge 1500$.
Note that \num{1500} is approximately \SI{81}{\percent}
of the maximum covariance-sum,
so this ensure we only group nodes with high covariance-sum.
Finally, we place all the remaining nodes in a second community.

With these disjoint communities $C_m$, we then calculate
the time-dependent, community-wise order parameter $\rho_m(t)$
as
\begin{equation}
  \rho_m(t) \coloneqq \frac{1}{N_m} \abs{\sum_{j \in C_m} e^{\im \phi_j}}
\end{equation}
across members $j$ of community $C_m$ with size $N_m$.
Then, we define a chimera-like index $\chi$
\begin{equation}
  \chi = \aab{
    \sigma_{\text{chi}}}_T
    \text{\qquad with \qquad}
    \sigma_{\text{chi}} \coloneqq \frac{1}{M-1} \sum_{m=1}^M
    \pab{\rho_m(t) - \aab{\rho_m(t)}_M}^2
  \label{eq:chimera_index}
\end{equation}
and a metastability index $\lambda$
\begin{equation}
  \lambda = \aab{
    \sigma_{\text{met}}}_M
    \text{\qquad with \qquad}
    \sigma_{\text{met}} \coloneqq \frac{1}{T-1} \sum_{t=1}^T
    \pab{\rho_m(t) - \aab{\rho_m(t)}_T}^2
  \label{eq:metastability_index}
\end{equation}
across the $M$ communities and $T$ time steps.
The chimera-like index measures the difference in coherence between communities:
complete homogeneity between communities
(\eg{} all fully synchronized \emph{or} fully desynchronized)
corresponds to $\chi = 0$,
while having two equally-sized communities
with one fully synchronized and one fully desynchronized
for all times yields a maximum $\chi = 2/7 \approx
\num{0.29}$~\citep{shanahan2010metastable}.
Likewise, the metastability index $\lambda$ measures how metastable
the system is (\ie{} transiting between synchronicity and desynchronicity).
A system that is fully synchronized or desynchronized gives $\lambda = 0$,
while spending equal times in each states yields $\lambda = 1/12
\approx \num{0.08}$~\citep{shanahan2010metastable}.

%\subsubsection{Acknowledgements}

\backmatter

\bibliography{references.bib}

\appendix
\section{Communicative fraction with symmetry breaking}
Here, we extend the derivation of the communicative fraction
in \tripp{} to include an asymmetry $\alpha$.
First, we use our $\alpha$-dependent payoff matrix, \cref{eq:payoff-matrix},
to calculate the $\alpha$-dependent average payoff functions.
Since the mutation rate is low, we can consider a population with only two strategies.
As in \tripp{}, we only consider the case where those strategies are different,
$E = (C, \phi_i)$ and $F = (N, \phi_j)$.
\begin{align}
  \pi_E(k) &= \ab(\frac{k-1}{n-1}) \pab{B_0 - c}
                + \ab(\frac{n-k}{n-1}) \ab(\beta(\Delta\phi) 2\alpha - c) \\
           &= \frac{1}{n-1}
               \ab(k \ab(B_0 - 2 \alpha \beta(\Delta\phi))
                 + 2 \alpha n \beta(\Delta\phi) - B_0 - (n-1) c)
\end{align}
and
\begin{align}
  \pi_F(k) &= \ab(\frac{k}{n-1}) \ab(\beta(\Delta\phi) 2 (1-\alpha))
                + \ab(\frac{n-k-1}{n-1}) \ab(0) \\
           &= \ab(\frac{k}{n-1}) 2 (1-\alpha) \beta(\Delta\phi)
\end{align}
where $k \in [0,N]$ is the number of $E$-players
$\Delta \phi$ is the phase difference between the $E$ and $F$
strategies,
and we defined $\beta(\Delta \phi) = \beta_0 f(\Delta \phi)$.
Note that the first term of $\pi_E$ corresponds
to the $E$-$E$ payoff, which has $\Delta \phi = 0$,
hence $B_0 f(\Delta \phi) = B_0$.

The derivation in section 4 of the appendix
from \tripp{}
carries through directly until equation 16
where a key parameter $\gamma_k$ is defined
as the ratio of the average $F$ payoff to average $E$ payoff.
Here, we make the small correction of replacing
the average fitness $f = \exp(\delta \pi)$ (used in the appendix)
with the total fitness $f = \exp(\delta (n-1) \pi)$
for a node with $(n-1)$ neighbors
(used in the main text and plots).
Thus, equation 16 becomes
\begin{align*}
  \gamma_k &= \frac{f_F(k)}{f_E(k)} \\
           &= \exp[\delta (n-1) (\pi_F(k) - \pi_E(k))]
           \\
           &= \exp\ab[\delta \ab(
    k 2 (1-\alpha) \beta(\Delta\phi)
    - k \ab(B_0 - 2 \alpha \beta(\Delta\phi))
                 - 2 \alpha n \beta(\Delta\phi) + B_0 + (n-1) c)] \\
      &= \exp\ab[\delta \ab(
    \ab(2 \beta(\Delta\phi) - B_0) k
                 + B_0 - 2 \alpha n \beta(\Delta\phi) + (n-1) c)]
\end{align*}

Continuing to follow the derivation of \tripp{},
we can calculate the ratio of fixation probabilities $\rho_F/\rho_E$ as
\begin{equation}
  \begin{aligned}
    \frac{\rho_F}{\rho_E} &= \prod_{k=1}^{n-1} \gamma_k \\
           &= \exp \ab\{
                \delta
                \sum_{k=1}^{n-1}
                \ab[
    \ab(2 \beta(\Delta\phi) - B_0) k
                 + B_0 - 2 \alpha n \beta(\Delta\phi) + (n-1) c
                 ]
                 \} \\
           &= \exp \ab\{
                \delta
                \ab[
                \ab(2 \beta(\Delta\phi) - B_0) \frac{n(n-1)}{2}
                 + \ab(B_0 - 2 \alpha n \beta(\Delta\phi) + (n-1) c) (n-1)
                 ]
                 \} \\
           &= \exp \ab\{
                \delta (n-1)
                \ab[
                (n-1) c + n \beta(\Delta\phi) (1 - 2 \alpha)
                - \frac{n-2}{2} B_0
                 ]
                 \}
  \end{aligned}
  \label{eq:fixation_prob_ratio}
\end{equation}
Notice, now the ratio of $\rho_F/\rho_E$ depends of $\Delta \phi$,
unlike in the $\alpha=1/2$ case considered by \tripp{}.

Similarly, since $\rho_E + \rho_F = 1$, we have
\begin{equation}
  \begin{aligned}
    \rho_E &= \frac{1}{1+\rho_F/\rho_E}
           = \frac{1}{1+\sum_{j=1}^{n-1} \prod_{k=1}^j \gamma_k} \\
           &= \ab(1+\sum_{j=1}^{n-1} \prod_{k=1}^j \exp\ab[\delta \ab(
           \ab(2 \beta(\Delta\phi_{qr}) - B_0) k
           + B_0 - 2 \alpha n \beta(\Delta\phi_{qr}) + (n-1) c)])^{-1}
                 \\
           &= \ab(1+\sum_{j=1}^{n-1} \exp\ab[\delta \sum_{k=1}^j \ab(
           \ab(2 \beta(\Delta\phi_{qr}) - B_0) k
           + B_0 - 2 \alpha n \beta(\Delta\phi_{qr}) + (n-1) c)])^{-1}
                 \\
           &= \ab(1+\sum_{j=1}^{n-1} \exp\ab[\delta \ab(
           \ab(2 \beta(\Delta\phi_{qr}) - B_0) j(j+1)/2
           + j \ab(B_0 - 2 \alpha n \beta(\Delta\phi_{qr}) + (n-1) c))])^{-1}
                 \\
           &= \ab(1+\sum_{j=1}^{n-1} \exp\ab[\delta \ab(
           \ab(\beta(\Delta\phi_{qr}) - B_0/2) j^2
           + j \ab(B_0/2 + \beta(\Delta\phi_{qr}) (1 - 2 \alpha n)  + (n-1) c))])^{-1}
  \end{aligned}
  \label{eq:comm_fixation_prob}
\end{equation}

Following section 5 in the appendix of \tripp{},
we apply these results from the two-population case
to the ($m=20$ phases) multi-population case with low-mutation rate
by identifying the fixation probability $\rho_{CN,\Delta \phi_{qr}}$
of an $(N,\phi_r)$ strategy invading an $(C,\phi_q)$ strategy
with $\rho_{CN,\Delta \phi_{qr}} = \rho_F$,
and likewise
$\rho_{NC,\Delta \phi_{rq}} = \rho_E$.
Equation 47 of \tripp{}
gives the ratio of the stationary state eigenvalues $s_1/s_2$ as
\begin{align}
  \frac{s_2}{s_1} &= \frac
    {\sum_{r=1}^{m} \rho_{CN,\Delta \phi_{rq}}}
    {\sum_{r=1}^{m} \rho_{NC,\Delta \phi_{qr}} }
  \\
  &=
  \frac
  {\sum_{r=1}^{m} \rho_{NC,\Delta \phi_{qr}}
    \exp \Bab{
      \delta (n-1)
      \bab{
        (n-1) c + n \beta(\Delta \phi_{qr}) (1 - 2 \alpha)
        - \frac{n-2}{2} B_0
      }
    }
  }
  {\sum_{r=1}^{m} \rho_{NC,\Delta \phi_{qr}}}
  \\
  &=
  \bab{
    \frac
    {\sum_{r=1}^{m} \frac
      {\exp \Bab{ \delta (n-1)
        \bab{
          (n-1) c + n \beta(\Delta \phi_{qr}) (1 - 2 \alpha) - \frac{n-2}{2} B_0
        }}
      }
      {1 + \sum_{j=1}^{n-1} \exp\Bab{
        \delta \bab{
         \pab{\beta(\Delta \phi_{qr}) - B_0/2} j^2
         + j \pab{B_0/2 + \beta(\Delta \phi_{qr}) (1 - 2 \alpha n) + (n-1) c}
        }}
      }
    }
    {\sum_{r=1}^{m} \frac{1}
      {1 + \sum_{j=1}^{n-1} \exp\Bab{
        \delta \bab{
         \pab{\beta(\Delta \phi_{qr}) - B_0/2} j^2
         + j \pab{B_0/2 + \beta(\Delta \phi_{qr}) (1 - 2 \alpha n) + (n-1) c}
        }}
      }
    }
  }
  \label{eq:full_analytic_frac}
\end{align}
The first equality used \cref{eq:fixation_prob_ratio}
and the second equality used \cref{eq:comm_fixation_prob}.
Unlike in the $\alpha = 1/2$ symmetric case, we cannot factor the
exponential component out of the sum and cancel the $\rho_{CN,\Delta
\phi_{qr}}$ terms.
However, we can asymptotically expand $s_2/s_1$ for small $B_0/c \ll 1$;
additionally, since all of our simulations use $\beta \propto B$,
we also assume $\beta/c \ll 1$.
Finally, to simplify the calculation,
we define the asymptotic expansion of
$\rho_{NC,\Delta \phi_{qr}} \coloneqq
\rho_{NC,\Delta \phi_{qr}}^{(0)}
+
\rho_{NC,\Delta \phi_{qr}}^{(1)}
+
\order{B_0^2}
$
where
$
\rho_{NC,\Delta \phi_{qr}}^{(i)}
$
depends only on terms of total order $i$ in $B_0$ and $\beta$.

Then, to first order in $B_0$ and $\beta$,
we have
\begin{align*}
  \frac{s_2}{s_1}
  &=
  \frac
  {\sum_{r=1}^{m} \rho_{NC,\Delta \phi_{qr}}
    \exp \Bab{
      \delta (n-1)
      \bab{
        (n-1) c + n \beta(\Delta \phi_{qr}) (1 - 2 \alpha)
        - \frac{n-2}{2} B_0
      }
    }
  }
  {\sum_{r=1}^{m} \rho_{NC,\Delta \phi_{qr}}}
  \\
  &=
  e^{\delta (n-1)^2 c}
  \frac
  {\sum_{r=1}^{m}
    \pab{
      \rho_{NC,\Delta \phi_{qr}}^{(0)}
      +
      \rho_{NC,\Delta \phi_{qr}}^{(1)}
    }
    \Bab{1 +
      \delta (n-1)
      \bab{
        n \beta(\Delta \phi_{qr}) (1 - 2 \alpha)
        - \frac{n-2}{2} B_0
      }
    }
  }
  {\sum_{r=1}^{m} \pab{
    \rho_{NC,\Delta \phi_{qr}}^{(0)}
    +
    \rho_{NC,\Delta \phi_{qr}}^{(1)}
  }}
  + \order{B_0^2}
  \\
  &=
  e^{\delta (n-1)^2 c}
  \Biggl\{
  \frac
  {\sum_{r=1}^{m}
    \pab{
      \rho_{NC,\Delta \phi_{qr}}^{(0)}
      +
      \rho_{NC,\Delta \phi_{qr}}^{(1)}
    }
  }
  {\sum_{r=1}^{m} \pab{
    \rho_{NC,\Delta \phi_{qr}}^{(0)}
    +
    \rho_{NC,\Delta \phi_{qr}}^{(1)}
  }}
  \\
  &\qquad
  +
  \frac
  {\sum_{r=1}^{m} \rho_{NC,\Delta \phi_{qr}}^{(0)}
    \delta (n-1)
    \bab{
      n \beta(\Delta \phi_{qr}) (1 - 2 \alpha)
      - \frac{n-2}{2} B_0
    }
  }
  {\sum_{r=1}^{m} \rho_{NC,\Delta \phi_{qr}}^{(0)}
  }
  \Biggr\}
  + \order{B_0^2}
  \\
  &=
  e^{\delta (n-1)^2 c}
  \Bab{
    1
    - \delta (n-1) \frac{n-2}{2} B_0
    +
    \frac{\delta n (n-1)}{m} (1 - 2 \alpha)
    \sum_{r=1}^{m} \beta(\Delta \phi_{qr})
  }
  + \order{B_0^2}
  \\
  &=
  \exp \Bab{\delta (n-1) \bab{
    (n-1) c
    - \frac{n-2}{2} B_0
    +
    \frac{n (1 - 2 \alpha)}{m}
    \sum_{r=1}^{m} \beta(\Delta \phi_{qr})
  }}
  + \order{B_0^2}
  \\
  &=
  \exp \Bab{ \delta (n-1) \bab{
    (n-1) c
    - \frac{n-2}{2} B_0
    +
    \frac{n (1 - 2 \alpha)}{2} \beta_0
  }}
  + \order{B_0^2}
\end{align*}
Where, in the last line, we used
the definition of
$\beta(\Delta \phi_{qr}) = \beta_0 \bab{1 + \cos(\Delta \phi_{qr})}/2$
to write
$\sum_{r=1}^m \beta(\Delta \phi_{qr})
= \beta_0 \sum_{r=1}^m \Bab{1 + \cos\bab{2 \pi (q-r)/m}}/2
= m \beta_0/2$
since the cosine sum gives zero.

Finally, using the fact that $m s_1$ and $m s_2$
are the probabilities of $C$ and $N$ fixation, respectively,
we find the probability of communicative fixation as
\begin{equation}
  m s_1 = \frac{1}{1 + s_2/s_1}
  \label{eq:full_analytic}
\end{equation}
with $s_2/s_1$ given by \cref{eq:full_analytic_frac},
and the asymptotic, small $B_0/c$ approximation given by
\begin{equation}
  m s_1 = \frac{1}{1 + s_2/s_1}
  \approx
  \frac{1}
  {1 + \exp \Bab{\delta (n-1) \bab{
    (n-1) c
    - \frac{n-2}{2} B_0
    +
    \frac{n (1 - 2 \alpha)}{2} \beta_0
    }}
  }
  \label{eq:analytic_first_term}
\end{equation}

\section{Majority game type criteria}
For \cref{fig:game-type}, we plotted the plurality game type.
While having all or most of the players with the same phase
easily makes the $\Delta \phi = 0$ game-type the plurality,
it is more challenging to find conditions
when the $\Delta \phi = 0$ game-type is \emph{not} the plurality.
This occurs when the number of edges \emph{between} groups
is greater than the number of edges \emph{amongst} groups.
In \cref{sec:discussion}, we described conditions under which
the $\Delta \phi = 0$ game type would \emph{not} be the most common game type
for a complete graph.
Here, we will show the derivations leading to those conditions.

\subsection{Multiple equal sized groups}
\label{sec:multiple_equal_groups}
Let us assume the $N$ total players are evenly divided
into $p$ groups.
Then, the $\Delta \phi \neq 0$ game types
arise from the edges between the $p$ groups.
Thus, we maximize the proportion of these $\Delta \phi \neq 0$ games
by having the same $\Delta \phi$ between as many groups as possible.
In other words, we want the phases of the $p$ groups
to be evenly distributed on the unit circle with phases $2 \pi j/p$
for $j \in [0,p-1]$.

Since there are $N/p$ players in each group,
the number of edges within a group is $(N/p)(N/p - 1)/2$.
Since there are $p$ groups, the number of $\Delta \phi = 0$ edges are
$N (N/p - 1)/2$.

The next most-frequent game type will be that corresponding to
$\Delta \phi = 2 \pi/p$:
that is, games between adjacent groups on the unit circle.
Since there are $p$ groups,
$N/p$ nodes in each group,
and each node connects to all $N/p$ nodes in the group
$2 \pi/p$ phase ahead of it,
the total number of $\Delta \pi = 2 \pi/p$ connections
are $(N/p)^2 p = N^2/p$.

It is easy to see that the number of $\Delta \phi = 2\pi/p$ edges,
$N^2/p$,
is always greater than the number of $\Delta \phi = 0$ edges,
$N(N/p - 1)/2$.
Therefore, when there are $p$ equal sized groups
with phases differing by $2\pi/p$,
the $\Delta \phi = 0$ game will \emph{not} be the plurality.

\subsection{Two unequal groups}
\label{sec:two_unequal_groups}
Next, we consider the case where we have two groups, $A$ and $B$,
with different phases and unequal sizes.
We denote the number of players in group $A$ by $N_A$.

The number of edges amongst group $A$ is $N_A (N_A - 1)/2$.
Likewise, the number of edges amongst group $B$ is
$(N-N_A) (N - N_A - 1)/2$.
To find the number of edges between groups $A$ and $B$,
we observe that each of the $N_A$ nodes in group $A$
connects to each of the $N - N_A$ nodes in group $B$,
giving $N_A (N - N_A)$ total edges.

Thus, the number of $\Delta \phi = 0$ edges
equal the number of $\Delta \phi \neq 0$ edges when
\begin{equation*}
  N_A (N_A - 1)/2 + (N - N_A) (N - N_A - 1)/2 = N_A (N - N_A)
  \implies \frac{N_A}{N} = \frac{1}{2} \pm \frac{1}{2 \sqrt{N}}
\end{equation*}
Therefore, when the fraction of players in one group is greater than
$\frac{1}{2} + \frac{1}{2\sqrt{N}}$,
then $\Delta \phi = 0$ will be the majority game.
But when the fraction of players in each group is within
$\pab{1/2 - 1/2\sqrt{N}, 1/2 + 1/2\sqrt{N}}$,
then the $\Delta \phi \neq 0$ game is the majority.

\subsubsection{Incomplete graphs}
\label{sec:two_unequal_groups_incomplete}
If the graph is not complete,
but every node has approximately the same degree $k$
and the two populations are randomly distributed,
then we can modify the above derivation.
In particular, since the populations are randomly distributed,
going from the complete graph with degree $N-1$
to the incomplete graph with degree $k$
is equivalent to keeping only a $k/(N-1)$ subset
of the original edges.
However, this means both the number of $\Delta \phi = 0$ edges
and the number of $\Delta \phi \neq 0$ edges
are scaled down by a factor of $k/(N-1)$.
This scaling does not change the values of $N_A/N$
where the two are equal, \ie $1/2 \pm 1/2\sqrt{N}$.


\end{document}

# vim: spelllang=en_gb
