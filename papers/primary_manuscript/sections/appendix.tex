\section{Communicative fraction with broken symmetry}
Here, we will extend the derivation of the communicative fraction
in \citet{tripp2022evolutionary} to include an asymmetry $\alpha$.
First, we use our $\alpha$-dependent payoff matrix, \cref{eq:payoff-matrix},
to calculate the $\alpha$-dependent average payoff functions.
For simplicity, we will define $\beta(\Delta \phi) = \beta_0 f(\Delta \phi)$.
Then, we have
\begin{align}
  \pi_E(k) &= \ab(\frac{k-1}{n-1}) \pab{B_0 - c}
                + \ab(\frac{n-k}{n-1}) \ab(\beta(\Delta\phi) 2\alpha - c) \\
           &= \frac{1}{n-1}
               \ab(k \ab(B_0 - 2 \alpha \beta(\Delta\phi))
                 + 2 \alpha n \beta(\Delta\phi) - B_0 - (n-1) c)
\end{align}
and
\begin{align}
  \pi_F(k) &= \ab(\frac{k}{n-1}) \ab(\beta(\Delta\phi) 2 (1-\alpha))
                + \ab(\frac{n-k-1}{n-1}) \ab(0) \\
           &= \ab(\frac{k}{n-1}) 2 (1-\alpha) \beta(\Delta\phi)
\end{align}
where $k \in [0,N]$ is the number of $E$-players
and $\Delta \phi$ is the phase difference between the $E$ and $F$
strategies.
Note that the first term of $\pi_E$ corresponds
to the $E$-$E$ payoff, which has $\Delta \phi = 0$,
hence $B_0 f(\Delta \phi) = B_0$.

Now, we follow the derivation of \citet{tripp2022evolutionary}
and consider two populations with strategies
$E=(C,\phi_j)$ and $F=(N,\phi_k)$.
The derivation in section 4 of the appendix
from \citet{tripp2022evolutionary}
carries through directly until equation 16
where a key parameter $\gamma_k$ is defined
as the ratio of the average $F$ payoff to average $E$ payoff.
Here, we make the small correction of replacing
the average fitness $f = \exp(\delta \pi)$ (used in the appendix)
with the total fitness $f = \exp(\delta (n-1) \pi)$
for a node with $(n-1)$ neighbors
(used in the main text and plots).
Thus, equation 16 becomes
\begin{align*}
  \gamma_k &= \frac{f_F(k)}{f_E(k)} \\
           &= \exp[\delta (n-1) (\pi_F(k) - \pi_E(k))]
           \\
           &= \exp\ab[\delta \ab(
    k 2 (1-\alpha) \beta(\Delta\phi)
    - k \ab(B_0 - 2 \alpha \beta(\Delta\phi))
                 - 2 \alpha n \beta(\Delta\phi) + B_0 + (n-1) c)] \\
      &= \exp\ab[\delta \ab(
    \ab(2 \beta(\Delta\phi) - B_0) k
                 + B_0 - 2 \alpha n \beta(\Delta\phi) + (n-1) c)]
\end{align*}

Continuing to follow the derivation of \citet{tripp2022evolutionary},
we can calculate the ratio of fixation probabilities $\rho_F/\rho_E$ as
\begin{equation}
  \begin{aligned}
    \frac{\rho_F}{\rho_E} &= \prod_{k=1}^{n-1} \gamma_k \\
           &= \exp \ab\{
                \delta
                \sum_{k=1}^{n-1}
                \ab[
    \ab(2 \beta(\Delta\phi) - B_0) k
                 + B_0 - 2 \alpha n \beta(\Delta\phi) + (n-1) c
                 ]
                 \} \\
           &= \exp \ab\{
                \delta
                \ab[
                \ab(2 \beta(\Delta\phi) - B_0) \frac{n(n-1)}{2}
                 + \ab(B_0 - 2 \alpha n \beta(\Delta\phi) + (n-1) c) (n-1)
                 ]
                 \} \\
           &= \exp \ab\{
                \delta (n-1)
                \ab[
                (n-1) c + n \beta(\Delta\phi) (1 - 2 \alpha)
                - \frac{n-2}{2} B_0
                 ]
                 \}
  \end{aligned}
  \label{eq:fixation_prob_ratio}
\end{equation}
Notice, now the ratio of $\rho_F/\rho_E$ depends of $\Delta \phi$,
unlike in the $\alpha=1/2$ case considered by \citet{tripp2022evolutionary}.

Similarly, since $\rho_E + \rho_F = 1$, we have
\begin{equation}
  \begin{aligned}
    \rho_E &= \frac{1}{1+\rho_F/\rho_E}
           = \frac{1}{1+\sum_{j=1}^{n-1} \prod_{k=1}^j \gamma_k} \\
           &= \ab(1+\sum_{j=1}^{n-1} \prod_{k=1}^j \exp\ab[\delta \ab(
           \ab(2 \beta(\Delta\phi_{qr}) - B_0) k
           + B_0 - 2 \alpha n \beta(\Delta\phi_{qr}) + (n-1) c)])^{-1}
                 \\
           &= \ab(1+\sum_{j=1}^{n-1} \exp\ab[\delta \sum_{k=1}^j \ab(
           \ab(2 \beta(\Delta\phi_{qr}) - B_0) k
           + B_0 - 2 \alpha n \beta(\Delta\phi_{qr}) + (n-1) c)])^{-1}
                 \\
           &= \ab(1+\sum_{j=1}^{n-1} \exp\ab[\delta \ab(
           \ab(2 \beta(\Delta\phi_{qr}) - B_0) j(j+1)/2
           + j \ab(B_0 - 2 \alpha n \beta(\Delta\phi_{qr}) + (n-1) c))])^{-1}
                 \\
           &= \ab(1+\sum_{j=1}^{n-1} \exp\ab[\delta \ab(
           \ab(\beta(\Delta\phi_{qr}) - B_0/2) j^2
           + j \ab(B_0/2 + \beta(\Delta\phi_{qr}) (1 - 2 \alpha n)  + (n-1) c))])^{-1}
  \end{aligned}
  \label{eq:comm_fixation_prob}
\end{equation}

Following section 5 in the appendix of \citet{tripp2022evolutionary},
we apply these results from the two-population case
to the ($m=20$ phases) multi-population case with low-mutation rate
by identifying the fixation probability $\rho_{CN,\Delta \phi_{qr}}$
of an $(N,\phi_r)$ strategy invading an $(C,\phi_q)$ strategy
with $\rho_{CN,\Delta \phi_{qr}} = \rho_F$,
and likewise
$\rho_{NC,\Delta \phi_{rq}} = \rho_E$.
Equation 47 of \citet{tripp2022evolutionary}
gives the ratio of the stationary state eigenvalues $s_1/s_2$ as
\begin{align}
  \frac{s_2}{s_1} &= \frac
    {\sum_{r=1}^{m} \rho_{CN,\Delta \phi_{rq}}}
    {\sum_{r=1}^{m} \rho_{NC,\Delta \phi_{qr}} }
  \\
  &=
  \frac
  {\sum_{r=1}^{m} \rho_{NC,\Delta \phi_{qr}}
    \exp \Bab{
      \delta (n-1)
      \bab{
        (n-1) c + n \beta(\Delta \phi_{qr}) (1 - 2 \alpha)
        - \frac{n-2}{2} B_0
      }
    }
  }
  {\sum_{r=1}^{m} \rho_{NC,\Delta \phi_{qr}}}
  \\
  &=
  \bab{
    \frac
    {\sum_{r=1}^{m} \frac
      {\exp \Bab{ \delta (n-1)
        \bab{
          (n-1) c + n \beta(\Delta \phi_{qr}) (1 - 2 \alpha) - \frac{n-2}{2} B_0
        }}
      }
      {1 + \sum_{j=1}^{n-1} \exp\Bab{
        \delta \bab{
         \pab{\beta(\Delta \phi_{qr}) - B_0/2} j^2
         + j \pab{B_0/2 + \beta(\Delta \phi_{qr}) (1 - 2 \alpha n) + (n-1) c}
        }}
      }
    }
    {\sum_{r=1}^{m} \frac{1}
      {1 + \sum_{j=1}^{n-1} \exp\Bab{
        \delta \bab{
         \pab{\beta(\Delta \phi_{qr}) - B_0/2} j^2
         + j \pab{B_0/2 + \beta(\Delta \phi_{qr}) (1 - 2 \alpha n) + (n-1) c}
        }}
      }
    }
  }
  \label{eq:full_analytic_frac}
\end{align}
The first equality used \cref{eq:fixation_prob_ratio}
and the second equality used \cref{eq:comm_fixation_prob}.
Unlike in the $\alpha = 1/2$ symmetric case, we cannot factor the
exponential component out of the sum and cancel the $\rho_{CN,\Delta
\phi_{qr}}$ terms.
However, we can asymptotically expand $s_2/s_1$ for small $B_0/c \ll 1$;
additionally, since all of our simulations use $\beta \propto B$,
we also assume $\beta/c \ll 1$.
Finally, to simplify the calculation,
we define the asymptotic expansion of
$\rho_{NC,\Delta \phi_{qr}} \coloneqq
\rho_{NC,\Delta \phi_{qr}}^{(0)}
+
\rho_{NC,\Delta \phi_{qr}}^{(1)}
+
\order{B_0^2}
$
where
$
\rho_{NC,\Delta \phi_{qr}}^{(i)}
$
depends only on terms of total order $i$ in $B_0$ and $\beta$.

Then, to first order in $B_0$ and $\beta$,
we have
\begin{align*}
  \frac{s_2}{s_1}
  &=
  \frac
  {\sum_{r=1}^{m} \rho_{NC,\Delta \phi_{qr}}
    \exp \Bab{
      \delta (n-1)
      \bab{
        (n-1) c + n \beta(\Delta \phi_{qr}) (1 - 2 \alpha)
        - \frac{n-2}{2} B_0
      }
    }
  }
  {\sum_{r=1}^{m} \rho_{NC,\Delta \phi_{qr}}}
  \\
  &=
  e^{\delta (n-1)^2 c}
  \frac
  {\sum_{r=1}^{m}
    \pab{
      \rho_{NC,\Delta \phi_{qr}}^{(0)}
      +
      \rho_{NC,\Delta \phi_{qr}}^{(1)}
    }
    \Bab{1 +
      \delta (n-1)
      \bab{
        n \beta(\Delta \phi_{qr}) (1 - 2 \alpha)
        - \frac{n-2}{2} B_0
      }
    }
  }
  {\sum_{r=1}^{m} \pab{
    \rho_{NC,\Delta \phi_{qr}}^{(0)}
    +
    \rho_{NC,\Delta \phi_{qr}}^{(1)}
  }}
  + \order{B_0^2}
  \\
  &=
  e^{\delta (n-1)^2 c}
  \Biggl\{
  \frac
  {\sum_{r=1}^{m}
    \pab{
      \rho_{NC,\Delta \phi_{qr}}^{(0)}
      +
      \rho_{NC,\Delta \phi_{qr}}^{(1)}
    }
  }
  {\sum_{r=1}^{m} \pab{
    \rho_{NC,\Delta \phi_{qr}}^{(0)}
    +
    \rho_{NC,\Delta \phi_{qr}}^{(1)}
  }}
  \\
  &\qquad
  +
  \frac
  {\sum_{r=1}^{m} \rho_{NC,\Delta \phi_{qr}}^{(0)}
    \delta (n-1)
    \bab{
      n \beta(\Delta \phi_{qr}) (1 - 2 \alpha)
      - \frac{n-2}{2} B_0
    }
  }
  {\sum_{r=1}^{m} \rho_{NC,\Delta \phi_{qr}}^{(0)}
  }
  \Biggr\}
  + \order{B_0^2}
  \\
  &=
  e^{\delta (n-1)^2 c}
  \Bab{
    1
    - \delta (n-1) \frac{n-2}{2} B_0
    +
    \frac{\delta n (n-1)}{m} (1 - 2 \alpha)
    \sum_{r=1}^{m} \beta(\Delta \phi_{qr})
  }
  + \order{B_0^2}
  \\
  &=
  \exp \Bab{\delta (n-1) \bab{
    (n-1) c
    - \frac{n-2}{2} B_0
    +
    \frac{n (1 - 2 \alpha)}{m}
    \sum_{r=1}^{m} \beta(\Delta \phi_{qr})
  }}
  + \order{B_0^2}
  \\
  &=
  \exp \Bab{ \delta (n-1) \bab{
    (n-1) c
    - \frac{n-2}{2} B_0
    +
    \frac{n (1 - 2 \alpha)}{2} \beta_0
  }}
  + \order{B_0^2}
\end{align*}
Where, in the last line, we used
the definition of
$\beta(\Delta \phi_{qr}) = \beta_0 \bab{1 + \cos(\Delta \phi_{qr})}/2$
to write
$\sum_{r=1}^m \beta(\Delta \phi_{qr})
= \beta_0 \sum_{r=1}^m \Bab{1 + \cos\bab{2 \pi (q-r)/m}}/2
= m \beta_0/2$
since the cosine sum gives zero.

Finally, using the fact that $m s_1$ and $m s_2$
are the probabilities of $C$ and $N$ fixation, respectively,
we find the probability of communicative fixation as
\begin{equation}
  m s_1 = \frac{1}{1 + s_2/s_1}
  \label{eq:full_analytic}
\end{equation}
with $s_2/s_1$ given by \cref{eq:full_analytic_frac},
and the asymptotic, small $B_0/c$ approximation given by
\begin{equation}
  m s_1 = \frac{1}{1 + s_2/s_1}
  \approx
  \frac{1}
  {1 + \exp \Bab{\delta (n-1) \bab{
    (n-1) c
    - \frac{n-2}{2} B_0
    +
    \frac{n (1 - 2 \alpha)}{2} \beta_0
    }}
  }
  \label{eq:analytic_first_term}
\end{equation}
