%! TeX program = lualatex
\documentclass[tikz]{standalone}

% Import custom style file containing common packages and options
\usepackage{../preamble}

% Import tikz options and preamble
% Define pgfplots classes
\pgfplotsset{simulation scatter/.style={only marks, blue}}
\pgfplotsset{full theory line/.style={orange}}
\pgfplotsset{approx theory line/.style={purple, dashed}}

\pgfplotsset{fraction communicative chart/.style={
        ymin={0},
        ymax={1},
        ytick={0.0,0.5,1.0},
        grid,
        table/x=B0,
        table/y=communicative_fraction,
        cycle list={
          simulation scatter,
          full theory line,
          approx theory line,
        },
        enlarge y limits=0.05,
    }
}

\pgfplotsset{
  harmony/.style={fill=Paired-G,draw=Paired-G},
  chicken/.style={fill=Paired-F,draw=Paired-F}, % exponential fixation time
  battle/.style={fill=Paired-E,draw=Paired-E}, % exponential fixation time
  hero/.style={fill=Paired-H,draw=Paired-H}, % exponential fixation time
  compromise/.style={fill=Paired-C,draw=Paired-C},
  concord/.style={fill=Paired-D,draw=Paired-D},
  staghunt/.style={fill=Paired-B,draw=Paired-B},
  dilemma/.style={fill=Paired-K,draw=Paired-K},
  deadlock/.style={fill=Paired-A,draw=Paired-A},
  assurance/.style={fill=Paired-J,draw=Paired-J},
  coordination/.style={fill=Paired-I,draw=Paired-I},
  peace/.style={fill=Paired-L,draw=Paired-L},
  all_communicative/.style={fill=lightgray,draw=lightgray},
  all_noncommunicative/.style={fill=darkgray,draw=darkgray},
}


\begin{document}
\begin{tikzpicture}
  \begin{groupplot}[
    width=0.4\textwidth,
    group style={group size=2 by 1,
    horizontal sep=2cm
    },
    xlabel={Asymmetry $\alpha$},
    xtick=data,
    ymin=0,
    scaled y ticks=false,
    yticklabel style={
      /pgf/number format/fixed,
      /pgf/number format/precision=3,
    },
    enlarge y limits={value=0.15,upper},
    ]
    \nextgroupplot[
    ylabel={Chimera-Like Index $\chi$},
    ]
    \addplot table [y=chimera_index] {../../../data/processed/chimeraindex/B_factor=1.5_adj_matrix_source=c-elegans_community_algorithm=covariance_covariance_cutoff=1500_nb_phases=20_payoff_update_method=single-update_selection_strength=0.2_time_steps=8000000.csv};
    \nextgroupplot[
    ylabel={Metastability Index $\lambda$},
    ]
    \addplot table [y=metastability_index] {../../../data/processed/chimeraindex/B_factor=1.5_adj_matrix_source=c-elegans_community_algorithm=covariance_covariance_cutoff=1500_nb_phases=20_payoff_update_method=single-update_selection_strength=0.2_time_steps=8000000.csv};
  \end{groupplot}
  \begin{scope}[shift={(2,-6)},rotate=90]
    % Read node coordinates and create nodes
    \csvreader[head to column names]{../../../data/processed/vertices_B_factor=1.5_adj_matrix_source=c-elegans_nb_phases=20_payoff_update_method=single-update_selection_strength=0.2_symmetry_breaking=0.0_time_step=560000_time_steps=8000000.csv}{}%
    {%
      \node[draw, circle, draw=none, index of colormap=\strategyIndex of strategycolors, fill=., inner sep=1.5] (\index) at (\x, \y) {};
    }

    \begin{scope}[on background layer]
      % Decrease total opacity to prevent high opacity of overlapping edges
      \begin{scope}[transparency group,opacity=.75]
      % Read edges and draw connections between nodes
      \csvreader[head to column names]{../../../data/processed/edges_adj_matrix_source=c-elegans.csv}{}%
      {%
        % Make individual opacity low; multiple overlapping edges can still have high opacity
        \ifthenelse{\equal{\src}{\dst}}
        { \path[in=-45,out=45,loop,->,>=latex,line width=0.1,opacity=0.2] (\src) edge (\dst);}
        { \path[->,>=latex,line width=0.1,opacity=0.2] (\src) edge (\dst);}
      }
      \end{scope}
    \end{scope}
  \end{scope}
  \begin{scope}[shift={(8,-6)},rotate=90]
    % Read node coordinates and create nodes
    \csvreader[head to column names]{../../../data/processed/vertices_B_factor=1.5_adj_matrix_source=c-elegans_nb_phases=20_payoff_update_method=single-update_selection_strength=0.2_symmetry_breaking=1.0_time_step=560000_time_steps=8000000.csv}{}%
    {%
      \node[draw, circle, draw=none, index of colormap=\strategyIndex of strategycolors, fill=., inner sep=1.5] (\index) at (\x, \y) {};
    }

    \begin{scope}[on background layer]
      % Decrease total opacity to prevent high opacity of overlapping edges
      \begin{scope}[transparency group,opacity=.75]
      % Read edges and draw connections between nodes
      \csvreader[head to column names]{../../../data/processed/edges_adj_matrix_source=c-elegans.csv}{}%
      {%
        % Make individual opacity low; multiple overlapping edges can still have high opacity
        \ifthenelse{\equal{\src}{\dst}}
        { \path[in=-45,out=45,loop,->,>=latex,line width=0.1,opacity=0.2] (\src) edge (\dst);}
        { \path[->,>=latex,line width=0.1,opacity=0.2] (\src) edge (\dst);}
      }
      \end{scope}
    \end{scope}
  \end{scope}
\end{tikzpicture}
\end{document}
