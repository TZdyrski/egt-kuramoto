%! TeX program = lualatex
\documentclass[pdflatex,lineno,referee,sn-nature]{sn-jnl}
%\documentclass[pdflatex,twocolumn,sn-nature,super]{sn-jnl}
\pdfvariable objcompresslevel=0

% Silence warning from caption about unknown document class
\usepackage{silence}
\WarningFilter*{caption}{Unknown document class}

% Memoize tikz drawings
\usepackage[extract=tex]{memoize}
\mmzset{
  padding=0pt, % Remove padding around figure for journal submission
  context={fsize={\csname f@size\endcsname}}, % Add font size to context
  verbatim,
}

% Import custom style file containing common packages and options
\usepackage{preamble}

% Include definitions and custom commands
% Define binomial nomenclatures
\setabbreviationstyle[species]{long-em-short-em}
\glssetcategoryattribute{species}{nohyper}{true}
\newabbreviation[category={species}]{celegans}{C. elegans}{Caenorhabditis elegans}
\newabbreviation[category={species}]{dmelanogaster}{D. melanogaster}{Drosophila melanogaster}

% Define custom macros
\newcommand{\im}{\mathrm{i}}
\DeclareMathOperator{\cov}{cov}

% Define custom caption separator
\DeclareCaptionLabelSeparator{bar}{|}
\captionsetup{labelsep=bar}

% Prevent floats from leaving subsection
%\pretocmd{\section}{\FloatBarrier}{}{}
%\pretocmd{\subsection}{\FloatBarrier}{}{}
%\pretocmd{\subsubsection}{\FloatBarrier}{}{}

% Add comma separator for numbers
\sisetup{group-minimum-digits=3}
\sisetup{group-separator = {,}}

\defasforeign[aka]{a.k.a.} % Define abbreviation of aka


% Import tikz options and preamble
% Define pgfplots classes
\pgfplotsset{simulation scatter/.style={only marks, blue}}
\pgfplotsset{full theory line/.style={orange}}
\pgfplotsset{approx theory line/.style={purple, dashed}}

\pgfplotsset{fraction communicative chart/.style={
        ymin={0},
        ymax={1},
        ytick={0.0,0.5,1.0},
        grid,
        table/x=B0,
        table/y=communicative_fraction,
        cycle list={
          simulation scatter,
          full theory line,
          approx theory line,
        },
        enlarge y limits=0.05,
    }
}

\pgfplotsset{
  harmony/.style={fill=Paired-G,draw=Paired-G},
  chicken/.style={fill=Paired-F,draw=Paired-F}, % exponential fixation time
  battle/.style={fill=Paired-E,draw=Paired-E}, % exponential fixation time
  hero/.style={fill=Paired-H,draw=Paired-H}, % exponential fixation time
  compromise/.style={fill=Paired-C,draw=Paired-C},
  concord/.style={fill=Paired-D,draw=Paired-D},
  staghunt/.style={fill=Paired-B,draw=Paired-B},
  dilemma/.style={fill=Paired-K,draw=Paired-K},
  deadlock/.style={fill=Paired-A,draw=Paired-A},
  assurance/.style={fill=Paired-J,draw=Paired-J},
  coordination/.style={fill=Paired-I,draw=Paired-I},
  peace/.style={fill=Paired-L,draw=Paired-L},
  all_communicative/.style={fill=lightgray,draw=lightgray},
  all_noncommunicative/.style={fill=darkgray,draw=darkgray},
}


% Set data search path
\pgfplotsset{
    table/search path={../../data/processed},
}
\newcommand\dataSearchPath{../../data/processed}

\begin{document}

\title{Evolutionary Kuramoto dynamics unravels origins of chimera states
in neural populations}

\author{
\href{https://orcid.org/0000-0003-3039-172X}{Thomas Zdyrski}$^{1}$,
\href{https://orcid.org/0000-0002-0281-2868}{Scott Pauls}$^{1}$,
and
\href{https://orcid.org/0000-0001-8252-1990}{Feng Fu}$^{1,2}$
}

\affil{$^{1}$Department of Mathematics, Dartmouth College, Hanover, NH 03755\\
$^{2}$Department of Biomedical Data Science, Geisel School of Medicine at Dartmouth, Lebanon, NH 03756
}

%\keywords{neural synchronization, game theory, evolutionary graph theory}
%\pw{}

%TC:ignore
\abstract{
Neural synchronization is central to cognition.
However, incomplete synchronization often produces chimera states,
where coherent and incoherent dynamics coexist.
While previous studies have explored such patterns
using networks of coupled oscillators,
it remains unclear why neurons commit to communication
or how chimera states persist.
Here, we investigate the coevolution of neuronal phases
and communication strategies on directed, weighted networks,
where interaction payoffs depend on phase alignment
and may be asymmetric due to unilateral communication.
We find that both connection weights and directionality influence the stability
of communicative strategies---and, consequently,
full synchronization---as well as the strategic nature of neuronal interactions.
Applying our framework to the C. elegans connectome,
we show that emergent payoff structures, such as the snowdrift game,
underpin the formation of chimera states.
Our computational results demonstrate a promising neurogame-theoretic perspective,
leveraging evolutionary graph theory to shed light on mechanisms
of neuronal coordination beyond classical synchronization models.
}
%TC:endignore

\maketitle
\glsresetall
%\tableofcontents

\section{Introduction}
%\subsection{Impact of structure on game evolution}
Evolutionary game theory (EGT) is the application
of game theory to evolving populations
of individuals with behavioral strategies.
This tool is useful for studying how local interaction rules
yield large-scale patterns such as cooperation~\citep{sigmund1999evolutionary}
and has found use in fields including international politics, ecology,
and protein folding~\citep{traulsen2023future}.
Studies~\citep{cohen2009evolutionary,antonioni2017coevolution,tripp2022evolutionary}
have even applied EGT to non-reproducing neurons
by viewing neuron plasticity as an evolutionary process
where firing patterns change and are ``learned'' over time.
Evolutionary \emph{graph} theory places evolutionary games
on graphs to investigate the role of structure in population evolution.
Prior studies have found that structure
qualitatively changes game evolution.
For instance, cooperation is enhanced by
small-degree nodes~\citep{ohtsuki2006simple}
or unidirectional edges~\citep{su2022evolution}
and suppressed by weighted edges~\citep{bhaumik2024constant}.
Thus, the study of evolving, structured populations
should account for the effects of
incompleteness, directedness, and weightedness.

%\subsection{Evolutionary Kuramoto dynamics}
Kuramoto networks are groups of coupled oscillators
where the coupling strength depends sinusoidally
on the oscillators' phase difference.
These networks are popular
models for neuron behaviour~\citep{cabral2011role,deng2024chimera}
because they exhibit tunable synchronization.
Prior studies~\citep{antonioni2017coevolution}
have modelled interneuron communication
with Kuramoto oscillators and EGT
using the prisoner's dilemma game type.
Other studies~\citep{tripp2022evolutionary}
generalized this approach to include dynamically changing game types
by introducing an evolutionary Kuramoto (EK) model
to show how the relationship between communication benefit and cost
influences the emergence of synchronized communication
or non-communication regimes.

%\subsection{Chimera states}
One intriguing aspect of neuron oscillations
is the observation of chimera states~\citep{majhi2019chimera}.
These states exhibit the simultaneous existence
of coherent and disordered phases~\citep{abrams2004chimera}.
Previous studies have proposed
chimera states as
a key component of human cognitive organization~\citep{bansal2019cognitive},
a facilitator of spiking and bursting phases~\citep{santos2017chimera},
and an outcome of modular networks~\citep{hizanidis2016chimera}.
Despite the observed importance of these chimera states,
the factors that give rise to coherent/disordered coexistence
remain incompletely characterized.

%\subsection{\glsfmtshort{celegans} connectome}
The nematode \gls{celegans}
is a model organism in neuroscience due to
its simple brain connectome~\citep{cook2019whole}
of only \num{302} neurons.
Despite their simplicity, models of the \gls{celegans} brain
still display a wide array of complex phenomena including
topologically-central rich clubs
crucial to motor neurons~\citep{towlson2013rich},
phenomenological connections to control theory~\citep{yan2017network},
and chimera states~\citep{hizanidis2016chimera}.

%\subsection{Outline}
In this paper, we will generalize
the evolutionary Kuramoto model
by adding an asymmetry between
the communicator and non-communicator payoffs
and applying it to incomplete, directed, and weighted graphs.
These generalizations will allow us to apply the EK model
to the \gls{celegans} connectome
and analyze its communicativeness properties.

Specifically, we will use evolutionary graph theory to connect single-neuron payoffs
with network-wide chimera-like states on the \gls{celegans} connectome graph.
To this end, this work will investigate the applicability
of evolutionary Kuramoto dynamics and chimera states
to the \gls{celegans} connectome model.
The model's simulation statistics will allow us to connect
the individual neurons' payoffs to the large-scale communicativeness properties
while also investigating the influence of the \gls{celegans} connectome
shape on chimera-like states.
This network shows stronger disorder than well-mixed populations do.
Additionally, we find the network has low metastability ($\lambda < \num{0.01}$)
but a large chimera-like index ($\xi \approx \num{0.13}$).

Our results connect individual neuron fitness and non-trivial brain topology
to chimera-like brain states.
Given current technological limitations with direct measurement of in-vivo neuron activity,
frameworks like ours create testable hypotheses connected to the theory's assumptions.
Our computational model represents a simple yet versatile framework
to illuminate the influence of neural connectivity on chimera-like brain states
beyond classical synchronization models.

\section{Results}\label{sec:results}

\subsection{Model}
\begin{figure*}
  \centering
  \begin{nomemoize} % NiceMatrix uses the tikz option `remember picture`;
                    % this automatically aborts memoization, so disable it to save time
    \begin{subcaptiongroup}
      \stackinset{l}{10pt}{t}{2.5in}%
        {\phantomcaption\label{fig:payoff_matrix}\captiontext*{}}{%
      \stackinset{l}{2.5in}{t}{10pt}%
        {\phantomcaption\label{fig:player_interactions}\captiontext*{}}{%
      \stackinset{l}{10pt}{t}{10pt}%
        {\phantomcaption\label{fig:graph_well-mixed}\captiontext*{}}{%
      {\includestandalone{tikz/model-setup}}%
    }}}
    \end{subcaptiongroup}
  \end{nomemoize}
  \caption{
    \textbf{
      Evolutionary Kuramoto dynamics with weighted neural connectivity.
    }
    \protect{\subref{fig:graph_well-mixed}}
    The graph of a well-mixed population with $N=20$ players
    where each pair of players is connected by a directed edge in each direction.
    \protect{\subref{fig:player_interactions}}
    The connectivity between two sample players, $i$ and $j$,
    showing directed, weighted edges $w_{ij}$ and $w_{ji}$.
    Each player has a strategy-phase pair $(s, \phi)$,
    with the strategy being either communicative ($C$) or non-communicative ($N$)
    and the phase being $\phi = 2\pi k/m$ with $m$ the number of phases
    and $k \in 0,\ldots,m-1$.
    \protect{\subref{fig:payoff_matrix}}
    The payoff matrix shows the reward the row-player $(C, \phi_i)$
    receives after playing a game with the column-player $(N, \phi_j)$
    depending on each player's strategy and
    their relative phase difference $\Delta \phi$
    and assuming either player can switch strategy and phase to the other's.
  }\label{fig:connectivity}
\end{figure*}

We extend the EK model in two ways:
placing the player network on a directed, weighted graph;
and introducing a payoff asymmetry.
First, we represent a well-mixed population
as a complete graph in \cref{fig:graph_well-mixed}
with players represented by nodes and games by edges.
We generalize this to directed graphs
where game (edge) payoffs only flow to the head players (nodes).
We can also represent bidirectional games with
a pair of edges in both directions,
as shown in \cref{fig:player_interactions}.
For reference, the \gls{celegans} connectome
has \num{38} self-loops, \num{669} bidirectional edge pairs,
and \num{2331} unpaired edges.
Finally, we interpret the \gls{celegans} integer-valued
edge weights as the number of connections between nodes,
so these weights scale each payoff.

We also generalize the payoff structure to incorporate
an asymmetry between communicators and non-communicators.
We characterized each player (node) by its
strategy $s_i$---either communicative $C$ or non-communicative $N$---
and its phase $\phi_i$---taking one of \num{20} evenly-spaced values
between $0$ and $2 \pi$.
Communicative players always pay a cost $c$.
A game's benefit depends on whether the number of communicators is
zero (\num{0} benefit), one ($\beta_0$ mixed benefit),
or two ($B_0 > \beta_0$ joint benefit).
We incorporate the relative phase difference $\Delta \phi$
by modulating these benefits with a sinusoidal Kuramoto coupling
$f(\Delta \phi) = [1+\cos(\phi_j - \phi_i)]/2$.
Our new contribution is an asymmetry $\alpha \in [0,1]$
that biases the mixed benefit $\beta_0$ when exactly one of the players is communicative.
We multiply the $C$ player's mixed benefit by $2\alpha$
and the $N$ player's benefit by $2(1 - \alpha)$.
Using $\alpha = 0.5$ reproduces the symmetric case,
and increasing (decreasing) $\alpha$ allows us
to promote (suppress) communicativeness.
The payoff matrix in \cref{fig:payoff_matrix} summarizes these payoffs
for $CN$ or $NC$ interactions;
$CC$ interactions are always double-cooperation games
while $NN$ interactions are always neutral games.

\subsection{Parameter space}
\begin{figure*}
  \centering
  \begin{subcaptiongroup}
    \stackinset{l}{2.7in}{t}{0pt}%
      {\phantomcaption\label{fig:phase-diagram-beta_alpha}\captiontext*{}}{%
    \stackinset{l}{10pt}{t}{0pt}%
      {\phantomcaption\label{fig:phase-diagram-beta_B}\captiontext*{}}{%
    {\includestandalone{tikz/phase-diagram}}%
  }}
  \end{subcaptiongroup}
  \caption{
    \textbf{
      Payoff asymmetry enriches neural interactions
      well beyond the classic prisoner's dilemma game type.
      Shown are region plots illustrating the diverse range
      of game types that neural populations can engage in
      during evolutionary dynamics.
    }
    Slices of the three-parameter game-type phase diagram
    in the
    \protect{\subref{fig:phase-diagram-beta_B}}
    $\beta$-$B_0$ plane
    and
    \protect{\subref{fig:phase-diagram-beta_alpha}}
    $\beta$-$\alpha$ plane.
    For two players with phase difference $\Delta \phi$,
    $B_0$
    and
    $\beta = \beta_0 [1 + \cos(\Delta \phi)]/2$.
    The legend displays the game type corresponding to each color.
    The cost $c$ is \num{0.1};
    for
    \protect{\subref{fig:phase-diagram-beta_B}}
    $\alpha = 0.5$,
    and for
    \protect{\subref{fig:phase-diagram-beta_alpha}}
    $B_0/c = 1.5$.
    The white dots represent the $m=20$ potential phase differences
    as well as the restriction
    \protect{\subref{fig:phase-diagram-beta_B}}
    $\beta_0 = \num{0.95} B_0$.
    or
    \protect{\subref{fig:phase-diagram-beta_alpha}}
    $\alpha \in [0,0.25,0.5,0.75,1]$.
  }\label{fig:phase-diagram}
\end{figure*}

One of the key aspects of the EK model
is that multiple $2 \times 2$ game types can emerge
among the players during the population's evolution.
These game types~\citep{bruns2015names} include
dilemma (\aka{} ``prisoner's dilemma''),
deadlock (dilemma with the middle two payoffs swapped,
\aka{} ``anti-prisoner's dilemma''),
chicken (\aka{} ``hawk-dove'' or ``snowdrift''),
compromise (deadlock with the lowest two payoffs swapped,
\aka{} ``anti-chicken''),
battle (\aka{} ``leader'' or ``Bach or Stavinsky''),
hero (battle with the lowest two payoffs swapped),
or
harmony (game with strong incentive alignment).
See Section 2 of the supplementary material
for order graphs depicting the payoff structure of each game type.
We can visualize these games by looking at two-dimensional slices
of the three-dimension phase space
characterized by $\alpha$, $\beta \coloneqq \beta_0 f(\Delta \phi)$,
and $B_0$.
\cref{fig:phase-diagram-beta_B} shows a $\beta$-$B_0$ slice of phase space
and generalizes figure 1 of a prior study~\citep{tripp2022evolutionary},
while \cref{fig:phase-diagram-beta_alpha} shows a $\beta$-$\alpha$ slice.
Each straight line of white dots represent permissible values of the $m = 20$ phases
$\Delta \phi_i$ across our simulations
and highlight which regions of parameter space are accessible.

\subsection{Complete graphs}\label{sec:complete_graph}

\begin{figure*}
  \centering
  \begin{subcaptiongroup}
    \stackinset{l}{3.2in}{t}{2.5in}%
      {\phantomcaption\label{fig:time-series_well-mixed_alpha-1}\captiontext*{}}{%
    \stackinset{l}{0.8in}{t}{2.5in}%
      {\phantomcaption\label{fig:time-series_well-mixed_alpha-075}\captiontext*{}}{%
    \stackinset{l}{3.2in}{t}{10pt}%
      {\phantomcaption\label{fig:time-series_well-mixed_alpha-0}\captiontext*{}}{%
    \stackinset{l}{0.8in}{t}{10pt}%
      {\phantomcaption\label{fig:multi-comm-frac}\captiontext*{}}{%
    {\includestandalone{tikz/well-mixed}}%
  }}}}
  \end{subcaptiongroup}
  \caption{
    \textbf{
      Impact of symmetry breaking on neural synchronization
      in well-mixed populations.
    }
    \protect{\subref{fig:multi-comm-frac}}
    Time-averaged fraction of players that are communicative as a function
    of the maximum joint benefit $B_0$
    for different values of the payoff asymmetry $\alpha$.
    The marks represent the simulation results
    and the lines represent the theory predictions.
    The $B_0$ step size is \num{0.04},
    and the simulations ran for \num{2E8} time steps.
    \protect{\subref{fig:time-series_well-mixed_alpha-0}}--\protect{\subref{fig:time-series_well-mixed_alpha-1}}
    Scatter plots where
    the left vertical axis of each subplot gives
    the instantaneous fraction of players that are communicative
    as a function of time.
    These points are color-coded
    grey if all players are communicative (``all-C'') or
    black if all players are non-communicative (``all-N'');
    otherwise, they are colored according
    to the plurality game type as indicated in the legend.
    The right vertical axis corresponds to the order parameter $\rho$
    (\pcref{eq:order_parameter}), in magenta, as a function of time.
    The asymmetry is
    \protect{\subref{fig:time-series_well-mixed_alpha-0}}
    $\alpha = 0$,
    \protect{\subref{fig:time-series_well-mixed_alpha-075}}
    $\alpha = 0.75$,
    and
    \protect{\subref{fig:time-series_well-mixed_alpha-1}}
    $\alpha = 1$,
    maximum joint benefit is $B_0 = 0.15$,
    and the simulations ran for \num{8E5} time steps.
    For all four
    \protect{\subref{fig:multi-comm-frac}}--\protect{\subref{fig:time-series_well-mixed_alpha-1}}
    subplots,
    the network topologies are the
    $N=20$ well-mixed population,
    the maximum mixed benefit $\beta_0$ is $\num{0.95} B_0$ at each step,
    the selection strength is $\delta=0.2$,
    the mutation rate is $\mu=\num{1E-4}$,
    and
    the cost $c$ is \num{0.1}.
  }\label{fig:well-mixed}
\end{figure*}

First, we will explore the influence of the newly introduced asymmetry
on a $N=20$-player, well-mixed population.
\Cref{fig:multi-comm-frac} compares the frequency of communicative strategies
$f_{\text{comm}}$ to the maximum joint benefit $B_0$.
The marks represent the simulation results
and the lines represent the full analytic
result (Eq.\ 10 in supplementary material) for a well-mixed population.
In general, $f_{\text{comm}}$ is low for small $B_0$,
rises to $f_{\text{comm}} = 0.5$ at some break-even $B_0$,
and plateaus to $f_{\text{comm}} \approx 1$ for large $B_0$.
We can validate our model by comparing our
$\alpha = 0.5$ case with a previous study~\citep{tripp2022evolutionary}
to observe qualitatively similar results,
with our break-even $B_0 \approx 0.21$
corresponding to their $B_0 = 2 (N-1) c/(N-2) = 0.21$ break-even condition.
We also see that increasing (decreasing)
the asymmetry $\alpha$ dilates (stretches) this sigmoid function
in the $B_0$ direction.
This $\alpha$-dependence is reasonable,
as increasing $\alpha$ corresponds to biasing the payoff
in a mixed $C$-$N$ interaction towards the communicative partner.

While \cref{fig:multi-comm-frac} displays
the time-averaged system state,
it is also useful to investigate the time-dependent variations.
\Crefrange{fig:time-series_well-mixed_alpha-0}{fig:time-series_well-mixed_alpha-1}
depict the frequency
of communicative strategies $f_{\text{comm}}$
as scatter plots of time on the left vertical axis
for different values of the asymmetry $\alpha$.
These time-series points are color-coded
grey if all players are communicative or
black if all players are non-communicative;
otherwise, the points are colored according
to the plurality game type, as indicated in the legend.
On the right vertical axes,
magenta line plots depict the order parameter
$\rho \in [0,1]$
given by \cref{eq:order_parameter}.
These plots shows the results
for a well-mixed population of $N=20$ players,
and the simulations ran
for \num{8E5} time steps with
selection strength $\delta = 0.2$,
asymmetry $\alpha = 0.75$,
mutation rate $\mu=\num{1E-4}$,
cost $c = \num{0.1}$,
maximum joint benefit $B_0 = 0.15$,
and maximum mixed benefit $\beta_0$ is $\num{0.95} B_0$.

\Cref{fig:time-series_well-mixed_alpha-0} shows the time-variation
when the system heavily favors non-communicative players
with $\alpha = 0$.
After an initial disordered, deadlock-type game (light blue),
the system synchronizes in the non-communicative regime
(black line at $f_{\text{comm}} = 0$).
\Cref{fig:time-series_well-mixed_alpha-075} depicts an asymmetry $\alpha = 0.75$
that moderately encourages communicativeness.
The system is synchronized in a communicative state
(gray line at $f_{\text{comm}} = 1$) \SI{92}{\percent} of the time,
with unstable excursions to synchronized, non-communicative states
(black line at $f_{\text{comm}} = 0$)
and disordered, hero-type game states
(orange line with $0 < f_{\text{comm}} < 1$).
This \SI{92}{\percent} communicative rate is higher than the expected \SI{75}{\percent}
from \cref{fig:multi-comm-frac} (for $B_0 = 0.15$ and $\alpha = 0.75$),
likely due to the small timespan shown in
\cref{fig:time-series_well-mixed_alpha-075}.
Finally, \cref{fig:time-series_well-mixed_alpha-1} shows the
$\alpha = 1$ case where communication is heavily incentivized.
In contrast to the \subref{fig:time-series_well-mixed_alpha-0} $\alpha = 0$ case,
the system is virtually always synchronized in the communicative regime,
with only brief excursions to deadlock (light blue)
or harmony (light orange) type games.
We note that all three
\subref{fig:time-series_well-mixed_alpha-0}-\subref{fig:time-series_well-mixed_alpha-1}
scenarios are virtually always synchronized;
the order parameter $\rho$ (magenta) is almost always $\rho = 1$,
with only occasional dips.
The temporary drops in communicative frequency $f_{\text{comm}}$
and order parameter $\rho$ are likely the result of mutations
which occur, on average, every $1/\mu = \num{1E4}$ time steps.

\subsection{\glsfmtshort{celegans} graphs}\label{sec:elegans_graph}

Here, we consider the weighted, directed hermaphroditic \gls{celegans}
chemical connectome~\citep{cook2019whole}.
For reference, \cref{fig:graph} shows the $N = 300$ nodes and the directed edges
(we exclude the two unconnected neurons CANL/CANR).
The nodes are colored according to their strategy at a particular time step,
with blues representing communicative strategies
and reds representing non-communicative ones,
with different shades corresponding to different phases $\phi$.
We note the chimera-like character
of the large, synchronized group of red nodes
coexisting with disordered neighboring nodes.

\begin{figure*}
  \centering
  \begin{subcaptiongroup}
    \stackinset{l}{0.65in}{t}{3.4in}%
      {\phantomcaption\label{fig:time-series_celegans-full}\captiontext*{}}{%
    \stackinset{l}{3in}{t}{4.2in}%
      {\phantomcaption\label{fig:comm-frac_celegans-undirected}\captiontext*{}}{%
    \stackinset{l}{3in}{t}{2.2in}%
      {\phantomcaption\label{fig:comm-frac_celegans-unweighted}\captiontext*{}}{%
    \stackinset{l}{3in}{t}{0.2in}%
      {\phantomcaption\label{fig:comm-frac_celegans-full}\captiontext*{}}{%
    \stackinset{l}{0.65in}{t}{0.5in}%
      {\phantomcaption\label{fig:graph}\captiontext*{}}{%
    {\includestandalone{tikz/c-elegans}}%
  }}}}}
  \end{subcaptiongroup}
  \caption{
    \textbf{
      Rise of chimera states in \gls{celegans} neural network.
    }
    \protect{\subref{fig:graph}}
    The network topology for the \gls{celegans}
    $N=300$ weighted, directed connectome.
    The colors represent the $2m = 40$ strategies
    at a particular time step;
    blue colors are communicative,
    red colors are non-communicative,
    and shades represent different phases $\phi$.
    \protect{\subref{fig:comm-frac_celegans-full}}--\protect{\subref{fig:comm-frac_celegans-undirected}}
    Time-averaged fraction of players that are communicative as a function
    of the maximum joint benefit $B_0$.
    The network topologies are the
    \protect{\subref{fig:comm-frac_celegans-full}}
    weighted, directed \gls{celegans} connectome,
    \protect{\subref{fig:comm-frac_celegans-unweighted}}
    unweighted, directed \gls{celegans} connectome,
    and
    \protect{\subref{fig:comm-frac_celegans-undirected}}
    weighted, undirected \gls{celegans} connectome.
    The $B_0$ step size is \num{0.04} and
    the simulations ran for \num{2E8} time steps.
    \protect{\subref{fig:time-series_celegans-full}}
    Scatter plot with the same axes and coloring
    as \protect{\crefrange{fig:time-series_well-mixed_alpha-0}{fig:time-series_well-mixed_alpha-1}}.
    The maximum joint benefit is $B_0 = 0.15$ and
    the simulation ran for \num{8E5} time steps.
    For all cases,
    the selection strength is $\delta=0.2$,
    the asymmetry is $\alpha=0.75$,
    the mutation rate is $\mu=\num{1E-4}$,
    the maximum mixed benefit $\beta_0$ is $\num{0.95} B_0$ at each step,
    and
    the cost $c$ is \num{0.1}.
  }\label{fig:c-elegans}
\end{figure*}

Next, to quantify the observations of \cref{fig:graph},
\cref{fig:comm-frac_celegans-full} shows the fraction of players
using communicative strategies $f_{\text{comm}}$ averaged across
the entire simulation of \num{2E8} time steps as a function
of the maximum joint benefit $B_0$
for the \gls{celegans} connectome network topologies.
All subplots use a selection strength of $\delta=0.2$
and an asymmetry of $\alpha=0.75$.

We note that the behaviour
of the \subref{fig:comm-frac_celegans-full} \gls{celegans} case
is qualitatively distinct
from the $\alpha = 0.75$ well-mixed case in \cref{fig:multi-comm-frac}.
The communicative fraction $f_{\text{comm}}$
is flatter for $B_0 \le 0.08$,
has a steep jump to \num{0.77} at $B_0 = 1.6$,
and decreases to a horizontal asymptote around \num{0.60}.
We can isolate the cause of this deviation
from the \cref{fig:multi-comm-frac} well-mixed
behavior by looking at variations to the \cref{fig:comm-frac_celegans-full}
\gls{celegans} network topology.
First, the \subref{fig:comm-frac_celegans-unweighted}
\emph{directed}, unweighted connectome
is qualitatively similar to the
\subref{fig:multi-comm-frac}
well-mixed case with a monotonic increase
from low communicativeness for $B_0 < 0.1$
to full communicativeness for $B_0 \ge 0.2$.
This implies that directedness
plays only a small role in the qualitative shape
of the \subref{fig:comm-frac_celegans-full} full \gls{celegans} case.
Conversely,
the \subref{fig:comm-frac_celegans-undirected}
\emph{weighted}, undirected connectome
looks similar to the \subref{fig:comm-frac_celegans-full}
full \gls{celegans} case,
displaying the same plateau at $f_{\text{comm}} \approx 0.7$,
though the peak around $B_0 = 0.15$ is less pronounced.
This similarity implies that the connectome's edge weights
cause most of the deviation
between the \subref{fig:comm-frac_celegans-full} full \gls{celegans}
case and the \cref{fig:multi-comm-frac} well-mixed case.

We can also investigate the time-evolution of the \gls{celegans} system
using the same parameters as the
\crefrange{fig:time-series_well-mixed_alpha-0}{fig:time-series_well-mixed_alpha-1}
well-mixed case but with the \gls{celegans} connectome graph.
Compared to the well-mixed case,
the \subref{fig:time-series_celegans-full} \gls{celegans} case
depicts a far more heterogeneous population.
Here, the population never stabilizes
to a fully communicative or non-communicative state.
Instead, both its communicative frequency and order parameter
stay between \SIrange{40}{70}{\percent}.
The mean communicative frequency $f_{\text{comm}} \approx \SI{60}{\percent}$
is similar to the expected $f_{\text{comm}} \approx \SI{70}{\percent}$
from \cref{fig:comm-frac_celegans-full} with $B_0 = 0.15$;
the discrepancy comes from the stochasticity
in this small, \num{8E5} time-step sample.
Finally, the \gls{celegans} case is less stable to mutations
than the \cref{fig:time-series_well-mixed_alpha-075} well-mixed case:
instead of stable synchronized states with transient impulses,
the \gls{celegans} case depicts disordered states with discontinuous offsets.

\subsection{Chimera-like index}
\begin{figure*}
  \centering
  \begin{subcaptiongroup}
    \stackinset{l}{3in}{t}{2.5in}%
      {\phantomcaption\label{fig:graph_celegans_asymmetry1}\captiontext*{}}{%
    \stackinset{l}{0.71in}{t}{2.5in}%
      {\phantomcaption\label{fig:graph_celegans_asymmetry0}\captiontext*{}}{%
    \stackinset{l}{3in}{t}{15pt}%
      {\phantomcaption\label{fig:metastability_index}\captiontext*{}}{%
    \stackinset{l}{0.7in}{t}{15pt}%
      {\phantomcaption\label{fig:chimera_index}\captiontext*{}}{%
    {\includestandalone{tikz/chimera-states}}%
  }}}}
  \end{subcaptiongroup}
  \caption{
    \textbf{
      Characterizing chimera states.
    }
    \protect{\subref{fig:chimera_index}}
    The chimera-like index (\protect{\pcref{eq:chimera_index}})
    and
    \protect{\subref{fig:metastability_index}}
    metastability index (\protect{\pcref{eq:metastability_index}})
    as functions of the asymmetry $\alpha$
    for the
    weighted, directed \gls{celegans} connectome.
    To calculate the indices,
    we split the graph into two communities according to the nodes'
    relative covariance (\cf{} \cref{sec:chimera-metastability-def}).
    The simulations ran for \num{8E6} time steps.
    \protect{\subref{fig:graph_celegans_asymmetry0}}--\protect{\subref{fig:graph_celegans_asymmetry1}}
    A snapshot of the \gls{celegans} connectome where
    blue colors are communicative,
    red colors are non-communicative,
    and shades represent different phases $\phi$.
    The asymmetry is
    \protect{\subref{fig:graph_celegans_asymmetry0}}
    $\alpha = 0$
    or
    \protect{\subref{fig:graph_celegans_asymmetry1}}
    $\alpha = 1$.
    For all cases,
    the selection strength is $\delta=0.2$,
    the mutation rate is $\mu=\num{1E-4}$,
    the cost $c$ is \num{0.1},
    the maximum joint benefit $B_0$ is \num{0.15},
    and
    the maximum mixed benefit $\beta_0$ is $\num{0.95} B_0$.
  }\label{fig:chimera-index}
\end{figure*}

Time-lapse animations of the system's time evolution
show a subset of the nodes exhibiting high synchronicity
with others remaining disordered: this is characteristic of a chimera state.
Animations depicting this phenomena are available as supplementary videos,
and an interactive website for exploring the full data set
through plots and animations is provided in the ``Code availability''
section at the end of the paper.
In order to quantify this chimera-like effect, we will investigate
the chimera-like index $\chi$ (\pcref{eq:chimera_index})
and metastability index $\lambda$ (\pcref{eq:metastability_index}).
As discussed in \cref{sec:chimera-metastability-def},
the chimera-like index
measures the coherence difference between communities of players,
is equal to the time-averaged community covariance,
and has a theoretical maximum value
of $M/[4(M-1)] = \num{0.5}$~\citep{shanahan2010metastable}.
Conversely, the metastability index
represents how often the system transitions
between synchronicity and disorder,
is equal to the community-average of the temporal covariance,
and has a practical maximum of \num{0.08}~\citep{shanahan2010metastable}.
\Cref{fig:chimera-index} shows the
\subref{fig:chimera_index} chimera-like index $\chi$
and
\subref{fig:metastability_index} metastability index $\lambda$
as functions of the asymmetry $\alpha$.
These simulations ran with the same parameters as the
\crefrange{fig:comm-frac_celegans-full}{fig:comm-frac_celegans-undirected}
\gls{celegans} time-series data.

The metastability displayed in \cref{fig:chimera-index}
is less than \num{0.01}, much smaller
than the theoretical maximum of \num{0.08}.
This implies that the system has low metastability
and spends most of its time at a nearly constant
synchronicity $\rho_m(t)$.
Furthermore, while the metastability is higher (more stability variations)
at $\approx \num{0.01}$ for the $\alpha = 0$ case,
the metastability drops precipitously for larger $\alpha$,
always staying below \num{0.004} implying that
higher asymmetry $\alpha$ makes the system more stable.
Conversely, the chimera-like index $\chi$
indicates a high chimeric quality
with values (\numrange{0.06}{0.13})
nearly a quarter of the theoretical maximum (\num{0.5}).
Given that we average the chimera-like index over time,
the system's deviation from a complete chimera state
arises from both imperfect separation of the coherent/disordered populations
as well as time fluctuations in the chimeric quality.
While we observe that $\chi$ has a maximum at $\alpha = \num{0.75}$,
we caution that this is likely the result of calculating the communities
based on the $\alpha = \num{0.75}$ covariances.
Nevertheless, the high $\chi$ value for all asymmetries $\alpha$
implies that the strong chimeric character is intrinsic
rather than an artifact of our $\alpha = \num{0.75}$ choice.

\subsection{Game types}
\begin{figure*}
  \centering
  \begin{subcaptiongroup}
    \stackinset{l}{2.4in}{t}{1.8in}%
      {\phantomcaption\label{fig:game-type_celegans-full-only-mixed-games}\captiontext*{}}{%
    \stackinset{l}{0in}{t}{1.8in}%
      {\phantomcaption\label{fig:game-type_well-mixed-only-mixed-games}\captiontext*{}}{%
    \stackinset{l}{2.4in}{t}{0.0in}%
      {\phantomcaption\label{fig:game-type_celegans-full}\captiontext*{}}{%
    \stackinset{l}{0in}{t}{0.0in}%
      {\phantomcaption\label{fig:game-type_well-mixed}\captiontext*{}}{%
    {\includestandalone{tikz/game-types}}%
  }}}}
  \end{subcaptiongroup}
  \caption{
    \textbf{
      Influence of payoff asymmetry on game types played.
    }
    The plurality game type amongst all player interactions
    expressed as a fraction of all games played
    for different values of the asymmetry $\alpha$.
    The game types are color-coded according to the legend;
    additionally, ``all-C'' and ``all-N'' represent when the population
    is entirely synchronized to communicativeness or non-communicativeness, respectively.
    The network topologies are the
    \protect{\subref{fig:game-type_well-mixed}}
    $N=20$ well-mixed population
    and
    \protect{\subref{fig:game-type_celegans-full}}
    weighted, directed \gls{celegans} connectome.
    Similarly,
    \protect{\subref{fig:game-type_well-mixed-only-mixed-games}},
    \protect{\subref{fig:game-type_celegans-full-only-mixed-games}}
    depict the plurality game type when only considering mixed game types
    for the
    \protect{\subref{fig:game-type_well-mixed-only-mixed-games}}
    $N=20$ well-mixed population
    and
    \protect{\subref{fig:game-type_celegans-full-only-mixed-games}}
    weighted, directed \gls{celegans} connectome.
    A blue inset below the
    \protect{\subref{fig:game-type_well-mixed-only-mixed-games}}
    well-mixed panel shows the magnified lower region.
    The selection strength is $\delta=0.2$,
    the mutation rate is $\mu=\num{1E-4}$,
    the cost $c$ is \num{0.1},
    the maximum joint benefit $B_0$ is \num{0.15},
    the maximum mixed benefit $\beta_0$ is $\num{0.95} B_0$,
    and the simulations ran for \num{8E6} time steps.
  }\label{fig:game-type}
\end{figure*}

Next, we seek to understand the types of games
that nodes play during the population's evolution.
In \cref{fig:game-type}, we investigate the plurality game type
amongst all player interactions at each time step~(\cf{} \cref{sec:plurality_game_type}).
\Cref{fig:game-type} shows the fraction of time
that a given game type is the plurality for different values
of the asymmetry $\alpha$ for
\subref{fig:game-type_well-mixed}
$N=20$ well-mixed,
and
\subref{fig:game-type_celegans-full}
\gls{celegans} connectome
network topologies.
The \subref{fig:game-type_well-mixed-only-mixed-games} well-mixed case
shows that system is fully synchronized
for over \SI{99.6}{\percent} of the runtime
across every asymmetry $\alpha$.
For the $< \SI{1}{\percent}$ of the time when the well-mixed system is
\emph{not} synchronized,
the magnified inset (blue) below
\cref{fig:game-type_well-mixed-only-mixed-games}
shows that the other notable plurality game types are
dilemma (\SI{0.1}{\percent} for $\alpha = 0, 0.25$)
chicken (\SI{0.2}{\percent} for $\alpha = 0.5$),
hero (\SI{0.2}{\percent} for $\alpha = 0.75$),
and
deadlock (\SI{0.1}{\percent} for $\alpha = 1$).
In contrast, the
\subref{fig:game-type_celegans-full} \gls{celegans} connectome system
is essentially never synchronized.
Instead, disordered game types (\ie{} states other than ``all-C'' or ``all-N'')
dominate,
with the dominant game type changing as the asymmetry varies:
dilemma for $\alpha = 0, 0.25$, chicken for $\alpha = 0.5$,
hero for $\alpha = 0.75$, and deadlock for $\alpha = 1$.
We note that these dominant game types
are the same as the most-frequent \emph{disordered} plurality games,
for each $\alpha$, as those played in
the \cref{fig:game-type_well-mixed} well-mixed setup (\cf{} blue inset).
Additionally, the \gls{celegans} system displays
some game-type heterogeneity, with
a second game type (deadlock) being the plurality
\SIrange{1}{14}{\percent} of the time
for asymmetries $\alpha \le 0.5$.
In line with \cref{fig:chimera-index}, the game types of chicken and hero,
with bigger chimera-like index for $\alpha$ values at $0.5$ and $0.75$, respectively,
underlie the neuronal interactions.
Despite their subtle difference in the ranking order of payoff values,
both game types allow coexistence in a snowdrift-game-like fashion.
Such payoff structure is known to have
an exponential fixation time~\citep{antal2006fixation},
thereby giving rise to the persistence of chimera states.
Overall, we see that graph structure decreases synchronization
and asymmetry influences the dominant game types, which, in turn,
underpin the formation of chimera states.

\section{Discussion}\label{sec:discussion}

%\subsection{Game-type dynamics}
The distribution of game types displayed in \cref{fig:game-type}
deserves additional discussion.
The \cref{fig:game-type_well-mixed} well-mixed case
is virtually always synchronized,
as corroborated by the order parameter $\rho \approx 1$ in
\crefrange{fig:time-series_well-mixed_alpha-0}{fig:time-series_well-mixed_alpha-1}.
Furthermore,
the $\alpha$-dependence of the
communicative synchronization (``all-C'') fraction
approximates the communicative frequency $f_{\text{comm}}$
in \cref{fig:multi-comm-frac} for $B_0=0.15$.

We can compare the \cref{fig:game-type_well-mixed}
$\alpha = 0.5$ case
to the time-series results
in previous studies~\citep{tripp2022evolutionary}.
Their weak-selection $\delta = 0.2$
results also show the system spending
virtually all of its time in a synchronized state.
When not synchronized, their system had plurality
chicken (denoted ``snowdrift'' therein)
and staghunt (denoted ``coordination'',
not to be confused with the coordination-type game) games.
Indeed, \cref{fig:phase-diagram-beta_B}
shows that staghunt, dilemma, and chicken
would be the only game types accessible for $\alpha = 0.5$,
which is corroborated in the inset below
\cref{fig:game-type_well-mixed-only-mixed-games}.

Next, we consider the distribution of disordered games
in \cref{fig:game-type} as functions of $\alpha$.
Recall that the most-frequent disordered games
in \cref{fig:game-type_well-mixed}
and \cref{fig:game-type_celegans-full} are identical,
so this discussion applies to both setups.
In the \cref{fig:phase-diagram-beta_alpha}
$B-\alpha$ parameter plane,
each of the five $\alpha$ values
correspond to a horizontal line of white dots.
Additionally, the warm colors (red, orange, and pink)
represent game types (chicken, hero, and battle, respectively)
with exponentially slow fixation times~\citep{antal2006fixation}.
So we might expect any constant-$\alpha$ line (white dots) intersecting
a warm-colored region
to be dominated by the corresponding game in \cref{fig:game-type}.
This is the case with $\alpha = 0.5, 0.75$,
but not for $\alpha = 1$,
which mainly shows deadlock type games.
Instead, we note that every $\alpha$ case \emph{is} dominated
by the game-type of the rightmost white dots
representing nearly-synchronized $\Delta \phi \approx 0$ configurations.

We can explain the prevalence of nearly-synchronous game types
by considering game-plurality requirements.
We easily obtain nearly-synchronized $\Delta \phi \approx 0$
game-type pluralities whenever the graph is largely synchronized.
Finding circumstances when $\Delta \phi \approx 0$ game types
are \emph{not} the plurality is more challenging,
but we identify two such scenarios in section 3 of the supplementary material:
configurations with $p$ equal-sized groups
with phases that differ by $2 \pi/p$ (\cf{} supplementary material section 3.1);
or when the population is almost evenly split between two groups
with fractions between $0.5 \pm 1/(2 \sqrt{N})$
(\cf{} supplementary material section 3.2.1).
Given the low mutation rate, it is unlikely to have
multiple populations with roughly equal sizes,
so the evenly-split scenario is more likely.
For the $N = 20$ well-mixed case, this corresponds
to \SIrange{39}{62}{\percent}, or \numrange{8}{12} players.
But the magenta order parameter in \cref{fig:time-series_well-mixed_alpha-075}
shows that the well-mixed system spends most of its time
with fewer than the required \num{8} invaders.
This explains why the well-mixed graph's disordered games
are still nearly-synchronized.

These graph-degree conditions also help explain the
game distributions for the
\cref{fig:game-type_celegans-full} \gls{celegans} graph.
Each $\alpha$-value is still dominated by the nearly-synchronized
game types, as these are still the most likely.
Nevertheless, the $\alpha = 0$, $0.25$, and $0.5$ cases
show significant levels of non-synchronized game types
(\SI{13}{\percent}, \SI{3}{\percent}, and \SI{1}{\percent} deadlock games,
respectively).
While the $0.5 \pm 1/(2 \sqrt{N})$ condition above seems to imply it
should be \emph{harder} for the \gls{celegans} graph to show
non-synchronized deadlock games since
the number of players $N=300$ is larger,
we note that the \gls{celegans} average degree $\overline{d}$
is \emph{smaller} (\num{12.36} \vs{} \num{19}).
Indeed, repeating the derivation for (random regular) incomplete graphs
(\cf{}  supplementary material section 3.2.2)
shows that we must replace $N$ with $N \overline{d}/(N-1)$,
widening the range to \SIrange{36}{64}{\percent}.
The order parameter in \cref{fig:time-series_celegans-full}
is usually in this range,
allowing for non-synchronized game types like deadlock.

%\subsection{Chimera states}
Next, we will discuss the observed chimera states.
While the chimera-like quality is inherently time-dependent
(\cf{} \pcref{eq:chimera_index})
carefully chosen snapshots can still convey some of the chimera-like aspects.
\Cref{fig:graph_celegans_asymmetry0} shows a snapshot of the communication strategies
for asymmetry $\alpha = 0$ using the same color scheme as \cref{fig:graph}.
We notice a large region of light red representing
a synchronized group of non-communicators,
as well as a mix of disordered neighbors.
This coexistence of strong synchronization and disorder
is consistent with the high
chimera-like index $\chi$ observed in \cref{fig:chimera-index} for $\alpha = 0$.
Likewise, the $\alpha = 0.75$ case depicted in \cref{fig:graph}
has an even higher $\chi = \num{0.13}$,
consistent with the coexistence of a large,
synchronized non-communicative group (red)
and disordered neighbors (varying phases/colors).
In contrast,
\cref{fig:graph_celegans_asymmetry1} shows a snapshot for $\alpha = 1$,
for which we expect a lower chimera-like index of $\chi = \num{0.06}$.
Indeed, we see that the large communicative (red) group
in \cref{fig:graph_celegans_asymmetry1}
or \cref{fig:graph} has fractured into two different
communicative (blue) groups, lowering the chimeric quality.
Therefore, these snapshots give a glimpse of the chimera-like quality
in the \gls{celegans} populations.
Finally, we note that \cref{fig:game-type_celegans-full}
shows that these $\alpha \le 0.75$ scenarios with high chimera-like index
are dominated by
exponentially slow fixation time games (chicken, harmony, and dilemma; warm colors).
In contrast, the $\alpha=1$ scenario with low chimera-like index
is dominated by a game type (deadlock) that is not exponentially slow.
This connection between exponentially-slow fixation time games and high chimera index
seems suggestive and requires further investigation.

Another study of the \gls{celegans} connectome also found
chimera-like states using an entirely non-game-theory tool,
modular neural networks~\citep{hizanidis2016chimera}.
That study found a maximum chimera-like index of \num{0.12},
in agreement with our value of \num{0.13}, despite the different network models
of the \gls{celegans} connectome considered.
Our study reveals that, in addition to the connectivity structure,
the emergent game type characterizing neuronal interactions is primarily responsible
for the chimera states.

%\section{Conclusion}
Chimera states of networked oscillators
exhibit coexisting
synchronized and disordered populations,
are present
in brains~\citep{santos2017chimera,bansal2019cognitive},
and may be pivotal in human cognition~\citep{bansal2019cognitive}.
Prior studies~\citep{deng2024chimera} investigated
chimera-like brain states in \gls{dmelanogaster}
using sinusoidally-coupled Kuramoto oscillators,
a frequent model for neuron dynamics~\citep{cabral2011role}.
However, the small-scale evolutionary factors
leading to chimera-like states remains poorly understood.
In this paper, we extended the evolutionary Kuramoto model
to include weighted, structured interaction graphs
and an asymmetry between
the communicator and non-communicator payoffs.
We first applied this model to a well-mixed population
and found that increasing (decreasing) the asymmetry
inhibits (promotes) communicativity.
Next, we applied the model to
a family of graphs deriving from the \gls{celegans} connectome.
This revealed that the graph's weightedness
has a much stronger influence on the communication frequency
than the graph's directedness does.
Finally, while the well-mixed players have homogeneous populations
that occasionally switch between synchronized communicative
and  non-communicative states,
the \gls{celegans} population remains far more heterogenous
with a stable, chimeric mix of disordered game types.

It is straightforward to apply the EK model
to families of generative brain network models~\citep{betzel2016generative}.
The limited number of empirical graphs analyzed in this study
hinders the identification of exact relationships
between the communicative fraction and graph properties
such as edge degree, weight, and directionality;
applying the model to parameterized families of graphs could allow
for fine-tuning the parameters to extract these relations.
Furthermore, the parameter-space analysis provided clues
regarding the types of games the neurons played,
but additional study is necessary to determine a causal relation.
Studying the parameter space attractors
could provide this connection
by revealing the long-time game distributions
and yielding important implications for the neurons' fitness landscape.
Finally, since this study only considers a single species' connectome,
it is unclear if our findings regarding the dependence
of chimeric activity on graph structure are generalizable.
Therefore, it is of interest to investigate this connection further
by applying our EK setup to other model organisms,
such as that of
\glsxtrlong{dmelanogaster}~\citep{schlegel2024whole}.

A key takeaway from this study was the importance
of edge weight on communicativity;
this has important parallels to neural computing,
where edge weights are a primary driver of functionality.
Additionally, the observation of chimeric states
arising from such simple neurogame-theoretic models
implies that the nature of neuron interactions is likely
a key component in producing these critical brain states.
Overall, evolutionary graph theory allowed us
to connect low-level payoff details for individual neurons
to high-level phenomena such as chimera-states,
and this model could serve as a valuable computational framework
for clarifying the influence of network structure on neural dynamics.

%TC:ignore
\section{Methods}\label{sec:methods}

\subsection{Game setup}\label{sec:game_setup}
We model the system of evolving, coupled oscillators
by discretizing the $2\pi$ phase angle into $m$ discrete phases $2 \pi j/m$
for $j \in 0, \ldots, m-1$.
The game's strategy space is the outer product of the $m$ phases
($m = 20$ for our simulations) and $2$ communicative choices,
communicative $C$ and non-communicative $N$.
For a given pair of phases, $\phi_i$ and $\phi_j$, the game is specified
by the payoff matrix
\begin{equation}
\begin{bNiceMatrix}[first-col,first-row,hvlines]
  & C, \phi_i & N, \phi_j \\
  C, \phi_i & B_0 f(0) - c & 2 \alpha \beta_0 f(\Delta \phi) - c \\
  N, \phi_j & 2 (1 - \alpha) \beta_0 f(\Delta \phi) & 0
\end{bNiceMatrix}
\label{eq:payoff-matrix}
\end{equation}
with $f(\Delta \phi) = [1+\cos(\phi_j - \phi_i)]/2$
(\pcref{fig:payoff_matrix}).
Here, $B_0$, $\beta_0$, $\alpha$, and $c$ are fixed parameters
defining the game.
The cost $c$ represents the penalty paid by communicative players,
and $B_0$ and $\beta_0$ are the maximum benefits paid with
joint $C$-$C$ communicators and mixed $C$-$N$ players, respectively.
The phase-dependent function $f(\Delta \phi)$ encodes
the influence of phase mismatches.
Finally, the benefit asymmetry $\alpha \in [0,1]$ breaks the symmetry
between the payoff for the communicator and the non-communicator
when exactly one player is communicative.

\subsection{Population setup}\label{sec:pop_setup}

Given $N$ players, we associate a pair
of weighted, directed graphs to the population.
First, we use an interaction graph with $N$ nodes representing players
and weighted, directed edges representing games between players.
Second, we implement a reproduction graph with the $N$ nodes
still representing players
but the edges now corresponding to the ability of nodes to replace one another.
For simplicity, our reproduction graphs are identical to the interaction graph
with a single self-loop added to each node.
These self-loops are necessary in the reproduction graph
to ensure that each node has positive indegree
as required by the Moran process described in the next section.

\subsection{Birth-death Moran process}\label{sec:evo_setup}
% TODO: Bd is hugely different from death-Birth dynamics (dB);
% this dB is more favorable for cooperation
% Try this instead?
The population is updated according to a birth-death Moran process
with exponential fitness~\citep{lieberman2005evolutionary}.
On each turn, the following steps are performed.
First, each edge in the interaction graph corresponds to a game
between head node $i$ and tail node $j$,
the edge's payoff $\pi_{ji}$ is scaled by the edge weight $w_{ji}$,
and the relevant payoff $w_{ji} \pi_{ji}$ is awarded to the head node only.
Since the \gls{celegans} edge weights are integers,
we can also interpret each edge weight as the number of games played
between the two nodes.
\Cref{fig:player_interactions} shows an illustration of this process
for a single pair of interacting players connected by a pair of
weighted, directed edges.
The total fitness for node $i$ is the exponential of the product
between the selection strength $\delta$
and the sum of payoffs to node $i$,
or $f_i = \exp(\delta \sum_j w_{ji} \pi_{ji})$ with the sum
over all edges inwardly incident to node $i$.
Then, a single focal node is chosen for reproduction
with probability proportional to the node's fitness $f_i$.
Finally, a node is chosen for replacement amongst the birth node's out-neighbors
with probability proportional to the reproduction graph's edge weight.
With mutation probability $\mu$,
the death node is replaced by a player with a uniformly random strategy;
otherwise, it is replaced by a player with the same strategy as the birth node.
This birth-death process is repeated for each turn.

\subsection{Communication frequency}
We define the frequency of communicative strategies
$f_{\text{comm}}(t)$ as the fraction of players employing
a communicative strategy $C$ at a given time step.
We also define the time-averaged communicative frequency
$f_{\text{comm}}$
by averaging $f_{\text{comm}}(t)$ over the entire simulation.
For simulation times long compared to the mutation turnover time
$T_{\text{turn}} \gg N/\mu$,
the initial, random distribution of strategies will be negligible
and the time-average will correspond to the long-time limit.
In section 2 of the supplementary material,
we derive an analytic expression (Eq.\ 10 of supplementary materials) for
$f_{\text{comm}}$ in the well-mixed case
by incorporating the benefit asymmetry $\alpha$.

\subsection{Game type nomenclature}
\label{sec:game-type-nomenclature}
Every edge of the interaction graph
defines a game between the two players it connects.
Using their relative phase difference $\Delta \phi$,
we can calculate the payoff matrix.
By comparing the order of each of the four entries,
we determine the game type using
a topological taxonomy~\citep{bruns2015names}.
Using this taxonomy, we calculate the ordinal rank of the four entries
in the \cref{eq:payoff-matrix} payoff matrix
and assign a unique name (\eg{} dilemma, deadlock, chicken, \etc{})
to each strict, symmetric game type.
However, \cref{fig:phase-diagram-beta_alpha}
shows that some games lie on the border between two game types,
such as when $\alpha = 1$ or $B/c = 0$
(bordering game types for $B/c > 0$ and $B/c < 0$, not shown).
The taxonomy~\citep{bruns2015names} classifies these non-strict games
according the number and location of ties (``high tie'', ``middle tie'',
``double tie'', \etc{}).
It also defines a convention for choosing one of the neighboring game types
to get a binomial nomenclature (\eg{} ``high harmony'', ``mid compromise'',
``double coordination'', \etc{}).
We follow the same convention for choosing a neighboring game
but drop the tie-indicator to keep our figure legends simple.
Specifically, referring to \cref{fig:phase-diagram-beta_alpha},
the $\alpha = 1$ tie between deadlock and compromise games
formally corresponds to ``low [dead]lock'',
but we label it as deadlock;
similarly, the tie between assurance and staghunt
is ``mid staghunt'', but we denote it as staghunt.
Likewise, all of the $B/c = 0$ payoff matrices
correspond to the ``double harmony'' game type,
but we denote them as simply ``harmony''.

\subsection{Plurality game type}\label{sec:plurality_game_type}
At each time step, we calculate the game type for each edge
of the interaction graph.
We determine the game type by creating a two-by-two payoff matrix of possibilities
where both players have the hypothetical option of switching to the other player's
strategy/phase pairing.
Using the taxonomy discussed in \cref{sec:game-type-nomenclature},
we then assign a game type to that interaction.
We then identify the plurality game type across all player interactions,
where each edge's count is weighted by its edge weight.
Then, we calculate the frequency of this ``plurality game'' across
all time steps of a given simulation to determine the distribution
of games commonly played.
Note that an edge's game type is independent
of the players' communication strategy (N or C) and only depends on
their dynamically-evolving relative phase $\Delta \phi$
and the fixed game parameters $c$, $B_0$, $\beta_0$, and $\alpha$.
Furthermore, this metric is only sensitive to the
plurality game and therefore provides no information
on the presence/absence of minority game types.

\subsection{Order parameter}
Given that the Kuramoto system of coupled oscillators
inspired this evolutionary game model,
we also define the standard Kuramoto order parameter:
\begin{equation}
  \rho = \frac{1}{N} \abs{\sum_{j=1}^N e^{\im \phi_j}}
  \label{eq:order_parameter}
\end{equation}
This parameter ranges from zero to one, inclusive,
and represents how coherent the population is,
with $\rho = 1$ for fully coherent and $\rho = 0$ fully disordered.

\subsection{Chimera-like index and metastability index}\label{sec:chimera-metastability-def}
To compare with a previous analysis~\citep{hizanidis2016chimera} of
chimera-like states \gls{celegans} models,
we define a pair of indices related to
chimera-like quality and metastability~\citep{shanahan2010metastable}.
First, we organize the game's nodes
into $M$ disjoint communities.
We split the nodes into $M=2$ communities $C_m$
by taking a subset
(the first \num{8E4} steps for computations ease)
of the simulation results
for the \gls{celegans} simulation with asymmetry $\alpha = \num{0.75}$.
We then calculate the covariance matrix $K_{i,j}$ of the strategy indices,
ensuring that communicative ($C$, $\phi_i$)
and non-communicative ($N$, $\phi_i$) are treated distinctly.
Next, we calculate the row-wise covariance-sums  $\sum_i K_{i,j}$.
After finding the node with the maximum covariance-sum
$\cov_{\text{max}} \coloneqq \max_j{\sum_i K_{i,j}}$,
we form a community by collecting all nodes $j$ with covariance-sum
at least \num{1500}, $\sum_i K_{i,j} \ge 1500$.
Note that \num{1500} is approximately \SI{81}{\percent}
of the maximum covariance-sum,
so this ensure we only group nodes with high covariance-sum.
Finally, we place all the remaining nodes in a second community.

With these disjoint communities $C_m$, we then calculate
the time-dependent, community-wise order parameter $\rho_m(t)$
as
\begin{equation}
  \rho_m(t) \coloneqq \frac{1}{N_m} \abs{\sum_{j \in C_m} e^{\im \phi_j}}
\end{equation}
across members $j$ of community $C_m$ with size $N_m$.
Then, we define a chimera-like index $\chi$
\begin{equation}
  \chi = \aab{
    \sigma_{\text{chi}}}_T
  \label{eq:chimera_index}
\end{equation}
where
\begin{equation*}
    \sigma_{\text{chi}} \coloneqq \frac{1}{M-1} \sum_{m=1}^M
    \pab{\rho_m(t) - \aab{\rho_m(t)}_M}^2
\end{equation*}
and a metastability index $\lambda$
\begin{equation}
  \lambda = \aab{
    \sigma_{\text{met}}}_M
  \label{eq:metastability_index}
\end{equation}
where
\begin{equation*}
    \sigma_{\text{met}} \coloneqq \frac{1}{T-1} \sum_{t=1}^T
    \pab{\rho_m(t) - \aab{\rho_m(t)}_T}^2
\end{equation*}
across the $M$ communities and $T$ time steps.
The chimera-like index measures the difference in coherence between communities:
complete homogeneity between communities
(\eg{} all fully synchronized \emph{or} fully disordered)
corresponds to $\chi = 0$,
while having $M$ communities
with half fully synchronized ($\rho_m = 1$)
and the other half fully disordered ($\rho_m = 0$)
for all times yields a maximum $\chi = M/[4(M-1)]
= 1/2$ for our $M=2$\approx~\citep{shanahan2010metastable}.
Likewise, the metastability index $\lambda$ measures how metastable
the system is (\ie{} transiting between synchronicity and disorder).
A system that is fully synchronized or disordered gives $\lambda = 0$;
$\lambda$ is maximized for a system spending equal times in each state,
where the variance of the uniform distribution gives
$\lambda = 1/12 \approx \num{0.08}$~\citep{shanahan2010metastable}.
%TC:endignore

\backmatter{}

\bmhead{Supplementary information}
Supplementary Figure 1 of two-player game order graphs.
Section 1, two-player game order graphs;
Section 2, derivation of well-mixed communicative fraction
with symmetry breaking;
Section 3, derivation of plurality game type criteria.
Supplementary videos 1-3,
time-evolution of \gls{celegans} player strategies
using the same color scheme as \cref{fig:graph}
with $B_0/c = 1.5$, $\beta_0/B_0 = 0.95$, $c = 0.1$,
$\mu = 0.0001$, $M = 20$, $\delta = 0.2$, and \num{8E6} time steps;
supplementary video 1 shows $\alpha = 0$,
supplementary video 2 shows $\alpha = 0.75$,
and
supplementary video 3 shows $\alpha = 1$.

%\bmhead{Acknowledgements}

\bmhead{Declarations}
The authors declare no competing interests.

\bmhead{Code availability}
An interactive webpage for exploring the data set is available at
\url{https://tzdyrski.github.io/egt-kuramoto/EKT_Plots.html}.
Additionally, all source code is available via GitHub
at \url{https://github.com/TZdyrski/egt-kuramoto/tree/1.0.0}.

\bmhead{Data availability}
The processed data for all simulated parameter ranges
is available on Zenodo at \url{https://doi.org/10.5281/zenodo.17135745}.

\bibliography{references.bib}

\end{document}

# vim: spelllang=en_gb
