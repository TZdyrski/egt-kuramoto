%! TeX program = lualatex
\documentclass[pdflatex,lineno,referee,sn-mathphys-ay]{class/sn-jnl}

% Import custom style file containing common packages and options
\usepackage{preamble}

\begin{document}

\title{In-Group and Out-Group Evolutionary Games}

\author{
\href{https://orcid.org/0000-0003-3039-172X}{Thomas Zdyrski}$^{1}$
and
\href{https://orcid.org/0000-0001-8252-1990}{Feng Fu}$^{1}$
}

\affil{$^{1}$Department of Mathematics, Dartmouth College, Hanover, NH 03755
}

%\keywords{xxxx, xxxx, xxxx}
%\pw{}

\abstract{
TODO: Abstract
}

\maketitle
\tableofcontents

\section{Introduction}
\subsection{In-group and out-group games}
\subsection{Multi-game evolution}
\subsection{Impact of structure on game evolution}
\subsection{\emph{C.\ Elegans} connectome}

\section{Methods and models}
\subsection{TODO}
\begin{itemize}
  \item Prove analytic prediction of communicative frequency vs $B(0)$
    (\ie{} orange line on cumulative plot)
  \item Re-run experiments while:
  \begin{itemize}
    \item Varying $B(0)$ and $\beta(0)$
    \item Using different graph structures
  \end{itemize}
  \item Figure out plots/metrics for myriad of data sets
  \begin{itemize}
    \item Correlations?
    \item Chimeric quality?
  \end{itemize}
  \item Figure out overarching story:
  \begin{itemize}
    \item Drop structured games and focus on symmetry breaking?
    \item Game-type dynamics
    \begin{itemize}
      \item Can we predict what game-type will be long-run result?
      \item Do evolutionary bottlenecks (like the chicken-game)
        necessarily result?
    \end{itemize}
    \item Simplifying reduction:
    \begin{itemize}
      \item Can $n_{\text{phase}} = 20$ case be qualitatively reproduced
        with just $n_{\text{phase}} = 2$ case (tuned to have same game
        type as long-run result)?
      \item Can this be understood as the interaction of two heterogeneous
        populations with identical inter-group games and (possibly
        distinct) identical intra-group games?
    \end{itemize}
  \end{itemize}
\end{itemize}
\subsection{Game setup}
\subsection{Birth-death Moran process}
\subsection{Statistics collected}
\subsection{Cooperative fraction: analytic}
Here, our fitness function $f_E$ of strategy $E$ depends on the total payoff (as opposed
to the average payoff) as
\begin{equation}
  f_E = \exp(\delta n_{\text{in}} \pi_E)
\end{equation}
where $\pi_E$ is the (in-degree averaged) \emph{mean} payoff (as used in
\cite{tripp2022evolutionary}) and $n_{\text{in}}$ is the in-degree of
the node.

\subsubsection{Structured}
Most of the derivation from \cite{tripp2022evolutionary} for the
fraction of communicative strategies carries through to the structured
graph case since it only relied on the form of the payoff matrix rather
than the adjacency matrix.
However, we do need to recalculate the ratio $\omega = \rho_{CN,\Delta
\phi_{qr}}/\rho_{NC,\Delta \phi_{qr}}$.

Unfortunately, we can't exactly follow the \cite{tripp2022evolutionary}
derivation of $\omega$ by constructing a Markov chain for the evolution
of two strategies parameterized by the state $i$ representing the number
of players with strategy $A$:
for a structured graph, the number of strategies $E$ is not sufficient
to determine the state.
Nevertheless, we can approximate $\omega$ by approximating the \emph{C.\
Elegans} connectome by a $k$-regular graph where $k$ is the average
in-degree of the connectome.

\textbf{FALSE: consider cycle of even length (which, I believe, is
  strongly regular): even if $i=n/2$, there is a large difference
between having two clusters of A and B vs every-other being A or B}
Then, we can construct the transition matrix $p$, where $p_{i,j}$
represents the probability of state $i$ transitioning to state $j$, as
\begin{gather*}
  p_{0,0} = 1, \\
  p_{n,n} = 1, \\
  p_{i,i-1} = \frac{i}{n} \frac{}
                              {i f_E(i) + (n-1) f_F(i)}
\end{gather*}

\textbf{TODO}

\subsubsection{Broken symmetry}
Similarly, the derivation for the communicative fraction
in section 4 of the appendix for \cite{tripp2022evolutionary} carries
through until equation 16.
Instead, with $n_{\text{in}} = n-1$, we have
\begin{equation}
  \begin{aligned}[b]
  \gamma_k &= \frac{p_{k,k-1}}{p_{k,k+1}} \\
           &= \frac{f_N(k)}{f_C(K)} \\
           &= \exp[\delta (n-1) (\pi_F(k) - \pi_E(k))]
\end{aligned}
\end{equation}
where strategy $F=(N,\phi_k)$, strategy $E=(C,\phi_j)$, and $k$ is the
number of communicative strategies $C$.
The new payoff functions $\pi_C$ and $\pi_N$ are given by
\begin{align}
  \pi_E(k) &= \ab(\frac{k-1}{n-1}) \ab(B(0) - c)
                + \ab(\frac{n-k}{n-1}) \ab(\beta(\Delta\phi) 2\alpha - c) \\
           &= \frac{1}{n-1}
               \ab(k \ab(B(\Delta\phi) - 2 \alpha \beta(\Delta\phi))
                 + 2 \alpha n \beta(\Delta\phi) - B(\Delta\phi) - (n-1) c)
\end{align}
and
\begin{align}
  \pi_F(k) &= \ab(\frac{k}{n-1}) \ab(\beta(\Delta\phi) 2 (1-\alpha))
                + \ab(\frac{n-k-1}{n-1}) \ab(0) \\
           &= \ab(\frac{k}{n-1}) 2 (1-\alpha) \beta(\Delta\phi)
\end{align}
where $\Delta \phi$ is the phase difference between the $E$ and $F$
strategies.
Note that $\pi_E$ has $B(0)$ instead of $B(\Delta \phi)$; this is
because the $B(\Delta \phi)$ payoff occurs when strategy $E$ plays
against strategy $E$, so $\Delta \phi = 0$ (recall we are only
considering a game with exactly two strategies, $E = (C, \phi_i)$ and
$F = (N, \phi_j)$.
Similar logic would apply to the $F$-$F$ strategy in $\pi_F$, but that
payoff is identically zero.

Thus, we have
\begin{equation}
  \begin{aligned}
    \gamma_k &= \exp\ab[\delta \ab(
    k 2 (1-\alpha) \beta(\Delta\phi)
    - k \ab(B(0) - 2 \alpha \beta(\Delta\phi))
                 - 2 \alpha n \beta(\Delta\phi) + B(0) + (n-1) c)] \\
      &= \exp\ab[\delta \ab(
    \ab(2 \beta(\Delta\phi) - B(0)) k
                 + B(0) - 2 \alpha n \beta(\Delta\phi) + (n-1) c)]
  \end{aligned}
\end{equation}
Next, we can calculate the ratio of fixation rates $\rho_F/\rho_E$ as
\begin{equation}
  \begin{aligned}
    \frac{\rho_F}{\rho_E} &= \prod_{k=1}^{n-1} \gamma_k \\
           &= \exp \ab\{
                \delta
                \sum_{k=1}^{n-1}
                \ab[
    \ab(2 \beta(\Delta\phi) - B(0)) k
                 + B(0) - 2 \alpha n \beta(\Delta\phi) + (n-1) c
                 ]
                 \} \\
           &= \exp \ab\{
                \delta
                \ab[
                \ab(2 \beta(\Delta\phi) - B(0)) \frac{n(n-1)}{2}
                 + \ab(B(0) - 2 \alpha n \beta(\Delta\phi) + (n-1) c) (n-1)
                 ]
                 \} \\
           &= \exp \ab\{
                \delta (n-1)
                \ab[
                (n-1) c + n \beta(\Delta\phi) (1 - 2 \alpha)
                - \frac{n-2}{2} B(0)
                 ]
                 \}
  \end{aligned}
\end{equation}
Notice, now the ratio of $\rho_F/\rho_E$ depends of $\Delta \phi$,
unlike in the $\alpha=1/2$ case.

Soon, we will require the expression for $\rho_{NC,\Delta \phi_{qr}}
\coloneqq \rho_E$.
Thus, we have
\begin{equation}
  \begin{aligned}
    \rho_E &= \frac{1}{1+\sum_{j=1}^{n-1} \prod_{k=1}^j \gamma_k} \\
           &= \ab(1+\sum_{j=1}^{n-1} \prod_{k=1}^j \exp\ab[\delta \ab(
           \ab(2 \beta(\Delta\phi_{qr}) - B(0)) k
           + B(0) - 2 \alpha n \beta(\Delta\phi_{qr}) + (n-1) c)])^{-1}
                 \\
           &= \ab(1+\sum_{j=1}^{n-1} \exp\ab[\delta \sum_{k=1}^j \ab(
           \ab(2 \beta(\Delta\phi_{qr}) - B(0)) k
           + B(0) - 2 \alpha n \beta(\Delta\phi_{qr}) + (n-1) c)])^{-1}
                 \\
           &= \ab(1+\sum_{j=1}^{n-1} \exp\ab[\delta \ab(
           \ab(2 \beta(\Delta\phi_{qr}) - B(0)) j(j+1)/2
           + j \ab(B(0) - 2 \alpha n \beta(\Delta\phi_{qr}) + (n-1) c))])^{-1}
                 \\
           &= \ab(1+\sum_{j=1}^{n-1} \exp\ab[\delta \ab(
           \ab(\beta(\Delta\phi_{qr}) - B(0)/2) j^2
           + j \ab(B(0)/2 + \beta(\Delta\phi_{qr}) (1 - 2 \alpha n)  + (n-1) c))])^{-1}
  \end{aligned}
\end{equation}

Then, for the multi-population, low-mutation case, we identify
$\rho_{NC,\Delta \phi_{qr}} = \rho_E$ and $\rho_{CN,\Delta \phi_{qr}} = \rho_F$
Then, we can use this to calculate the ratio of the stationary state
eigenvalues $s_1/s_2$
\begin{align*}
  \frac{s_1}{s_2} &= \frac{\sum_{r=1}^{d} \rho_{NC,\Delta \phi_{qr}} }
  {\sum_{r=1}^{d} \rho_{CN,\Delta \phi_{qr}}} \\
                  &= \frac{\sum_{r=1}^{d} \rho_{NC,\Delta \phi_{qr}}
                  }
  {\sum_{r=1}^{d} \rho_{NC,\Delta \phi_{qr}}
                \exp \ab\{
                \delta (n-1)
                \ab[
                (n-1) c + n \beta(\Delta\phi) (1 - 2 \alpha)
                - \frac{n-2}{2} B(0)
                 ]
                 \}
               }
               \\
      &=
      \ab[
\frac
{\sum_{r=1}^{d}
  \frac{1}{1+\sum_{j=1}^{n-1} \exp\ab[\delta \ab(
           \ab(\beta(\Delta\phi_{qr}) - B(0)/2) j^2
           + j \ab(B(0)/2 + \beta(\Delta\phi_{qr}) (1 - 2 \alpha n)  + (n-1) c))]}
                  }
{\sum_{r=1}^{d}
  \frac{\exp \ab\{
                \delta (n-1)
                \ab[
                (n-1) c + n \beta(\Delta\phi) (1 - 2 \alpha)
                - \frac{n-2}{2} B(0)
                 ]
                 \}
}{1+\sum_{j=1}^{n-1} \exp\ab[\delta \ab(
           \ab(\beta(\Delta\phi_{qr}) - B(0)/2) j^2
           + j \ab(B(0)/2 + \beta(\Delta\phi_{qr}) (1 - 2 \alpha n)  +
         (n-1) c))]}
                               }
                               ]
\end{align*}
Again, unlike in the $\alpha = 1/2$ symmetric case, we cannot factor the
exponential component out of the sum and cancel the $\rho_{CN,\Delta
\phi_{qr}}$ terms.
Also, note that \cite{tripp2022evolutionary} has a typo by placing the
exponential in the numerator.

We cannot simplify the sum as $\sum_j \exp(j^2)$ doesn't have a closed form solution.
However, we can approximate it.
First, we define $\nu \coloneqq s_2/s_1$ so that $\nu = \omega$ for the
symmetric $\alpha = 1/2$ case.
Then, using the fact that $d s_1 + d s_2 = 1$, we find the probability
of communicative fixation $d s_1$ to be
\begin{equation}
  \begin{aligned}
    \frac{s_2}{s_1} &= \frac{1 - d s_1}{d s_1} = \nu \\
                    &\iff d s_1 = \frac{1}{1+\nu} =
                    \frac{1}{1+e^{\ln(\nu)}}
  \end{aligned}
\end{equation}
Additionally, we will define $\Phi(B(0);\beta(\Delta\Phi),c,\alpha,n)
\coloneqq \ln(\nu)$.
For the symmetric case $\alpha=1/2$, this exactly simplifies to $\Phi =
\delta(n-1)[(n-1)c - (n-2)B(0)/2]$, while the closed-form is intractable
for the $\alpha \neq 1/2$ case.

Numerical simulations suggest that $s_1$ approximately has the form of
$1/(1+\exp(m_0 + m_1 B(0)))$ as a function of $B(0)$ even when $\alpha
\neq 1/2$.
Therefore, we will attempt to Taylor expand $\Phi(B(0))$ for small
$B(0) \ll 1$; since all of our simulations use $\beta \propto B$, we also
assume $\beta \ll 1$.
To zeroth order in $B(0)$ and $\beta$, we have
\begin{equation}
  \begin{aligned}
    \Phi^{(0)} &= \ln\ab(
\frac{\sum_{r=1}^{d}
\ab(1+\sum_{j=1}^{n-1} \exp\ab[\delta \ab( j (n-1) c)])^{-1}
                \exp \ab\{ \delta (n-1) (n-1) c \}
               }
{\sum_{r=1}^{d}
\ab(1+\sum_{j=1}^{n-1} \exp\ab[\delta \ab( j \ab((n-1) c))])^{-1} }
  ) \\
  &= \ln\ab( \exp\{ \delta (n-1)^2 c \}) \\
  &= \delta (n-1)^2 c
  \end{aligned}
\end{equation}
To first order in $B(0)$ and $\beta$, we have
\begin{equation}
  \begin{aligned}
    \Phi^{(0)+(1)} &= \ln\ab(
  e^{\delta(n-1)^2 c - \frac{n-2}{2} B(0)}
\frac{\sum_{r=1}^{d}
  \frac{1 + \delta (n-1) n \beta(\Delta\phi) (1 - 2 \alpha)
    }{1+\sum_{j=1}^{n-1} \delta \ab(
           \ab(\beta(\Delta\phi_{qr}) - B(0)/2) j^2
           + j \ab(B(0)/2 + \beta(\Delta\phi_{qr}) (1 - 2 \alpha n)))
           \exp\ab[\delta j (n-1) c]
         }
                               }
{\sum_{r=1}^{d}
  \frac{1}{1+\sum_{j=1}^{n-1} \delta \ab(
           \ab(\beta(\Delta\phi_{qr}) - B(0)/2) j^2
           + j \ab(B(0)/2 + \beta(\Delta\phi_{qr}) (1 - 2 \alpha n)))
           \exp\ab[\delta (n-1) c]
         }
                  }
  )
  \\
  &=
\delta(n-1)^2 c - \frac{n-2}{2} B(0) \\
  &\qquad +
\ln\ab(
\frac{\sum_{r=1}^{d}
  \frac{1 + \delta (n-1) n \beta(\Delta\phi) (1 - 2 \alpha)
    }{1+\sum_{j=1}^{n-1} \delta \ab(
           \ab(\beta(\Delta\phi_{qr}) - B(0)/2) j^2
           + j \ab(B(0)/2 + \beta(\Delta\phi_{qr}) (1 - 2 \alpha n)))
           \exp\ab[\delta j (n-1) c]
         }
                               }
{\sum_{r=1}^{d}
  \frac{1}{1+\sum_{j=1}^{n-1} \delta \ab(
           \ab(\beta(\Delta\phi_{qr}) - B(0)/2) j^2
           + j \ab(B(0)/2 + \beta(\Delta\phi_{qr}) (1 - 2 \alpha n)))
           \exp\ab[\delta j (n-1) c]
         }
                  }
  )
  \end{aligned}
  \label{eq:phase_first-order_big-frac}
\end{equation}
As a short detour, we will evaluate the $j$ sum separately:
\begin{equation}
  \begin{aligned}
    &
\sum_{j=1}^{n-1} \delta \ab(
           \ab(\beta(\Delta\phi_{qr}) - B(0)/2) j^2
           + j \ab(B(0)/2 + \beta(\Delta\phi_{qr}) (1 - 2 \alpha n)))
           \exp\ab[\delta j (n-1) c]
    \\
    &=
    \delta \ab(\beta(\Delta\phi_{qr}) - B(0)/2) \sum_{j=1}^{n-1}
           j^2 \delta \exp\ab[\delta j (n-1) c]
    + \delta \ab(B(0)/2 + \beta(\Delta\phi_{qr}) (1 - 2 \alpha n)) \sum_{j=1}^{n-1}
           j \delta \exp\ab[\delta j (n-1) c]
    \\
    &=
    \delta \ab(\beta(\Delta\phi_{qr}) - B(0)/2)
    e^{\delta(n-1)c} \frac{\ab(1 + e^{\delta(n-1)c} + n^2 e^{\delta(n-1)^2c} + 2n(n-1)
    e^{\delta n(n-1)c} - (n-1)^2 e^{\delta(n^2-1)c})}{\ab(1-e^{\delta(n-1)c})^3}
           \\
    &\qquad
    + \delta \ab(B(0)/2 + \beta(\Delta\phi_{qr}) (1 - 2 \alpha n))
    e^{\delta (n-1) c} \frac{\ab(1 - n e^{\delta (n-1)^2 c}
    + (n-1) e^{\delta n (n-1) c})}{\ab(1-e^{\delta (n-1) c})^2}
    \\
    &=
    \frac{\delta e^{\delta(n-1)c}}{\ab(1-e^{\delta (n-1)c})^3}
\biggl[
  \\
    &\qquad
\ab(\beta(\Delta\phi_{qr}) - B(0)/2)
\ab(1 + e^{\delta(n-1)c} + n^2 e^{\delta(n-1)^2c} + 2n(n-1)
    e^{\delta n(n-1)c} - (n-1)^2 e^{\delta(n^2-1)c})
    \\
    &\qquad
+
\ab(B(0)/2 + \beta(\Delta\phi_{qr}) (1 - 2 \alpha n))
\ab(1 - n e^{\delta (n-1)^2 c}
    + (n-1) e^{\delta n (n-1) c})
\ab(1-e^{\delta(n-1)c})
\biggr]
    \\
    &=
    \frac{\delta e^{\delta(n-1)c}}{\ab(1-e^{\delta (n-1)c})^3}
\biggl[
  \\
    &\qquad
\ab(\beta(\Delta\phi_{qr}) - B(0)/2)
\ab(1 + e^{\delta(n-1)c} + n^2 e^{\delta(n-1)^2c} + 2n(n-1)
    e^{\delta n(n-1)c} - (n-1)^2 e^{\delta(n^2-1)c})
    \\
    &\qquad
+
\ab(B(0)/2 + \beta(\Delta\phi_{qr}) (1 - 2 \alpha n))
\Bigl(1 - n e^{\delta (n-1)^2 c} + (2n-1) e^{\delta n (n-1) c}
    \\
    &\qquad\qquad
    - e^{\delta(n-1)c} - (n-1) e^{\delta (n^2-1) c}
    \Bigr)
\biggr]
    \\
    &=
    \frac{\delta e^{\delta(n-1)c}}{\ab(1-e^{\delta (n-1)c})^3}
\biggl[
  \\
    &\qquad
B(0)
\ab( - e^{\delta(n-1)c} - \frac{n (n+1)}{2} e^{\delta(n-1)^2c}
- \frac{(2n-1)^2}{2} e^{\delta n(n-1)c} + \frac{(n-1)(n-2)}{2} e^{\delta(n^2-1)c})
    \\
    &\qquad
    +
\beta(\Delta\phi_{qr})
\Bigl(2(1-\alpha n) + 2 \alpha n e^{\delta(n-1)c}
    + n(n -1 +2\alpha) e^{\delta(n-1)^2c}
    \\
    &\qquad\qquad
    + \ab[4n^2-1 -2\alpha n (2n-1)] e^{\delta n(n-1)c}
    - \ab(2\alpha-1)n(n-1) e^{\delta(n^2-1)c}
    - n (1-2\alpha n) e^{\delta (n-1)^2 c}
\Bigr)
\biggr]
  \\
    &\coloneqq
    \frac{\delta e^{\delta(n-1)c}}{\ab(1-e^{\delta (n-1)c})^3}
    \ab[
  M B(0) + N \beta(\Delta \phi_{qr})]
  \end{aligned}
\end{equation}
Additionally, we can approximate
\begin{equation}
  \begin{aligned}
  &\frac{1}{1+\sum_j \ldots}
  = 1 -
    \frac{\delta e^{\delta(n-1)c}}{\ab(1-e^{\delta (n-1)c})^3}
    \ab[
  M B(0) + N \beta(\Delta \phi_{qr})]
  \end{aligned}
\end{equation}

Substituting this result into \cref{eq:phase_first-order_big-frac} gives
\begin{equation}
  \begin{aligned}
    \Phi^{(0)+(1)} &=
\delta(n-1)^2 c - \frac{n-2}{2} B(0) \\
  &\qquad +
\ln\ab(
\frac{\sum_{r=1}^{d}
  \ab[
  1 + \delta (n-1) n \beta(\Delta\phi_{qr}) (1 - 2 \alpha)]
  \ab\{
1 -
    \frac{\delta e^{\delta(n-1)c}}{\ab(1-e^{\delta (n-1)c})^3}
    \ab[
  M B(0) + N \beta(\Delta \phi_{qr})]
\}
                               }
{\sum_{r=1}^{d}
  \ab\{
1 -
    \frac{\delta e^{\delta(n-1)c}}{\ab(1-e^{\delta (n-1)c})^3}
    \ab[
  M B(0) + N \beta(\Delta \phi_{qr})]
\}
                  }
  )
  \\
  &=
\delta(n-1)^2 c - \frac{n-2}{2} B(0) \\
  &\qquad +
\ln\ab(
\frac{
  \begin{aligned}
&d\ab(1 -
    \frac{\delta e^{\delta(n-1)c}}{\ab(1-e^{\delta (n-1)c})^3}
  M B(0))
  \\
&\qquad-
  \delta (n-1) n (1 - 2 \alpha)
  \frac{\delta e^{\delta(n-1)c}}{\ab(1-e^{\delta (n-1)c})^3} N
  \sum_{r=1}^{d}
  \beta^2(\Delta\phi_{qr})
  \\
&\qquad+
  \ab[
  \delta (n-1) n (1 - 2 \alpha) \ab( 1 -
    \frac{\delta e^{\delta(n-1)c}}{\ab(1-e^{\delta (n-1)c})^3}
  M B(0))
  - \frac{\delta e^{\delta(n-1)c}}{\ab(1-e^{\delta (n-1)c})^3} N
  ]
  \sum_{r=1}^{d}
  \beta(\Delta \phi_{qr})
\end{aligned}
}
{d\ab(1 -
    \frac{\delta e^{\delta(n-1)c}}{\ab(1-e^{\delta (n-1)c})^3}
  M B(0))
  - N \frac{\delta e^{\delta(n-1)c}}{\ab(1-e^{\delta (n-1)c})^3}
  \sum_{r=1}^{d} \beta(\Delta \phi_{qr})
                  }
  )
\end{aligned}
\end{equation}
Dropping terms of order $\beta_0^2$ gives
\begin{equation}
  \begin{aligned}
    \Phi^{(0)+(1)} &=
\delta(n-1)^2 c - \frac{n-2}{2} B(0) \\
  &\qquad +
\ln\ab(
\frac{
  \begin{aligned}
&d\ab(1 -
    \frac{\delta e^{\delta(n-1)c}}{\ab(1-e^{\delta (n-1)c})^3}
  M B(0))
  \\
&\qquad+
  \ab[
-
    \frac{\delta e^{\delta(n-1)c}}{\ab(1-e^{\delta (n-1)c})^3} N
    +
\delta (n-1) n (1 - 2 \alpha)
]
  \sum_{r=1}^{d}
  \beta(\Delta \phi_{qr})
\end{aligned}
}
{d\ab(1 -
    \frac{\delta e^{\delta(n-1)c}}{\ab(1-e^{\delta (n-1)c})^3}
  M B(0))
  - N \frac{\delta e^{\delta(n-1)c}}{\ab(1-e^{\delta (n-1)c})^3}
  \sum_{r=1}^{d} \beta(\Delta \phi_{qr})
                  }
  )
\end{aligned}
\end{equation}

Note the following values:
\textbf{TODO: Prove $\sum_1^d \cos = 0$}
\begin{equation}
  \sum_{r=1}^d \beta(\Delta \phi) = \frac{d}{2} \beta_0
\end{equation}

Thus, we have
\begin{equation}
  \begin{aligned}
    \Phi^{(0)+(1)} &=
\delta(n-1)^2 c - \frac{n-2}{2} B(0) \\
  &\qquad +
\ln\ab(
\frac{
1 -
    \frac{\delta e^{\delta(n-1)c}}{\ab(1-e^{\delta (n-1)c})^3}
  M B(0)
+
\frac{1}{2}\beta_0\ab[
-
    \frac{\delta e^{\delta(n-1)c}}{\ab(1-e^{\delta (n-1)c})^3} N
    +
\delta (n-1) n (1 - 2 \alpha)
]
}
{1 -
    \frac{\delta e^{\delta(n-1)c}}{\ab(1-e^{\delta (n-1)c})^3}
  M B(0)
  - N \beta_0 \frac{\delta e^{\delta(n-1)c}}{2\ab(1-e^{\delta (n-1)c})^3}
                  }
  )
  \\
  &\approx
\delta(n-1)^2 c - \frac{n-2}{2} B(0) \\
  &\qquad +
\ln \biggl(
  1 + \frac{\beta_0}{2} \delta n (n-1) (1-2\alpha)
\biggr)
  \\
  &\approx
  \delta(n-1)^2 c - \frac{n-2}{2} B(0) +
  \frac{\beta_0}{2} \delta n (n-1) (1-2\alpha)
\end{aligned}
\end{equation}
\textbf{TODO: We should be able to significantly shorten this
  derivation: it turns out, to first order, the $\rho_{NC}$
  numerator/denominator terms cancel and only the $\omega$ term remains}
\textbf{TODO: Can we analytically calculate where $\nu = 1$ (\ie{} where
  the inflection point is) and then expand about there to get a more
  accurate result than expanding about $B(0) = \beta_0 = 0$ (and hence,
  bad approximation for $\alpha \to 0$ which pushes $\nu=1$ to large
  $B(0)$}

\section{Results}
\subsection{Complete graphs}
\subsection{Structured graphs}
\begin{itemize}
  \item For the strong-selection case ($B_0 = 2.5c$ and $\delta = 5$
    with $t_{\text{step}} = \num{2e4}$):
  \begin{itemize}
    \item The weightedness has virtually no effect
    \item The directionality generally enhances cooperation
      and suppresses the order parameter to approximately
      \num{0.5} each
    \begin{itemize}
      \item The one exception case is $\alpha = \num{0.25}$ where a
        chicken-type game with relatively high order parameter
        (\num{0.7}) and low cooperation (\num{0.3}) change to a
        deadlock-type game with medium order parameter and cooperation
        (approximately \num{0.5} each)
    \end{itemize}
  \end{itemize}
\item For the weak-selection case ($B_0 = 1.5c$ and $\delta = 0.2$) with
  $t_{\text{step}} = \num{2e4}$), things change dramatically
\end{itemize}
\subsubsection{Random graphs}
\subsubsection{\emph{C.\ Elegans}}

\begin{itemize}
  \item Weighted (connectome) vs.\ unweighted (connectome w/o
    weights): weighting reduces synchronization
  \item Connectome is connected, weighted, directed cyclic graph
  \begin{itemize}
    \item Therefore, \emph{cannot} directly use
      \cite{allen2017evolutionary}, since it only applies to undirected
      graphs
  \end{itemize}
  \item Low selection strength ($0.005$) keeps mostly around synchronicity
    $\approx 1.0$, with a small break to 0.8 \emph{before} $B_{\text{crit}}$
  \item High selection strength ($0.2$) has low synchronicity
    ($\approx 0.0$) before $B_{\text{crit}}$ but increases (at roughly
    $B_{\text{crit}}$) to $\approx 0.5$
  \item Hence, it seems that
  \begin{enumerate}
    \item Neither low nor high selection strength corresponds to
      well-mixed case
    \item Low and high selection strengths do not appear to be simple
      stretching about $B_{\text{crit}}$ as in the well-mixed case
  \end{enumerate}
  \item We can calculate the average number of communicative at each
    time step and calculate the Pearson correlation coefficient of
    this number against each node's communicative state:
  \begin{enumerate}
    \item Results for c-elegans, $B_{\text{factor}} = 1.5$, $2E6$ timesteps,
      $\delta = 2$, $\mu = 0.2$
    \item We find that the maximum correlation is \num{0.41}, mean
      correlation is \num{0.03}, std correlation is \num{0.06}. That
      is, there are a few outliers with high correlation
    \item Unfortunately, these are \emph{not} the premotor neurons
      (all had correlation $\approx \num{0.1}$)
    \item Instead, the ones with correlation above \num{0.35} were
      indices 101, 117, 211, and 219 (IL1L, AUAL, SMBDL, and SIADL)
  \end{enumerate}
\end{itemize}

\section{Discussion}
\subsection{Game-type dynamics}
\subsubsection{Markov chain}
\subsubsection{Attractors}

\subsection{Forcing strength dependence}
\begin{itemize}
  \item For strong selection $\delta = 0.2$:
  \begin{itemize}
    \item Undirected and unweighted looks like a spread-out, symmetric
      version of well-mixed
    \item Adding weights suppresses communication, lowering ceiling
      for fraction of communication from \num{1} to \num{0.8}
    \item Adding directedness introduces strange staircase-like shape
      and slightly depresses ceiling from \num{1} to \num{0.9}
    \item Full weighted, directed version shows both staircase-like
      shape and strongly suppressed ceiling to \num{0.6}
  \end{itemize}
  \item For weak selection $\delta = 0.005$:
  \begin{itemize}
    \item Undirected and unweighted has strange staircase-like shape
    \item Adding weights sharpens jump around $B_\text{crit}$, nearly
      matching well-mixed case
    \item Adding directedness strongly suppresses selection, with
      ceiling and floor compressed to the range $[0.3,0.6]$
    \item Full weighted, directed version shows strange, non-monotonic
      behavior. More investigation is needed
  \end{itemize}
\end{itemize}

\subsection{Structured graph effects}
\subsubsection{Random graphs}
\subsubsection{\emph{C.\ Elegans}}

\section{Conclusion}
\subsection{Key results}
\subsection{Next steps}

\subsubsection{Acknowledgements}

\backmatter

\bibliography{references.bib}

\end{document}

# vim: spelllang=en_gb
