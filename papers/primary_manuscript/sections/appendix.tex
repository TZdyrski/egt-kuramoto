\section{Broken symmetry}
Similarly, the derivation for the communicative fraction
in section 4 of the appendix for \cite{tripp2022evolutionary} carries
through until equation 16.
Instead, with $n_{\text{in}} = n-1$, we have
\begin{equation}
  \begin{aligned}[b]
  \gamma_k &= \frac{p_{k,k-1}}{p_{k,k+1}} \\
           &= \frac{f_N(k)}{f_C(K)} \\
           &= \exp[\delta (n-1) (\pi_F(k) - \pi_E(k))]
\end{aligned}
\end{equation}
where strategy $F=(N,\phi_k)$, strategy $E=(C,\phi_j)$, and $k$ is the
number of communicative strategies $C$.
The new (edge-averaged) payoff functions $\pi_C$ and $\pi_N$ are given by
\begin{align}
  \pi_E(k) &= \ab(\frac{k-1}{n-1}) \ab(B(0) - c)
                + \ab(\frac{n-k}{n-1}) \ab(\beta(\Delta\phi) 2\alpha - c) \\
           &= \frac{1}{n-1}
               \ab(k \ab(B(\Delta\phi) - 2 \alpha \beta(\Delta\phi))
                 + 2 \alpha n \beta(\Delta\phi) - B(\Delta\phi) - (n-1) c)
\end{align}
and
\begin{align}
  \pi_F(k) &= \ab(\frac{k}{n-1}) \ab(\beta(\Delta\phi) 2 (1-\alpha))
                + \ab(\frac{n-k-1}{n-1}) \ab(0) \\
           &= \ab(\frac{k}{n-1}) 2 (1-\alpha) \beta(\Delta\phi)
\end{align}
where $\Delta \phi$ is the phase difference between the $E$ and $F$
strategies.
Note that $\pi_E$ has $B(0)$ instead of $B(\Delta \phi)$; this is
because the $B(\Delta \phi)$ payoff occurs when strategy $E$ plays
against strategy $E$, so $\Delta \phi = 0$ (recall we are only
considering a game with exactly two strategies, $E = (C, \phi_i)$ and
$F = (N, \phi_j)$.
Similar logic would apply to the $F$-$F$ strategy in $\pi_F$, but that
payoff is identically zero.

Thus, we have
\begin{equation}
  \begin{aligned}
    \gamma_k &= \exp\ab[\delta \ab(
    k 2 (1-\alpha) \beta(\Delta\phi)
    - k \ab(B(0) - 2 \alpha \beta(\Delta\phi))
                 - 2 \alpha n \beta(\Delta\phi) + B(0) + (n-1) c)] \\
      &= \exp\ab[\delta \ab(
    \ab(2 \beta(\Delta\phi) - B(0)) k
                 + B(0) - 2 \alpha n \beta(\Delta\phi) + (n-1) c)]
  \end{aligned}
\end{equation}
Next, we can calculate the ratio of fixation rates $\rho_F/\rho_E$ as
\begin{equation}
  \begin{aligned}
    \frac{\rho_F}{\rho_E} &= \prod_{k=1}^{n-1} \gamma_k \\
           &= \exp \ab\{
                \delta
                \sum_{k=1}^{n-1}
                \ab[
    \ab(2 \beta(\Delta\phi) - B(0)) k
                 + B(0) - 2 \alpha n \beta(\Delta\phi) + (n-1) c
                 ]
                 \} \\
           &= \exp \ab\{
                \delta
                \ab[
                \ab(2 \beta(\Delta\phi) - B(0)) \frac{n(n-1)}{2}
                 + \ab(B(0) - 2 \alpha n \beta(\Delta\phi) + (n-1) c) (n-1)
                 ]
                 \} \\
           &= \exp \ab\{
                \delta (n-1)
                \ab[
                (n-1) c + n \beta(\Delta\phi) (1 - 2 \alpha)
                - \frac{n-2}{2} B(0)
                 ]
                 \}
  \end{aligned}
\end{equation}
Notice, now the ratio of $\rho_F/\rho_E$ depends of $\Delta \phi$,
unlike in the $\alpha=1/2$ case.

Soon, we will require the expression for $\rho_{NC,\Delta \phi_{qr}}
\coloneqq \rho_E$.
Thus, we have
\begin{equation}
  \begin{aligned}
    \rho_E &= \frac{1}{1+\sum_{j=1}^{n-1} \prod_{k=1}^j \gamma_k} \\
           &= \ab(1+\sum_{j=1}^{n-1} \prod_{k=1}^j \exp\ab[\delta \ab(
           \ab(2 \beta(\Delta\phi_{qr}) - B(0)) k
           + B(0) - 2 \alpha n \beta(\Delta\phi_{qr}) + (n-1) c)])^{-1}
                 \\
           &= \ab(1+\sum_{j=1}^{n-1} \exp\ab[\delta \sum_{k=1}^j \ab(
           \ab(2 \beta(\Delta\phi_{qr}) - B(0)) k
           + B(0) - 2 \alpha n \beta(\Delta\phi_{qr}) + (n-1) c)])^{-1}
                 \\
           &= \ab(1+\sum_{j=1}^{n-1} \exp\ab[\delta \ab(
           \ab(2 \beta(\Delta\phi_{qr}) - B(0)) j(j+1)/2
           + j \ab(B(0) - 2 \alpha n \beta(\Delta\phi_{qr}) + (n-1) c))])^{-1}
                 \\
           &= \ab(1+\sum_{j=1}^{n-1} \exp\ab[\delta \ab(
           \ab(\beta(\Delta\phi_{qr}) - B(0)/2) j^2
           + j \ab(B(0)/2 + \beta(\Delta\phi_{qr}) (1 - 2 \alpha n)  + (n-1) c))])^{-1}
  \end{aligned}
\end{equation}

Then, for the multi-population, low-mutation case, we identify
$\rho_{NC,\Delta \phi_{qr}} = \rho_E$ and $\rho_{CN,\Delta \phi_{qr}} = \rho_F$
Then, we can use this to calculate the ratio of the stationary state
eigenvalues $s_1/s_2$
\begin{align}
  \frac{s_1}{s_2} &= \frac{\sum_{r=1}^{d} \rho_{NC,\Delta \phi_{qr}} }
  {\sum_{r=1}^{d} \rho_{CN,\Delta \phi_{qr}}} \\
                  &= \frac{\sum_{r=1}^{d} \rho_{NC,\Delta \phi_{qr}}
                  }
  {\sum_{r=1}^{d} \rho_{NC,\Delta \phi_{qr}}
                \exp \ab\{
                \delta (n-1)
                \ab[
                (n-1) c + n \beta(\Delta\phi) (1 - 2 \alpha)
                - \frac{n-2}{2} B(0)
                 ]
                 \}
               }
               \\
      &=
      \ab[
\frac
{\sum_{r=1}^{d}
  \frac{1}{1+\sum_{j=1}^{n-1} \exp\ab[\delta \ab(
           \ab(\beta(\Delta\phi_{qr}) - B(0)/2) j^2
           + j \ab(B(0)/2 + \beta(\Delta\phi_{qr}) (1 - 2 \alpha n)  + (n-1) c))]}
                  }
{\sum_{r=1}^{d}
  \frac{\exp \ab\{
                \delta (n-1)
                \ab[
                (n-1) c + n \beta(\Delta\phi) (1 - 2 \alpha)
                - \frac{n-2}{2} B(0)
                 ]
                 \}
}{1+\sum_{j=1}^{n-1} \exp\ab[\delta \ab(
           \ab(\beta(\Delta\phi_{qr}) - B(0)/2) j^2
           + j \ab(B(0)/2 + \beta(\Delta\phi_{qr}) (1 - 2 \alpha n)  +
         (n-1) c))]}
                               }
                               ]
  \label{eq:full_analytic}
\end{align}
Again, unlike in the $\alpha = 1/2$ symmetric case, we cannot factor the
exponential component out of the sum and cancel the $\rho_{CN,\Delta
\phi_{qr}}$ terms.
Also, note that \cite{tripp2022evolutionary} has a typo by placing the
exponential in the numerator.

We cannot simplify the sum as $\sum_j \exp(j^2)$ doesn't have a closed form solution.
However, we can approximate it.
First, we define $\nu \coloneqq s_2/s_1$ so that $\nu = \omega$ for the
symmetric $\alpha = 1/2$ case.
Then, using the fact that $d s_1 + d s_2 = 1$, we find the probability
of communicative fixation $d s_1$ to be
\begin{equation}
  \begin{aligned}
    \frac{s_2}{s_1} &= \frac{1 - d s_1}{d s_1} = \nu \\
                    &\iff d s_1 = \frac{1}{1+\nu} =
                    \frac{1}{1+e^{\ln(\nu)}}
  \end{aligned}
\end{equation}
Additionally, we will define $\Phi(B(0);\beta(\Delta\Phi),c,\alpha,n)
\coloneqq \ln(\nu)$.
For the symmetric case $\alpha=1/2$, this exactly simplifies to $\Phi =
\delta(n-1)[(n-1)c - (n-2)B(0)/2]$, while the closed-form is intractable
for the $\alpha \neq 1/2$ case.

Numerical simulations suggest that $s_1$ approximately has the form of
$1/(1+\exp(m_0 + m_1 B(0)))$ as a function of $B(0)$ even when $\alpha
\neq 1/2$.
Therefore, we will attempt to Taylor expand $\Phi(B(0))$ for small
$B(0) \ll 1$; since all of our simulations use $\beta \propto B$, we also
assume $\beta \ll 1$.
To zeroth order in $B(0)$ and $\beta$, we have
\begin{equation}
  \begin{aligned}
    \Phi^{(0)} &= \ln\ab(
\frac{\sum_{r=1}^{d}
\ab(1+\sum_{j=1}^{n-1} \exp\ab[\delta \ab( j (n-1) c)])^{-1}
                \exp \ab\{ \delta (n-1) (n-1) c \}
               }
{\sum_{r=1}^{d}
\ab(1+\sum_{j=1}^{n-1} \exp\ab[\delta \ab( j \ab((n-1) c))])^{-1} }
  ) \\
  &= \ln\ab( \exp\{ \delta (n-1)^2 c \}) \\
  &= \delta (n-1)^2 c
  \end{aligned}
\end{equation}
To first order in $B(0)$ and $\beta$, we have
\begin{equation}
  \begin{aligned}
    \Phi^{(0)+(1)} &= \ln\ab(
  e^{\delta(n-1)^2 c - \frac{n-2}{2} B(0)}
\frac{\sum_{r=1}^{d}
  \frac{1 + \delta (n-1) n \beta(\Delta\phi) (1 - 2 \alpha)
    }{1+\sum_{j=1}^{n-1} \delta \ab(
           \ab(\beta(\Delta\phi_{qr}) - B(0)/2) j^2
           + j \ab(B(0)/2 + \beta(\Delta\phi_{qr}) (1 - 2 \alpha n)))
           \exp\ab[\delta j (n-1) c]
         }
                               }
{\sum_{r=1}^{d}
  \frac{1}{1+\sum_{j=1}^{n-1} \delta \ab(
           \ab(\beta(\Delta\phi_{qr}) - B(0)/2) j^2
           + j \ab(B(0)/2 + \beta(\Delta\phi_{qr}) (1 - 2 \alpha n)))
           \exp\ab[\delta (n-1) c]
         }
                  }
  )
  \\
  &=
\delta(n-1)^2 c - \frac{n-2}{2} B(0) \\
  &\qquad +
\ln\ab(
\frac{\sum_{r=1}^{d}
  \frac{1 + \delta (n-1) n \beta(\Delta\phi) (1 - 2 \alpha)
    }{1+\sum_{j=1}^{n-1} \delta \ab(
           \ab(\beta(\Delta\phi_{qr}) - B(0)/2) j^2
           + j \ab(B(0)/2 + \beta(\Delta\phi_{qr}) (1 - 2 \alpha n)))
           \exp\ab[\delta j (n-1) c]
         }
                               }
{\sum_{r=1}^{d}
  \frac{1}{1+\sum_{j=1}^{n-1} \delta \ab(
           \ab(\beta(\Delta\phi_{qr}) - B(0)/2) j^2
           + j \ab(B(0)/2 + \beta(\Delta\phi_{qr}) (1 - 2 \alpha n)))
           \exp\ab[\delta j (n-1) c]
         }
                  }
  )
  \end{aligned}
  \label{eq:phase_first-order_big-frac}
\end{equation}
As a short detour, we will evaluate the $j$ sum separately:
\begin{equation}
  \begin{aligned}
    &
\sum_{j=1}^{n-1} \delta \ab(
           \ab(\beta(\Delta\phi_{qr}) - B(0)/2) j^2
           + j \ab(B(0)/2 + \beta(\Delta\phi_{qr}) (1 - 2 \alpha n)))
           \exp\ab[\delta j (n-1) c]
    \\
    &=
    \delta \ab(\beta(\Delta\phi_{qr}) - B(0)/2) \sum_{j=1}^{n-1}
           j^2 \delta \exp\ab[\delta j (n-1) c]
    + \delta \ab(B(0)/2 + \beta(\Delta\phi_{qr}) (1 - 2 \alpha n)) \sum_{j=1}^{n-1}
           j \delta \exp\ab[\delta j (n-1) c]
    \\
    &=
    \delta \ab(\beta(\Delta\phi_{qr}) - B(0)/2)
    e^{\delta(n-1)c} \frac{\ab(1 + e^{\delta(n-1)c} + n^2 e^{\delta(n-1)^2c} + 2n(n-1)
    e^{\delta n(n-1)c} - (n-1)^2 e^{\delta(n^2-1)c})}{\ab(1-e^{\delta(n-1)c})^3}
           \\
    &\qquad
    + \delta \ab(B(0)/2 + \beta(\Delta\phi_{qr}) (1 - 2 \alpha n))
    e^{\delta (n-1) c} \frac{\ab(1 - n e^{\delta (n-1)^2 c}
    + (n-1) e^{\delta n (n-1) c})}{\ab(1-e^{\delta (n-1) c})^2}
    \\
    &=
    \frac{\delta e^{\delta(n-1)c}}{\ab(1-e^{\delta (n-1)c})^3}
\biggl[
  \\
    &\qquad
\ab(\beta(\Delta\phi_{qr}) - B(0)/2)
\ab(1 + e^{\delta(n-1)c} + n^2 e^{\delta(n-1)^2c} + 2n(n-1)
    e^{\delta n(n-1)c} - (n-1)^2 e^{\delta(n^2-1)c})
    \\
    &\qquad
+
\ab(B(0)/2 + \beta(\Delta\phi_{qr}) (1 - 2 \alpha n))
\ab(1 - n e^{\delta (n-1)^2 c}
    + (n-1) e^{\delta n (n-1) c})
\ab(1-e^{\delta(n-1)c})
\biggr]
    \\
    &=
    \frac{\delta e^{\delta(n-1)c}}{\ab(1-e^{\delta (n-1)c})^3}
\biggl[
  \\
    &\qquad
\ab(\beta(\Delta\phi_{qr}) - B(0)/2)
\ab(1 + e^{\delta(n-1)c} + n^2 e^{\delta(n-1)^2c} + 2n(n-1)
    e^{\delta n(n-1)c} - (n-1)^2 e^{\delta(n^2-1)c})
    \\
    &\qquad
+
\ab(B(0)/2 + \beta(\Delta\phi_{qr}) (1 - 2 \alpha n))
\Bigl(1 - n e^{\delta (n-1)^2 c} + (2n-1) e^{\delta n (n-1) c}
    \\
    &\qquad\qquad
    - e^{\delta(n-1)c} - (n-1) e^{\delta (n^2-1) c}
    \Bigr)
\biggr]
    \\
    &=
    \frac{\delta e^{\delta(n-1)c}}{\ab(1-e^{\delta (n-1)c})^3}
\biggl[
  \\
    &\qquad
B(0)
\ab( - e^{\delta(n-1)c} - \frac{n (n+1)}{2} e^{\delta(n-1)^2c}
- \frac{(2n-1)^2}{2} e^{\delta n(n-1)c} + \frac{(n-1)(n-2)}{2} e^{\delta(n^2-1)c})
    \\
    &\qquad
    +
\beta(\Delta\phi_{qr})
\Bigl(2(1-\alpha n) + 2 \alpha n e^{\delta(n-1)c}
    + n(n -1 +2\alpha) e^{\delta(n-1)^2c}
    \\
    &\qquad\qquad
    + \ab[4n^2-1 -2\alpha n (2n-1)] e^{\delta n(n-1)c}
    - \ab(2\alpha-1)n(n-1) e^{\delta(n^2-1)c}
    - n (1-2\alpha n) e^{\delta (n-1)^2 c}
\Bigr)
\biggr]
  \\
    &\coloneqq
    \frac{\delta e^{\delta(n-1)c}}{\ab(1-e^{\delta (n-1)c})^3}
    \ab[
  M B(0) + N \beta(\Delta \phi_{qr})]
  \end{aligned}
\end{equation}
Additionally, we can approximate
\begin{equation}
  \begin{aligned}
  &\frac{1}{1+\sum_j \ldots}
  = 1 -
    \frac{\delta e^{\delta(n-1)c}}{\ab(1-e^{\delta (n-1)c})^3}
    \ab[
  M B(0) + N \beta(\Delta \phi_{qr})]
  \end{aligned}
\end{equation}

Substituting this result into \cref{eq:phase_first-order_big-frac} gives
\begin{equation}
  \begin{aligned}
    \Phi^{(0)+(1)} &=
\delta(n-1)^2 c - \frac{n-2}{2} B(0) \\
  &\qquad +
\ln\ab(
\frac{\sum_{r=1}^{d}
  \ab[
  1 + \delta (n-1) n \beta(\Delta\phi_{qr}) (1 - 2 \alpha)]
  \ab\{
1 -
    \frac{\delta e^{\delta(n-1)c}}{\ab(1-e^{\delta (n-1)c})^3}
    \ab[
  M B(0) + N \beta(\Delta \phi_{qr})]
\}
                               }
{\sum_{r=1}^{d}
  \ab\{
1 -
    \frac{\delta e^{\delta(n-1)c}}{\ab(1-e^{\delta (n-1)c})^3}
    \ab[
  M B(0) + N \beta(\Delta \phi_{qr})]
\}
                  }
  )
  \\
  &=
\delta(n-1)^2 c - \frac{n-2}{2} B(0) \\
  &\qquad +
\ln\ab(
\frac{
  \begin{aligned}
&d\ab(1 -
    \frac{\delta e^{\delta(n-1)c}}{\ab(1-e^{\delta (n-1)c})^3}
  M B(0))
  \\
&\qquad-
  \delta (n-1) n (1 - 2 \alpha)
  \frac{\delta e^{\delta(n-1)c}}{\ab(1-e^{\delta (n-1)c})^3} N
  \sum_{r=1}^{d}
  \beta^2(\Delta\phi_{qr})
  \\
&\qquad+
  \ab[
  \delta (n-1) n (1 - 2 \alpha) \ab( 1 -
    \frac{\delta e^{\delta(n-1)c}}{\ab(1-e^{\delta (n-1)c})^3}
  M B(0))
  - \frac{\delta e^{\delta(n-1)c}}{\ab(1-e^{\delta (n-1)c})^3} N
  ]
  \sum_{r=1}^{d}
  \beta(\Delta \phi_{qr})
\end{aligned}
}
{d\ab(1 -
    \frac{\delta e^{\delta(n-1)c}}{\ab(1-e^{\delta (n-1)c})^3}
  M B(0))
  - N \frac{\delta e^{\delta(n-1)c}}{\ab(1-e^{\delta (n-1)c})^3}
  \sum_{r=1}^{d} \beta(\Delta \phi_{qr})
                  }
  )
\end{aligned}
\end{equation}
Dropping terms of order $\beta_0^2$ gives
\begin{equation}
  \begin{aligned}
    \Phi^{(0)+(1)} &=
\delta(n-1)^2 c - \frac{n-2}{2} B(0) \\
  &\qquad +
\ln\ab(
\frac{
  \begin{aligned}
&d\ab(1 -
    \frac{\delta e^{\delta(n-1)c}}{\ab(1-e^{\delta (n-1)c})^3}
  M B(0))
  \\
&\qquad+
  \ab[
-
    \frac{\delta e^{\delta(n-1)c}}{\ab(1-e^{\delta (n-1)c})^3} N
    +
\delta (n-1) n (1 - 2 \alpha)
]
  \sum_{r=1}^{d}
  \beta(\Delta \phi_{qr})
\end{aligned}
}
{d\ab(1 -
    \frac{\delta e^{\delta(n-1)c}}{\ab(1-e^{\delta (n-1)c})^3}
  M B(0))
  - N \frac{\delta e^{\delta(n-1)c}}{\ab(1-e^{\delta (n-1)c})^3}
  \sum_{r=1}^{d} \beta(\Delta \phi_{qr})
                  }
  )
\end{aligned}
\end{equation}

Note the following values:
\textbf{TODO: Prove $\sum_1^d \cos = 0$}
\begin{equation}
  \sum_{r=1}^d \beta(\Delta \phi) = \frac{d}{2} \beta_0
\end{equation}

Thus, we have
\begin{equation}
  \begin{aligned}
    \Phi^{(0)+(1)} &=
\delta(n-1)^2 c - \frac{n-2}{2} B(0) \\
  &\qquad +
\ln\ab(
\frac{
1 -
    \frac{\delta e^{\delta(n-1)c}}{\ab(1-e^{\delta (n-1)c})^3}
  M B(0)
+
\frac{1}{2}\beta_0\ab[
-
    \frac{\delta e^{\delta(n-1)c}}{\ab(1-e^{\delta (n-1)c})^3} N
    +
\delta (n-1) n (1 - 2 \alpha)
]
}
{1 -
    \frac{\delta e^{\delta(n-1)c}}{\ab(1-e^{\delta (n-1)c})^3}
  M B(0)
  - N \beta_0 \frac{\delta e^{\delta(n-1)c}}{2\ab(1-e^{\delta (n-1)c})^3}
                  }
  )
  \\
  &\approx
\delta(n-1)^2 c - \frac{n-2}{2} B(0) \\
  &\qquad +
\ln \biggl(
  1 + \frac{\beta_0}{2} \delta n (n-1) (1-2\alpha)
\biggr)
  \\
  &\approx
  \delta(n-1)^2 c - \frac{n-2}{2} B(0) +
  \frac{\beta_0}{2} \delta n (n-1) (1-2\alpha)
\end{aligned}
\label{eq:analytic_first_term}
\end{equation}
\textbf{TODO: We should be able to significantly shorten this
  derivation: it turns out, to first order, the $\rho_{NC}$
  numerator/denominator terms cancel and only the $\omega$ term remains}
\textbf{TODO: Can we analytically calculate where $\nu = 1$ (\ie{} where
  the inflection point is) and then expand about there to get a more
  accurate result than expanding about $B(0) = \beta_0 = 0$ (and hence,
  bad approximation for $\alpha \to 0$ which pushes $\nu=1$ to large
  $B(0)$}
