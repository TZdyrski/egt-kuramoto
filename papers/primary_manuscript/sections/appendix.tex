\section{Broken symmetry}
Similarly, the derivation for the communicative fraction
in section 4 of the appendix for \cite{tripp2022evolutionary} carries
through until equation 16.
Instead, with $n_{\text{in}} = n-1$, we have
\begin{equation}
  \begin{aligned}[b]
  \gamma_k &= \frac{p_{k,k-1}}{p_{k,k+1}} \\
           &= \frac{f_N(k)}{f_C(K)} \\
           &= \exp[\delta (n-1) (\pi_F(k) - \pi_E(k))]
\end{aligned}
\end{equation}
where strategy $F=(N,\phi_k)$, strategy $E=(C,\phi_j)$, and $k$ is the
number of communicative strategies $C$.
The new (edge-averaged) payoff functions $\pi_C$ and $\pi_N$ are given by
\begin{align}
  \pi_E(k) &= \ab(\frac{k-1}{n-1}) \ab(B(0) - c)
                + \ab(\frac{n-k}{n-1}) \ab(\beta(\Delta\phi) 2\alpha - c) \\
           &= \frac{1}{n-1}
               \ab(k \ab(B(\Delta\phi) - 2 \alpha \beta(\Delta\phi))
                 + 2 \alpha n \beta(\Delta\phi) - B(\Delta\phi) - (n-1) c)
\end{align}
and
\begin{align}
  \pi_F(k) &= \ab(\frac{k}{n-1}) \ab(\beta(\Delta\phi) 2 (1-\alpha))
                + \ab(\frac{n-k-1}{n-1}) \ab(0) \\
           &= \ab(\frac{k}{n-1}) 2 (1-\alpha) \beta(\Delta\phi)
\end{align}
where $\Delta \phi$ is the phase difference between the $E$ and $F$
strategies.
Note that $\pi_E$ has $B(0)$ instead of $B(\Delta \phi)$; this is
because the $B(\Delta \phi)$ payoff occurs when strategy $E$ plays
against strategy $E$, so $\Delta \phi = 0$ (recall we are only
considering a game with exactly two strategies, $E = (C, \phi_i)$ and
$F = (N, \phi_j)$.
Similar logic would apply to the $F$-$F$ strategy in $\pi_F$, but that
payoff is identically zero.

Thus, we have
\begin{equation}
  \begin{aligned}
    \gamma_k &= \exp\ab[\delta \ab(
    k 2 (1-\alpha) \beta(\Delta\phi)
    - k \ab(B(0) - 2 \alpha \beta(\Delta\phi))
                 - 2 \alpha n \beta(\Delta\phi) + B(0) + (n-1) c)] \\
      &= \exp\ab[\delta \ab(
    \ab(2 \beta(\Delta\phi) - B(0)) k
                 + B(0) - 2 \alpha n \beta(\Delta\phi) + (n-1) c)]
  \end{aligned}
\end{equation}
Next, we can calculate the ratio of fixation rates $\rho_F/\rho_E$ as
\begin{equation}
  \begin{aligned}
    \frac{\rho_F}{\rho_E} &= \prod_{k=1}^{n-1} \gamma_k \\
           &= \exp \ab\{
                \delta
                \sum_{k=1}^{n-1}
                \ab[
    \ab(2 \beta(\Delta\phi) - B(0)) k
                 + B(0) - 2 \alpha n \beta(\Delta\phi) + (n-1) c
                 ]
                 \} \\
           &= \exp \ab\{
                \delta
                \ab[
                \ab(2 \beta(\Delta\phi) - B(0)) \frac{n(n-1)}{2}
                 + \ab(B(0) - 2 \alpha n \beta(\Delta\phi) + (n-1) c) (n-1)
                 ]
                 \} \\
           &= \exp \ab\{
                \delta (n-1)
                \ab[
                (n-1) c + n \beta(\Delta\phi) (1 - 2 \alpha)
                - \frac{n-2}{2} B(0)
                 ]
                 \}
  \end{aligned}
\end{equation}
Notice, now the ratio of $\rho_F/\rho_E$ depends of $\Delta \phi$,
unlike in the $\alpha=1/2$ case.

Soon, we will require the expression for $\rho_{NC,\Delta \phi_{qr}}
\coloneqq \rho_E$.
Thus, we have
\begin{equation}
  \begin{aligned}
    \rho_E &= \frac{1}{1+\sum_{j=1}^{n-1} \prod_{k=1}^j \gamma_k} \\
           &= \ab(1+\sum_{j=1}^{n-1} \prod_{k=1}^j \exp\ab[\delta \ab(
           \ab(2 \beta(\Delta\phi_{qr}) - B(0)) k
           + B(0) - 2 \alpha n \beta(\Delta\phi_{qr}) + (n-1) c)])^{-1}
                 \\
           &= \ab(1+\sum_{j=1}^{n-1} \exp\ab[\delta \sum_{k=1}^j \ab(
           \ab(2 \beta(\Delta\phi_{qr}) - B(0)) k
           + B(0) - 2 \alpha n \beta(\Delta\phi_{qr}) + (n-1) c)])^{-1}
                 \\
           &= \ab(1+\sum_{j=1}^{n-1} \exp\ab[\delta \ab(
           \ab(2 \beta(\Delta\phi_{qr}) - B(0)) j(j+1)/2
           + j \ab(B(0) - 2 \alpha n \beta(\Delta\phi_{qr}) + (n-1) c))])^{-1}
                 \\
           &= \ab(1+\sum_{j=1}^{n-1} \exp\ab[\delta \ab(
           \ab(\beta(\Delta\phi_{qr}) - B(0)/2) j^2
           + j \ab(B(0)/2 + \beta(\Delta\phi_{qr}) (1 - 2 \alpha n)  + (n-1) c))])^{-1}
  \end{aligned}
\end{equation}

Then, for the multi-population, low-mutation case, we identify
$\rho_{NC,\Delta \phi_{qr}} = \rho_E$ and $\rho_{CN,\Delta \phi_{qr}} = \rho_F$
Then, we can use this to calculate the ratio of the stationary state
eigenvalues $s_1/s_2$
\begin{align}
  \frac{s_2}{s_1} &= \frac
    {\sum_{r=1}^{m} \rho_{CN,\Delta \phi_{qr}}}
    {\sum_{r=1}^{m} \rho_{NC,\Delta \phi_{qr}} }
  \\
  &=
  \frac
  {\sum_{r=1}^{m} \rho_{NC,\Delta \phi_{qr}}
    \exp \Bab{
      \delta (n-1)
      \bab{
        (n-1) c + n \beta(\Delta \phi_{qr}) (1 - 2 \alpha)
        - \frac{n-2}{2} B(0)
      }
    }
  }
  {\sum_{r=1}^{m} \rho_{NC,\Delta \phi_{qr}}}
  \\
  &=
  \bab{
    \frac
    {\sum_{r=1}^{m} \frac
      {\exp \Bab{ \delta (n-1)
        \bab{
          (n-1) c + n \beta(\Delta \phi_{qr}) (1 - 2 \alpha) - \frac{n-2}{2} B(0)
        }}
      }
      {1 + \sum_{j=1}^{n-1} \exp\Bab{
        \delta \bab{
         \pab{\beta(\Delta \phi_{qr}) - B(0)/2} j^2
         + j \pab{B(0)/2 + \beta(\Delta \phi_{qr}) (1 - 2 \alpha n) + (n-1) c}
        }}
      }
    }
    {\sum_{r=1}^{m} \frac{1}
      {1 + \sum_{j=1}^{n-1} \exp\Bab{
        \delta \bab{
         \pab{\beta(\Delta \phi_{qr}) - B(0)/2} j^2
         + j \pab{B(0)/2 + \beta(\Delta \phi_{qr}) (1 - 2 \alpha n) + (n-1) c}
        }}
      }
    }
  }
  \label{eq:full_analytic_frac}
\end{align}
where $m = 20$ is the number of phases that $\phi_i$ can take.
Unlike in the $\alpha = 1/2$ symmetric case, we cannot factor the
exponential component out of the sum and cancel the $\rho_{CN,\Delta
\phi_{qr}}$ terms.
However, we can asymptotically expand $\Phi(B(0))$ for small $B(0) \ll 1$;
additionally, since all of our simulations use $\beta \propto B$,
we also assume $\beta \ll 1$.
Furthermore, we note that the only $\Delta \phi$ dependence
appears in the $\beta$ terms.
Finally, to simplify the calculation,
we define the asymptotic expansion of
$\rho_{NC,\Delta \phi_{qr}} \coloneqq
\rho_{NC,\Delta \phi_{qr}}^{(0)}
+
\rho_{NC,\Delta \phi_{qr}}^{(1)}
+
\order{B_0^2, \beta^2, B_0 \beta}
$
where
$
\rho_{NC,\Delta \phi_{qr}}^{(i)}
$
depends only on terms of the form $B_0^j \beta^k$
with $j+k=i$.

Then, to first order in $B_0$ and $\beta$,
we have
\begin{align*}
  \frac{s_2}{s_1}
  &=
  \frac
  {\sum_{r=1}^{m} \rho_{NC,\Delta \phi_{qr}}
    \exp \Bab{
      \delta (n-1)
      \bab{
        (n-1) c + n \beta(\Delta \phi_{qr}) (1 - 2 \alpha)
        - \frac{n-2}{2} B(0)
      }
    }
  }
  {\sum_{r=1}^{m} \rho_{NC,\Delta \phi_{qr}}}
  \\
  &=
  e^{\delta (n-1)^2 c}
  \frac
  {\sum_{r=1}^{m}
    \pab{
      \rho_{NC,\Delta \phi_{qr}}^{(0)}
      +
      \rho_{NC,\Delta \phi_{qr}}^{(1)}
    }
    \Bab{1 +
      \delta (n-1)
      \bab{
        n \beta(\Delta \phi_{qr}) (1 - 2 \alpha)
        - \frac{n-2}{2} B(0)
      }
    }
  }
  {\sum_{r=1}^{m} \pab{
    \rho_{NC,\Delta \phi_{qr}}^{(0)}
    +
    \rho_{NC,\Delta \phi_{qr}}^{(1)}
  }}
  + \order{B_0^2}
  \\
  &=
  e^{\delta (n-1)^2 c}
  \Biggl\{
  \frac
  {\sum_{r=1}^{m}
    \pab{
      \rho_{NC,\Delta \phi_{qr}}^{(0)}
      +
      \rho_{NC,\Delta \phi_{qr}}^{(1)}
    }
  }
  {\sum_{r=1}^{m} \pab{
    \rho_{NC,\Delta \phi_{qr}}^{(0)}
    +
    \rho_{NC,\Delta \phi_{qr}}^{(1)}
  }}
  \\
  &\qquad
  +
  \frac
  {\sum_{r=1}^{m} \rho_{NC,\Delta \phi_{qr}}^{(0)}
    \delta (n-1)
    \bab{
      n \beta(\Delta \phi_{qr}) (1 - 2 \alpha)
      - \frac{n-2}{2} B(0)
    }
  }
  {\sum_{r=1}^{m} \rho_{NC,\Delta \phi_{qr}}^{(0)}
  }
  \Biggr\}
  + \order{B_0^2}
  \\
  &=
  e^{\delta (n-1)^2 c}
  \Bab{
    1
    - \delta (n-1) \frac{n-2}{2} B(0)
    +
    \frac{\delta n (n-1)}{m} (1 - 2 \alpha)
    \sum_{r=1}^{m} \beta(\Delta \phi_{qr})
  }
  + \order{B_0^2}
  \\
  &=
  \exp \Bab{\delta (n-1) \bab{
    (n-1) c
    - \frac{n-2}{2} B(0)
    +
    \frac{n (1 - 2 \alpha)}{m}
    \sum_{r=1}^{m} \beta(\Delta \phi_{qr})
  }}
  + \order{B_0^2}
  \\
  &=
  \exp \Bab{ \delta (n-1) \bab{
    (n-1) c
    - \frac{n-2}{2} B(0)
    +
    \frac{n (1 - 2 \alpha)}{2} \beta_0
  }}
  + \order{B_0^2}
\end{align*}
Where, in the last line, we used
the definition of
$\beta(\Delta \phi_{qr}) = \beta_0 \bab{1 + \cos(\Delta \phi_{qr})}/2$
to write
$\sum_{r=1}^m \beta(\Delta \phi_{qr})
= \beta_0 \sum_{r=1}^m \Bab{1 + \cos\bab{2 \pi (q-r)/m}}/2
= m \beta_0/2$
since the cosine sum gives zero.

Finally, using the fact that $d s_1 + d s_2 = 1$, we find the probability
of communicative fixation $d s_1$ to be
\begin{equation}
  d s_1 = \frac{1}{1 + s_2/s_1}
  \label{eq:full_analytic}
\end{equation}
with $s_2/s_1$ given by \cref{eq:full_analytic_frac},
and the asymptotic, small $B_0/c$ approximation given by
\begin{equation}
  d s_1 = \frac{1}{1 + s_2/s_1}
  \approx
  \frac{1}
  {1 + \exp \Bab{\delta (n-1) \bab{
    (n-1) c
    - \frac{n-2}{2} B(0)
    +
    \frac{n (1 - 2 \alpha)}{2} \beta_0
    }}
  }
  \label{eq:analytic_first_term}
\end{equation}
